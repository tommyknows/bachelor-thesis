% -*- mode: latex; coding: utf-8; TeX-master: ../thesis -*-
% !TEX TS-program = pdflatexmk
% !TEX encoding = UTF-8 Unicode
% !TEX root = ../thesis.tex

The aforementioned extensions to the Go language and its tooling should be a help to learn
functional programming. I believe that through these extensions it is easier to write
purely functional code in Go, enabling a developer to learn functional programming with
a familiar syntax in an obvious way. Here, Go's simplicity and verbosity are a key differentiator
to other languages. Instead of having as many features as possible to support every usecase,
Go has been designed with simplicity in mind\footnote{For example, Go only has 25 keywords, compared
to 37 in C99 and 84 in C++11}.

In many cases, this leads to more `verbose' code --- more lines of code compared to a similar
implementation in other languages. However, I argue that, especially for the first steps
in functional programming,

\begin{quote}
Clear is better than clever\autocite{cheney-clear}
\end{quote}

Staying in touch with this core Go principle, this results in functional code that may be
verbose, but easy to read and understand.

It should be clear that the result is not a `production-ready' functional programming language.
It is a language to help getting started with functional programming; either by re-implemeting pieces
of code that have not been clear in how they work, or by taking an imperative block of code
and refactoring it to make it purely functional.

In many cases, the resulting code will still look familiar to the imperative counterpart,
even if `funcheck' assures that it is purely functional. This, I believe, bridges the gap
that developers usually have to overcome by themselves.

To be a purely functional programming language, Go is missing too many features that would be
required to write concise functional code. The very basic type system\footnote{Not only
	does Go not have polymorphism (yet), Go's type system is simple by design: there are no
	implicit type conversions, no sum types (tagged unions, variant) and almost no type inference.
}, no advanced pattern matching and only explicit currying are all examples why Go is not useful
in day-to-day functional programming.

At the same time, the obvious nature of Go is exactly because it is missing all
of these features. The Go team explicitly tries not to include too many features within
the language in order to keep the complexity of code to a minimum\autocite{go-feature}.
The simplicity of the language is a key feature of Go and an important reason why it was
chosen to implement the ideas in the first place.
Especially for learning new concepts, hiding implementations and ideas behind features
may not be what is desired and helpful.

On another note, what has not been an aspect in this thesis is
performance. Go by itself is relatively performant, however functional constructs, for example
recursive function calls, come with a performance cost. While in purely functional languages
this can be optimised, Go cannot or does not want to do these optimisations\footnote{For example,
with tail call optimisation, the Go team explicitly decided not to do it because the stack trace
would be lost\autocite{go-tco-nope}}.

While the newly built-in functions do not have any low-hanging fruits in regards to optimisation,
they would benefit from more aggressive inlining\autocite{go-compiler-inline}, for example.
Benchmarking and optimising these functions was out of scope for this thesis, but may be tackled
in the future.

\newglossaryentry{sumtypes}{name=sum types,description={Sum types, often also called aggregated types,
variant or tagged union is a data structure that can hold one of several, predefined data types. For
example, Haskell's \mintinline{haskell}|Either| holds either a value of type A or type B. Similar to that,
\mintinline{haskell}|Maybe| can hold either a concrete value, or `Nothing'}}

The number one issue that still exists is the simple type system. Not only the lack of polymorphism, which
has been mitigated slightly by providing the most used higher-order functions as built-ins, but also the
lack of algebraic data types, especially \gls{sumtypes}.

Algebraic data types can be split up into two groups, product types and sum types.
Most product types can be built with Go too; records are basically
equal to structs, and tuples are not needed too often, as functions can just return multiple values.
Sum types however are not available in Go at all. It is possible to imitate sum types in Go
with interfaces (see the example code in~\ref{code:funcexample}), but the compiler does not ensure that
all cases are matched against\footnote{There is tooling available to check this\autocite{sushi-sumtypes},
	but it is third party and not built into the language}. This may be an interesting area for further
research and implementation possibilities.
