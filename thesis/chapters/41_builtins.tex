\subsection{Required Steps}

Adding a builtin function to the Go language requires a few more steps than just
adding support within the compiler. While it would technically be enough to
support the translation between Go code and the compiled binary, there would be
no visibility for a developer that there is a function that could be used.
For a complete implementation, the following steps are necessary:
\begin{itemize}
    \item Adding the GoDoc\autocite{godoc} that describes the function and it's usage
    \item Adding type-checking support in external packages for tools like
    Gopls\autocite{gopls}
    \item Adding the implementation within the internal\footnote{
            ``An import of
            a path containing the element “internal” is disallowed if the
            importing code is outside the tree rooted at the parent of the
            `internal' directory.''\autocite{internal-packages}
        }
        package of the compiler
        \begin{itemize}
            \item Adding the \gls{ast} node type
            \item Adding type-checking for that node type
            \item Adding the AST traversal for that node type, translating it
                to AST nodes that the compiler already knows and can translate
                to builtin runtime-calls or \gls{ssa}
        \end{itemize}
\end{itemize}

The go source code that is relevant for this thesis can be classified into three different
types. One is the godoc - the documentation for the new built-in functions. The
other two are the `public' and the `private' implementation of these builtins.

% TODO: this could need a rewrite
The `private' implementation is everything that is located within the
\textit{src/cmd/compile/internal} package\autocite{internal-packages}. It can only
be used by the package in \textit{src/cmd/compile}, which contains the
implementation of the compiler itself. When calling \mintinline{shell}|go build .|,
the compiler is invoked indirectly. To directly invoke the compiler,
\mintinline{shell}|go tool compile| can be used. The compile tool gets compiled
from the main package located in \textit{src/cmd/compile}, which in turn
uses the internal package.

\newglossaryentry{gopls}{name=Gopls,description={Gopls is go's official language
server implementation\autocite{gopls}}}

Everything that is not in \textit{src/cmd/compile} is referred to as the `public'
part of the compiler in this thesis. The `public' parts are used by external
tools, for example \gls{gopls}, for type-checking, source code validation and
analysis.

\subsection{Adding the GoDoc}
In Go, documentation is generated directly from comments within the source code
\autocite{godoc}. This also applies to builtin functions in the compiler, which
have a function stub to document their behaviour\autocite{godoc-builtin}, but
no implementation, as that is done in the compiler\autocite{builtin-impl}.

The documentation for builtins should be as short and precise as possible.
The usage of `Type' and `Type1' has been decided based on other builtins
like `append' and 'delete'.
The function headers are derived from their Haskell counterparts, adjusted
to the Go nomenclature.

\begin{code}
    \captionof{listing}{Godoc for the new built-in functions}
    \gofilerange{../work/go/src/builtin/builtin.go}{begin-newbuiltins}{end-newbuiltins}
\end{code}
\subsection{Public packages}

\begin{quote}
Note that the `go/*` family of packages, such as `go/parser` and `go/types`,
have no relation to the compiler. Since the compiler was initially written in C,
the `go/*` packages were developed to enable writing tools working with Go code,
such as `gofmt` and `vet`.\autocite{compiler-readme}
\end{quote}

To enable tooling support for the new built-in functions, they have to be
registered in the `go/*' packages. The only package that is affected by new
builtins is `go/types'.

In the `types' package, the builtins have to be registered as such and as
`predeclared' functions:

\begin{code}
    \captionof{listing}{Registering new built-in functions}
    \gofilerange{../work/go/src/go/types/universe.go}{start-builtin}{end-builtin}
    \gofilerange{../work/go/src/go/types/universe.go}{start-predeclared}{end-predeclared}
\end{code}
This registration defines the type of the built-in - they are all expressions,
as they return a value - and the number of arguments.
After that, the type-checking and its associated tests can be implemented.
% TODO: reference the implementation

This concludes the type-checking for external tools and makes `gopls' return errors
Once type-checking the new built-in functions is implemented, `gopls' can be
compiled against the new public packages.\footnote{
This means pointing the go toolchain to the correct directory by setting
the value of `GOROOT' with \mintinline{shell}|go env -w GOROOT=<path>|.
}
It will then return errors if the wrong types are used. For example, when trying
to prepend an integer to a string slice:

\begin{gocode}
package main

import "fmt"

func main() {
    fmt.Println(prepend(3, []string{"hello", "world"}))
}
\end{gocode}

Gopls will report a type-checking error:
\begin{bashcode}
$ gopls check main.go
/tmp/playground/main.go:6:22-23: cannot convert 3 (untyped int constant) to string
\end{bashcode}

\subsection{Private packages}

In the private packages - the actual compiler - the expressions have to be
type-checked, ordered and transformed.

The type-checking process is similar to the one executed for external tools.
It should also check the node's child nodes, meaning an operations
arguments, body and init statements. Furthermore, during the type-checking
process, the built-in function's return types are set and node types
may be converted, if possible and necessary.
An operation may expect it's arguments to be in \mintinline{go}|node.Left|
and \mintinline{go}|node.Right|, which means type-checking will also need
to move the argument nodes from their default location in
\mintinline{go}|node.List| to \mintinline{go}|node.Left| and
\mintinline{go}|node.Right|.

Ordering ensures the evaluation order and re-orders expressions. All of
the new built-in functions will be evaluated left-to-right and there are now
special cases to handle.

Transforming means changing the AST nodes from the built-in operation to
nodes that the compiler knows how to translate to SSA. The actual algorithm
that these functions use cannot be implemented in normal Go code, they have to be
translated directly to AST nodes and statements.

There are more steps to compiling Go code, for example escape-checking,
SSA conversion and a lot of optimisations. These are not necessary to
implement and do not have a direct relation to the new built-ins, which
is why these steps are elided in this paper.

The actual algorithms and part of the implementations for the builtin
functions are covered in the following chapters.
\footnote{
    The full implementations can be viewed by diff-ing the git repository
    between the references `bachelor-thesis' and `go1.14'\autocite{ba-go1-14-thesis-diff}.
}

\subsubsection{fmap}

To make the implementation in the AST easier, the algorithm will first be
developed in Go, and then translated. Implementing fmap in Go is relatively
simple:

\begin{code}
    \captionof{listing}{Fmap implementation in Go}
    \label{code:fmap-go}
    \begin{gocode}
func fmap(fn func(Type) Type1, src []Type) (dest []Type1) {
    for _, elem := range src {
        dest = append(dest, fn(elem))
    }
    return dest
}
\end{gocode}
\end{code}
However, there is room for improvement within that function. Instead
of calling \mintinline{go}|append| at every iteration of the loop, the slice can
be allocated with \mintinline{go}|make| at the beginning of the function. Thus,
calls to grow the slice at runtime can be saved.

\begin{code}
    \captionof{listing}{Improved implementation of fmap}
    \label{code:fmap-go-improved}
    \begin{gocode}
func fmap(fn func(Type) Type1, src []Type) []Type1 {
    dest := make([]Type1, len(src))
    for i, elem := range src {
        dest[i] = fn(elem)
    }
    return dest
}
    \end{gocode}
\end{code}
This algorithm can be translated to the following AST node:

\begin{code}
    \captionof{listing}{fmap AST translation\autocite{fmap-walk-implementation}}
    \gofilerange{../work/go/src/cmd/compile/internal/gc/walk.go}{start-fmap-header}{end-fmap-header}
\end{code}
\subsubsection{prepend}

The general algorithm for `prepend' is:
\begin{code}
    \captionof{listing}{prepend implementation in Go}
    \begin{gocode}
func prepend(elem Type, slice []Type) []Type {
    dest := make([]Type, 1, len(src)+1)
    dest[0] = elem
    return append(dest, slice...)
}
    \end{gocode}
\end{code}
The call to \mintinline{go}|make(...)| creates a slice with the length of 1 and the capacity
to hold all elements of the source slice, plus one. By allocating the slice with the full
length, another slice allocation within the call to \mintinline{go}|append(...)| is saved.
The element to prepend is added as the first element of the slice, and append will then
copy the `src' slice into `dest'.

The implementation within `walkprepend' reflects these lines of Go code, but
as AST nodes:

\begin{code}
    \captionof{listing}{prepend AST translation\autocite{prepend-walk-implementation}}
    \gofilerange{../work/go/src/cmd/compile/internal/gc/walk.go}{start-prepend-header}{end-prepend-header}
\end{code}
\subsubsection{foldr and foldl}

As outlined in Chapter~\ref{sec:fold}, there will be two fold functions;
foldr and foldl. foldr behaves exactly like its Haskell counterpart,
while foldl behaves like foldl' in Haskell.

While the fold algorithms are most obvious when using recursion, due to
performance considerations, an imperative implementation has been chosen:

\begin{code}
    \captionof{listing}{fold implementation in Go}
    \begin{gocode}
func foldr(fn func(Type, Type1) Type1, acc Type1, slice []Type) Type1 {
    for i := len(s) - 1; i >= 0; i-- {
        acc = fn(s[i], acc)
    }
    return acc
}

func foldl(fn func(Type1, Type) Type1, acc Type1, slice []Type) Type1 {
    for i := 0; i < len(s); i++ {
        acc = f(acc, s[i])
    }
    return acc
}
\end{gocode}
\end{code}
The code further clarifies the differences between the two different folds;
the slice is processed in reverse order for foldr (as it would be if this
algorithm would have been implemented with recursion), and the order of
arguments to the fold function is switched.

The AST walk translates fold to:
\begin{code}
    \captionof{listing}{fold AST translation\autocite{fold-walk-implementation}}
    \gofilerange{../work/go/src/cmd/compile/internal/gc/walk.go}{start-fold-header}{end-fold-header}
\end{code}
\subsubsection{filter}

Being a slice-manipulating function, filter also needs to traverse the whole
slice in a for-loop. However, compared to the other newly built-in functions,
the size for the target slice is unknown until all items have been traversed,
which is why filter does not allow for the same optimisations as the other
functions.

\begin{code}
    \captionof{listing}{filter implementation in Go}
    \begin{gocode}
func filter(f func(Type) bool, s []Type) []Type {
    var dst []Type
    for i := range s {
            if f(s) {
                dst = append(dst, s[i])
            }
    }
}
    \end{gocode}
\end{code}
And the same algorithm, but translated to AST statements:

\begin{code}
    \captionof{listing}{filter AST translation\autocite{filter-walk-implementation}}
    \gofilerange{../work/go/src/cmd/compile/internal/gc/walk.go}{start-filter-header}{end-filter-header}
\end{code}
