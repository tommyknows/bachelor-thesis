To learn functional programming without a purely functional language, the
developer needs to know which statements are functional and which
would not exist or be possible in a purely functional language.
For this reason, a `functional checker' should be created. It should work
in a similar fashion to linters like `shellcheck', `go vet' or `gosimple'\footnote{
    A list of Go linters can be found in golangci-lint\autocite{golangci-lint}.
}, with different `rules' that are reported upon. The name for this tool
will be `funcheck' because functional programming should be fun. % TODO: seriously?

A set of rules should be compiled to identify common `non-functional' constructs
which should then be reported upon.

The first step is to define a set of rules.

\subsection{Functional Purity}

The goal of funcheck is to report on everything that is not purely
functional. However, there is no agreed upon definition\autocite{functional-controversy}.
In the context of this thesis, the following definition will be used:

\begin{quote}
    A style of building the structure and elements of computer programs that
    treats all computation as the evaluation of mathematical functions. Purely
    functional programming may also be defined by forbidding changing-state and
    mutable data.

    Purely functional programming consists in ensuring that functions, inside
    the functional paradigm, will only depend on their arguments, regardless of
    any global or local state.\autocite{functional-purity-wiki}
\end{quote}

This definition roughly brakes down into two parts; immutability and function
purity. These parts and how they translate to Go will be discussed in the
following chapters.

\subsection{Forbidding mutability and changing state}\label{sec:mutability}

\newglossaryentry{copy-by-value}{name=copy by value, description={Copy by value
        refers to the argument-passing style where the supplied argumens are
        copied. To pass references in Go, the developer needs to use pointers
instead}}

Immutability refers to objects --- variables --- that are unchangable. This means
that their state cannot be modified after it's creation and first assignment.

Many programming languages provide features for setting variables as immutable.

In Rust for example, variables are immutable by default. To explicitely declare
a mutable variable, the `mut' keyword has to be used: \mintinline{rust}|let mut x = 5;|.

Java, in contrast, uses the \mintinline{java}|final| keyword:
\begin{quote}
    A final variable can only be initialized once\autocite{final-java}
\end{quote}
`Only be initialized once' means that it cannot be reassigned later, effectively
making that variable immutable.
A caveat with Java's \mintinline{java}|final| keyword is that the immutability
is only tied to the reference of that object, not
it's contents (one may still change a field of an immutable object).

C and C++ have two features to achieve immutability, the \mintinline{c}|#define|
preprocessor and the \mintinline{c}|const| keyword.
The \mintinline{c}|#define| directive does not declare variables, but rather
works in a similar way to string replacement at compile time. These directives
can also be expressions. For example, this is a valid define statement:
\begin{ccode}
#define AGE (20/2)
\end{ccode}

However, because \mintinline{c}|#define| is just text substitution, these
identifiers are untyped.

The \mintinline{c}|const| qualifier specifies that a variable's value
will not be changed, making it immutable. This is effectively the equivalent
to Java's \mintinline{java}|final|.

Go has the \mintinline{go}|const| keyword too, and similar to C, constants can only be
characters, strings, booleans or numeric values\footnote{In contrast to C, Go does
have a boolean type}. A complex type --- struct, interface, function --- cannot be constant.

This means that the programmer cannot write
\mintinline{go}|const x = MyStruct{A: a, B: b}| to make a struct immutable,
as a struct is a complex type. Therefore, the solution to making variables
immutable is to explicitely not allow any mutations in a program\footnote{
    Although this may not be a very strict definition of immutability and doesn't
    allow for compiler optimisations or similar, it is enough in the context of
learning functional programming and learning that variables cannot be modified.}.
While this incorporates a lot of different aspects, the simplest solution
to that is to disallow the assignment operator \mintinline{go}|=|, and only
allow it in combination with a declaration (\mintinline{go}|:=|).

This solution also has the side-effect that it makes it impossible to mutate
existing, complex data structures like maps and slices\footnote{By not
    allowing assignments, elements can not be updated by
\mintinline{go}|s[0] = "zero"| / \mintinline{go}|m["key"] = "value"|.}.

There are additional operators that work in a similar way to the assignment
operator, for example \mintinline{go}|++|, \mintinline{go}|--| and
\mintinline{go}|x += y|. These are all `hidden' assignments and will need to
be reported upon.

Go being a \gls{copy-by-value} language, mutating existing variables
    accross functions works by passing pointers to these variables instead\footnote{
        An example for this can be found at~\ref{appendix:mutation}.}.
When removing the regular assignment operators, the usage of pointers becomes
second nature. Technically, they could be used everywhere now, as it is not
possible to mutate anything either way. It is left to the developer to decide,
but it is not necessary to use pointers anymore. Functional languages, in contrast,
do not provide the option to address variables, since this is merely an optimisation
that is left to the compiler.

A feature not discussed so far is shadowing. Shadowing a variable in a different
scope is still possible by redeclaring it. As expected, even with the above
mentioned limitations, the rules for shadowing variables do not change. This
can be seen in Appendix~\ref{appendix:shadowing}.

\subsection{Pure functions}

A pure function is a function where the return value is only dependent on
the arguments. With the same arguments, the function should always return
the same values. The evaluation of the function is also not allowed to have
any `side effects', meaning that it should not mutate non-local variables or
references\footnote{This also includes `local static' variables, but Go does
not have a way to make variables `static'.}.

As discussed in the last Chapter~\ref{sec:mutability}, if re-assignments
are not allowed, mutating global variables is not possible. The same logic
can be followed with variables passed by reference (pointers).

If global state cannot be changed, it also cannot have a dynamic influence
over a function's return values.

As such, forbidding re-assignments is an extremely simple solution to
ensure functional purity.

\subsubsection{A Note about IO}
The only thing that is left is interacting with the outside world; network,
files, user input and sensors, to name a few. Haskell's IO Package ensures
that these technically impure functions work the way one would expect.
There are a few issues with IO in a pure function context. For example,
the following function to write a string to a specified file:

\begin{haskellcode}
    writeFile :: String -> String
\end{haskellcode}

If the programmers calls this function
\mintinline{haskell}|writeFile "hello.txt" "Hello World"|, there are
no return values anymore, As such, the compiler would simply omit the call
to this `writeFile' function completely. The rationale behind this is that
it expects all functions to be pure, and as such, a function with no return
value will not do anything meaningful and can be omitted.

Similarly, for getting user input:
\begin{haskellcode}
    getChar :: Char
\end{haskellcode}

If this function is called multiple times, the compiler may reorder the
execution of these calls, again, because it expects the functions to be
pure, so the execution order should not matter:
\begin{haskellcode}
    get2Chars = [getChar, getChar]
\end{haskellcode}

However, in this example, execution order is extremely important. To
solve this problem, Haskell introduced the IO monad, in which they
wrap all functions with a `RealWorld' argument. So the
\mintinline{haskell}|getChar| example from before gets rewritten as

\begin{haskellcode}
    getChar :: RealWorld -> (Char, RealWorld)
\end{haskellcode}

This can be read as `take the current RealWorld, do something with it,
and return a Char and a (possibly changed) RealWorld'\autocite{haskell-io}.

With this solution, the compiler cannot reorder or omit IO actions, ensuring
the correct execution of the program.

The IO Monad is a sophisticated solution to a problem that exists because of
compiler optimisations. These can only be guaranteed to be correct because
the compiler knows (or rather expects) that all functions are pure.

In Go, the compiler does not and can not make this assumption. There also
is no `pure' keyword to mark a function as such. For that reason, the
compiler cannot optimise as aggressivly, and the whole IO problem does
not exist\footnote{Although when nesting calls, one may need to be careful about
execution order.}.

\subsection{Assignments in Go}

As desribed in Section~\ref{sec:mutability}, the only check that needs
to be done is that there are no re-assignments in the code. There are a few
exceptions, which will be covered later. First, it is important to have
a few examples and test cases to be precise on what should be reported.

To declare variables in Go, the \mintinline{go}|var| keyword is used.
The var keyword needs to be followed by one or more identifiers and maybe
types or the initial value. These are all valid declarations:

\begin{code}
    \captionof{listing}{Go Variable Declarations}
    \begin{gocode}
var i int
var U, V, W float64
var k = 1
var x, y float32 = -1, -2
    \end{gocode}
\end{code}
What can be seen in this example is that the type of the variable can be
deducted automatically\footnote{The compiler will infer the type at compile
    time. That operation is always successful, although it may not be what
    the programmer desires. For example, \mintinline{go}|var x = 5| will
always result in \mintinline{go}|x| to be of type \mintinline{go}|int|.},
or be specified explicitely.

However, the notation \mintinline{go}|var k = 1| is seldomely seen. This
is due to the fact that there is a second way to declare and initialise
variables to a specific value, the `Short variable declaration' syntax:

\begin{gocode}
    k := 1
\end{gocode}

\begin{quote}
    It is shorthand for a regular variable declaration with initializer expressions but no types:

    \begin{bnfcode}
"var" IdentifierList = ExpressionList .
    \end{bnfcode}
\end{quote}\autocite{short-hand-decl}

In practice, the short variable declaration is used more often due to the
fact that there is no need to specify the type of the variable manually and
it is less verbose than its counterpart \mintinline{go}|var x = y|.

Go also has increment, decrement and various other operators to re-assign
variables. Most of the operators that take a `left' and `right' expression
also have a short-hand notation to re-assign the value back to the variable,
for example:

\begin{code}
    \captionof{listing}{Go Assignment Operators}
\begin{gocode}
    var x = 5
    x += 5 // equal to x = x + 5
    x %= 3 // equal to x = x % 3
    x <<= 2 // equal to x = x << 2 (bitshift)
\end{gocode}
\end{code}\footnote{Non-exhaustive list of operators. A complete listing
of operators can be found in the Go language spec\autocite{spec-operators}.}

All these operators are re-assigning values and are not available in functional
programming.

However, there are a few exceptions to this rule, which are covered in the
following chapters.

\subsubsection{Function Assignments}

A variable declaration brings an identifier into scope. The Go Language Specification
defines this scope according to the following rule:
\begin{quote}
    The scope of a constant or variable identifier declared inside a function begins
    at the end of the ConstSpec or VarSpec (ShortVarDecl for short variable
    declarations) and ends at the end of the innermost containing block.
\autocite{spec-scope}\end{quote}

This definition holds an important caveat for function definitions:
\begin{quote}
    The scope [...] begins \textbf{at the end} of the ConstSpec or VarSpec [...].
\end{quote}

This means that when defining a function, the functions identifier is not
in scope within that function:

\begin{code}
    \captionof{listing}{Go scoping issue with recursive functions}
\begin{gocode}
func X() {
    f := func() {
        f() // <- 'f' is not in scope yet.
    }
}
\end{gocode}
\end{code}
In the above example, the function `f' cannot be called recursively. This is due
to the fact that the identifier `f' is not in scope at f's definition. Instead,
it only enters scope at the closing brace of function f.

This cannot be changed within the compiler without touching a lot of the scoping
rules. Changing these scoping rules may have unintended side effects. A naive
solution to the issue would be to bring the identifier into scope right after the
assignment operator (\mintinline{go}|=|). However, this would introduce further
edge cases, as for example \mintinline{go}|x := x|, and extra measures would have
to be taken to make such constructs illegal again.

However, there is a simple, although verbose solution to this problem: The function
has to be declared in advance:

\begin{code}
    \captionof{listing}{Fixing the scope issue on recursive functions}
    \begin{gocode}
func X() {
    var f func() // f enters scope after this line
    f = func() {
        f() // this is allowed, as f is in scope
    }
}
\end{gocode}
\end{code}
This exception will need to be allowed by the assignment checker, as recursive functions
are one of the building blocks in functional programming.

\subsubsection{Multiple Assignments}

\newglossaryentry{ebnf}{name=Extended Backus-Naur Form, description={An extended version of the `Backus-Naur
Form, a notation technique to describe programming language syntax}}

As discussed at the beginning of this Chapter, the short variable declaration brings
a variable into scope and assigns it to a given expression (value). It is also
possible to declare multiple variables at once, as can be seen in the \gls{ebnf}
notation:
\begin{code}
    \begin{bnfcode}
ShortVarDecl = IdentifierList ":=" ExpressionList .
    \end{bnfcode}
\end{code}
This clearly shows that both left- and right-hand side take a list, meaning one or more
elements, making a statement like \mintinline{go}|x, y := 1, 2| valid\footnote{The
\mintinline{go}|var| keyword also takes lists, so this is possible too}.

However, there is an important note to be found in the language specification:

\begin{quote}
Unlike regular variable declarations, a short variable declaration may redeclare
variables provided they were originally declared earlier in the same block (or
the parameter lists if the block is the function body) with the same type, and
at least one of the non-blank\footnote{The blank identifier (\mintinline{go}|_|)
    discards a value; \mintinline{go}|x, _ := f()|} variables is new. As a consequence, redeclaration
can only appear in a multi-variable short declaration. Redeclaration does not
introduce a new variable; it \textbf{just assigns a new value to the original}.
    \begin{gocode}
field1, offset := nextField(str, 0)
field2, offset := nextField(str, offset)  // redeclares offset
    \end{gocode}
\autocite{short-hand-decl}
\end{quote}

This should not be allowed, as the programmer can introduce mutation by redeclaring
already existing variables in a short variable declaration.
