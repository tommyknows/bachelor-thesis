\section{Implementing the new built-in functions}

\subsection{Required Steps}

Adding a builtin function to the Go language requires a few more steps than just
adding support within the compiler. While it would technically be enough to
support the translation between Go code and the compiled binary, there would be
no visibility for a developer that there is a function that could be used.
For a complete implementation, the following steps are necessary:
\begin{itemize}
    \item Adding the GoDoc\autocite{godoc} that describes the function and it's usage
    \item Adding type-checking support in external packages for tools like
        Gopls\footnote{Gopls is Go's official language server implementation\autocite{gopls}.}
    \item Adding the implementation within the internal\footnote{
            ``An import of
            a path containing the element “internal” is disallowed if the
            importing code is outside the tree rooted at the parent of the
            `internal' directory.''\autocite{internal-packages}
        }
        package of the compiler
        \begin{itemize}
            \item Adding the \gls{ast} node type
            \item Adding type-checking for that node type
            \item Adding the AST traversal for that node type, translating it
                to AST nodes that the compiler already knows and can translate
                to builtin runtime-calls or \gls{ssa}
        \end{itemize}
\end{itemize}

The Go source code that is relevant for this thesis can be classified into three different
types. One is the godoc - the documentation for the new built-in functions. The
other two are the `public' and the `private' implementation of these builtins.

The `private' implementation is located within the
\textit{src/cmd/compile/internal} package\autocite{internal-packages}. It can only
be used by the package in \textit{src/cmd/compile}, which contains the
implementation of the compiler itself.

When calling \mintinline{shell}|go build .|, the compiler is invoked indirectly
through the main `go' binary. To directly invoke the compiler,
\mintinline{shell}|go tool compile| can be used.

Everything that is not in \textit{src/cmd/compile} is referred to as the `public'
part of the compiler in this thesis. The `public' parts are used by external
tools, for example Gopls, for type-checking, source code validation and
analysis.

\subsection{Adding the GoDoc}
In Go, documentation is generated directly from comments within the source code
\autocite{godoc}. This also applies to builtin functions in the compiler, which
have a function stub to document their behaviour\autocite{godoc-builtin}, but
no implementation, as that is done in the compiler\autocite{builtin-impl}.

The documentation for builtins should be as short and precise as possible.
The usage of `Type' and `Type1' has been decided based on other builtins
like `append' and 'delete'.
The function headers are derived from their Haskell counterparts, adjusted
to the Go nomenclature.

\begin{code}
    \captionof{listing}{Godoc for the new built-in functions}
    \gofilerange{../work/go/src/builtin/builtin.go}{begin-newbuiltins}{end-newbuiltins}
\end{code}
\subsection{Public packages}

\begin{quote}
Note that the `go/*` family of packages, such as `go/parser` and `go/types`,
have no relation to the compiler. Since the compiler was initially written in C,
the `go/*` packages were developed to enable writing tools working with Go code,
such as `gofmt` and `vet`.\autocite{compiler-readme}
\end{quote}

To enable tooling support for the new built-in functions, they have to be
registered in the `go/*' packages. The only package that is affected by new
builtins is `go/types'.

In the `types' package, the builtins have to be registered as such and as
`predeclared' functions:

\begin{code}
    \captionof{listing}{Registering new built-in functions}
    \gofilerange{../work/go/src/go/types/universe.go}{start-builtin}{end-builtin}
    \gofilerange{../work/go/src/go/types/universe.go}{start-predeclared}{end-predeclared}
\end{code}
This registration defines the type of the built-in - they are all expressions,
as they return a value - and the number of arguments.
After that, the type-checking and its associated tests are to be implemented.

This concludes the type-checking for external tools and makes `gopls' return errors
Once type-checking the new built-in functions is implemented, `gopls' can be
compiled against the new public packages.\footnote{
This means pointing the Go toolchain to the correct directory by setting
the value of `GOROOT' with \mintinline{shell}|go env -w GOROOT=<path>|.
}
It will then return errors if the wrong types are used. For example, when trying
to prepend an integer to a string slice:

\begin{gocode}
package main

import "fmt"

func main() {
    fmt.Println(prepend(3, []string{"hello", "world"}))
}
\end{gocode}

Gopls will report a type-checking error:
\begin{bashcode}
$ gopls check main.go
/tmp/playground/main.go:6:22-23: cannot convert 3 (untyped int constant) to string
\end{bashcode}

\subsection{Private packages}

In the private packages - the actual compiler - the expressions have to be
type-checked, ordered and transformed.

The type-checking process is similar to the one executed for external tools.
It should also check the node's child nodes, meaning an operations
arguments, body and init statements. Furthermore, during the type-checking
process, the built-in function's return types are set and node types
may be converted, if possible and necessary.
An operation may expect it's arguments to be in \mintinline{go}|node.Left|
and \mintinline{go}|node.Right|, which means type-checking will also need
to move the argument nodes from their default location in
\mintinline{go}|node.List| to \mintinline{go}|node.Left| and
\mintinline{go}|node.Right|.

Ordering ensures the evaluation order and re-orders expressions. All of
the new built-in functions will be evaluated left-to-right and there are now
special cases to handle.

Transforming means changing the AST nodes from the built-in operation to
nodes that the compiler knows how to translate to SSA. The actual algorithm
that these functions use cannot be implemented in normal Go code, they have to be
translated directly to AST nodes and statements.

There are more steps to compiling Go code, for example escape-checking,
SSA conversion and a lot of optimisations. These are not necessary to
implement and do not have a direct relation to the new built-ins, which
is why these steps are elided in this paper.

The actual algorithms and part of the implementations for the builtin
functions are covered in the following chapters.
\footnote{
    The full implementations can be viewed by diff-ing the git repository
    between the references `bachelor-thesis' and `go1.14'\autocite{ba-go1-14-thesis-diff}.
}

\subsubsection{fmap}\label{ch:impl-fmap}

To make the implementation in the AST easier, the algorithm will first be
developed in Go, and then translated. Implementing fmap in Go is relatively
simple:

\begin{code}
    \captionof{listing}{Fmap implementation in Go}
    \label{code:fmap-go}
    \begin{gocode}
func fmap(fn func(Type) Type1, src []Type) (dest []Type1) {
    for _, elem := range src {
        dest = append(dest, fn(elem))
    }
    return dest
}
\end{gocode}
\end{code}
However, there is room for improvement within that function. Instead
of calling \mintinline{go}|append| at every iteration of the loop, the slice can
be allocated with \mintinline{go}|make| at the beginning of the function. Thus,
calls to grow the slice at runtime can be saved.

\begin{code}
    \captionof{listing}{Improved implementation of fmap}
    \label{code:fmap-go-improved}
    \begin{gocode}
func fmap(fn func(Type) Type1, src []Type) []Type1 {
    dest := make([]Type1, len(src))
    for i, elem := range src {
        dest[i] = fn(elem)
    }
    return dest
}
    \end{gocode}
\end{code}
This algorithm can be translated to the following AST node:

\begin{code}
    \captionof{listing}{fmap AST translation\autocite{fmap-walk-implementation}}
    \gofilerange{../work/go/src/cmd/compile/internal/gc/walk.go}{start-fmap-header}{end-fmap-header}
\end{code}
\subsubsection{prepend}

The general algorithm for `prepend' is:
\begin{code}
    \captionof{listing}{prepend implementation in Go}
    \begin{gocode}
func prepend(elem Type, slice []Type) []Type {
    dest := make([]Type, 1, len(src)+1)
    dest[0] = elem
    return append(dest, slice...)
}
    \end{gocode}
\end{code}
The call to \mintinline{go}|make(...)| creates a slice with the length of 1 and the capacity
to hold all elements of the source slice, plus one. By allocating the slice with the full
length, another slice allocation within the call to \mintinline{go}|append(...)| is saved.
The element to prepend is added as the first element of the slice, and append will then
copy the `src' slice into `dest'.

The implementation within `walkprepend' reflects these lines of Go code, but
as AST nodes:

\begin{code}
    \captionof{listing}{prepend AST translation\autocite{prepend-walk-implementation}}
    \gofilerange{../work/go/src/cmd/compile/internal/gc/walk.go}{start-prepend-header}{end-prepend-header}
\end{code}
\subsubsection{foldr and foldl}

As outlined in Chapter~\ref{sec:fold}, there will be two fold functions;
foldr and foldl. foldr behaves exactly like its Haskell counterpart,
while foldl behaves like foldl' in Haskell.

While the fold algorithms are most obvious when using recursion, due to
performance considerations, an imperative implementation has been chosen:

\begin{code}
    \captionof{listing}{fold implementation in Go}
    \begin{gocode}
func foldr(fn func(Type, Type1) Type1, acc Type1, slice []Type) Type1 {
    for i := len(s) - 1; i >= 0; i-- {
        acc = fn(s[i], acc)
    }
    return acc
}

func foldl(fn func(Type1, Type) Type1, acc Type1, slice []Type) Type1 {
    for i := 0; i < len(s); i++ {
        acc = f(acc, s[i])
    }
    return acc
}
\end{gocode}
\end{code}
The code further clarifies the differences between the two different folds;
the slice is processed in reverse order for foldr (as it would be if this
algorithm would have been implemented with recursion), and the order of
arguments to the fold function is switched.

The AST walk translates fold to:
\begin{code}
    \captionof{listing}{fold AST translation\autocite{fold-walk-implementation}}
    \gofilerange{../work/go/src/cmd/compile/internal/gc/walk.go}{start-fold-header}{end-fold-header}
\end{code}
\subsubsection{filter}\label{ch:impl-filter}

Being a slice-manipulating function, filter also needs to traverse the whole
slice in a for-loop. However, compared to the other newly built-in functions,
the size for the target slice is unknown until all items have been traversed,
which is why filter does not allow for the same optimisations as the other
functions.

\begin{code}
    \captionof{listing}{filter implementation in Go}
    \begin{gocode}
func filter(f func(Type) bool, s []Type) []Type {
    var dst []Type
    for i := range s {
            if f(s) {
                dst = append(dst, s[i])
            }
    }
}
    \end{gocode}
\end{code}
And the same algorithm, but translated to AST statements:

\begin{code}
    \captionof{listing}{filter AST translation\autocite{filter-walk-implementation}}
    \gofilerange{../work/go/src/cmd/compile/internal/gc/walk.go}{start-filter-header}{end-filter-header}
\end{code}


\section{Functional Check}

As discussed in Chapter~\ref{sec:funcheck-theory}, a linter needs to be written
to detect reassignments within a Go program.

To get a grasp about the issues this linter is trying to report, the first step
is to capture examples, cases that should be matched against.

\subsection{Examples}

The simplest cases are standalone reassignments and assignment operators:
\begin{gocode}
x := 5
x = 6 // forbidden
// or
var y = 5
y = 6   // forbidden
y += 6  // forbidden
y <<= 2 // forbidden
y++     // forbidden
\end{gocode}
Where the statement \mintinline{go}|x = 6| and \mintinline{go}|y = 6| should be reported.

Adding block scoping to this, shadowing the old variable needs to be allowed:
\begin{gocode}
x := 5
{
	x = 6  // forbidden, changing the old value
	x := 6 // allowed, as this shadows the old variable
}
\end{gocode}
What should be illegal is to declare the variable first and then assign a
value to it:
\begin{gocode}
var x int
x = 6 // forbidden
\end{gocode}
The exception here are functions, as they need to be declared first in order
to recursively call them:
\begin{gocode}
var f func()
f = func() {
	f()
}
\end{gocode}
Furthermore, the linter also needs to be able to handle multiple variables
at once:
\begin{gocode}
var f func()
x, f, y := 1, func() { f() }, 2
\end{gocode}

All the aforementioned examples and more can be found in the testcases for funcheck\autocite{funcheck-examples}.

\subsection{Building a linter}

The Go ecosystem already provides an official library for building code analysis tools,
the `analysis' package from the Go Tools repository\autocite{go-analysis}. With this package,
implementing a static code analyizer is being reduced to writing the actual AST node analysis.

To define an analysis, a variable of type \mintinline{go}|*analysis.Analyzer| has to be declared:

\begin{gocode}
var Analyzer = &analysis.Analyzer{
	Name: "assigncheck",
	Doc:  "reports re-assignments",
	Run:  func(*analysis.Pass) (interface{}, error)
}
\end{gocode}
The necessary steps are now adding the `Run' function and registering the analyser
in the \mintinline{go}|main()| function.

The `Run' function takes a `Pass' type. The Pass provides information about the package
that is being analysed and some helper-functions to report diagnostics.

With `analysis.Pass.Files` and the help of the `go/ast` package, traversing the syntax
tree of every file in a package is made extremely convenient:

\begin{gocode}
for _, file := range pass.Files {
	ast.Inspect(file, func(n ast.Node) bool {
		// node analysing here
	})
}
\end{gocode}
To implement funcheck as described, five different AST node types need to be
taken care of. The simpler ones are
\mintinline{go}|*ast.IncDecStmt|, \mintinline{go}|*ast.ForStmt| and \mintinline{go}|*ast.RangeStmt|.
An `IncDecStmt' node is a \mintinline{go}|x++| or \mintinline{go}|x--|
expression and should always be reported.
`ForStmt' and `RangeStmt' are similar; a `RangeStmt' is a `for' loop with the
\mintinline{go}|range| keyword instead of an init-, condition- and post-stametent.

Both of these loop-types need to be reported explicitly as they do not show up
as reassignments in the AST.
The basic building blocks for our is the following \mintinline{go}|switch|
statement:

\begin{gocode}
switch as := n.(type) {
case *ast.IncDecStmt:
	pass.Reportf(as.Pos(), "inline re-assignment of %s", as.X)

case *ast.ForStmt:
	pass.Reportf(as.Pos(), "internal reassignment (for loop) in %q", renderFor(pass.Fset, as))

case *ast.RangeStmt:
	pass.Reportf(as.Pos(), "internal reassignment (for loop) in %q", renderRange(pass.Fset, as))
}
\end{gocode}
The remaining two node types are \mintinline{go}|*ast.DeclStmt| and \mintinline{go}|*ast.AssignStmt|.
They are not as simple to handle, which is why they are covered in their own chapters.

\subsection{Detecting reassignments}

To recapitulate, the goal of this step is to detect all assignments except blank identifiers
(discarded values cannot be mutated) and function literals, if the function is declared in the
last statement\footnote{This rule is to simplify the logic of the checker and make it easier
    for developers to read the code. It means that no code may be between \mintinline{go}|var f func|
and \mintinline{go}|f = func() { ... }|.}.

To detect such reassignments, funcheck iterates over all identifiers on the left-hand side
of an assignment statement.

On the left-hand side of an assignment is a list of expressions. These expressions can be
identifiers, index expressions (\mintinline{go}|*ast.IndexExpr|, for map and slice access),
a `star expression' (\mintinline{go}|*ast.StarExpr|\footnote{star expressions
    are expressions that are prefixed by an asterisk, dereferencing a pointer. For example
\mintinline{go}|*x = 5|, if \mintinline{go}|x| is of type \mintinline{go}|*int|.}) or others.

If the expression is not an identifier, the assignment must be a reassignment, as all non-identifier
expressions contain an already declared identifier. For example, the slice index expression
\mintinline{go}|s[5]| is of type \mintinline{go}|*ast.IndexExpr|:
\begin{gocode}
// An IndexExpr node represents an expression followed by an index.
IndexExpr struct {
	X      Expr      // expression
	Lbrack token.Pos // position of "["
	Index  Expr      // index expression
	Rbrack token.Pos // position of "]"
}
\end{gocode}

Where \mintinline{go}|IndexExpr.X| is our identifier `s' (of type \mintinline{go}|*ast.Ident|)
and a \mintinline{go}|IndexExpr.Index| is \mintinline{go}|5| (of type \mintinline{go}|*ast.BasicLit|).

As these nested identifiers already need to be declared beforehand (else they could not be used
in the expression), all expressions on the left-hand side of an assignment that are not identifiers
are reassignments.

Identifiers are the only expressions that can occur in declarations and reassignments. A naive
approach would be to check for the colon in a short variable declaration (\mintinline{go}|:=|).
However, as touched upon in Chapter~\ref{sec:multi-assign}, even short variable declarations may
contain redeclarations, if at least one variable is new.

Thus, another approach is needed.

Every identifier (an AST node with type \mintinline{go}|*ast.Ident|) contains an object\footnote{`An
    object describes a named language entity such as a package, constant, type, variable,
function (incl. methods), or a label'\autocite{go-ast-object}.} that links to the declaration.

This declaration, of whatever type it may be, always has a position (and a corresponding function
to retrieve that position) in the source file.

A reassignment is detected if an identifier's declaration position does not match the assignment's
position (indicating that the variable is being assigned at a different place to where it is
declared).

This is illustrated in the code block~\ref{code:assign-pos}. What can be clearly seen is that in
the assignment \mintinline{go}|y = 3|, \mintinline{go}|y|'s declaration refers to the position
of the first assignment \mintinline{go}|x, y := 1, 2|, the position where \mintinline{go}|y| has
been declared.

\begin{code}
    \captionof{listing}{Illustration of an assignment node and corresponding positions\autocite{ast-positions}}
    \begin{gocode}
Assignment "x, y := 1, 2": 2958101
        Ident "x": 2958101
                Decl "x, y := 1, 2": 2958101
        Ident "y": 2958104
                Decl "x, y := 1, 2": 2958101
Assignment "y = 3": 2958115
        Ident "y": 2958115
                Decl "x, y := 1, 2": 2958101
    \end{gocode}
  \label{code:assign-pos}
\end{code}
As this technique works on an identifier level, multi-variable declarations or assignments
can be verified without any additional effort.
If a variable in a short variable declaration is being reassigned, the variable's `Declaration'
field will point to the original position of its declaration, which can be easily detected
(as shown in code block~\ref{code:assign-pos}).

\subsection{Handling function declarations}

In contrast to all other variable types, function variables may be `reassigned' once.
As discussed in Chapter~\ref{sec:func-reassign}, this is to allow recursive function
literals. Detecting and not reporting these assignments is a two-step process, as two
consecutive AST nodes need to be inspected.

The first step is to detect function declarations; statements of the form
\mintinline{go}|var f func() |. Should such a statement be encountered,
its position will be saved for the following AST node.

In the consecutive AST node it is ensured that, if the node is an assignment and
the assignee identifier is of type function literal, the position matches the
previously saved one.

The position of the declaration and AST node structure can be seen in~\ref{code:func-reassign}

\begin{code}
    \captionof{listing}{Illustration of a function literal assignment\autocite{ast-positions}}
    \begin{gocode}
Declaration "var f func() int": 2958142
        Ident "f func() int": 2958146
Assignment "f = func() int { return y }": 2958160
        Ident "f": 2958160
                Decl "f func() int": 2958146
    \end{gocode}
    \label{code:func-reassign}
\end{code}
With this technique it is possible to excempt functions from the no-reassignment rule.

\subsection{Testing Funcheck}

The analysis-package is distributed with a subpackage `analysistest'. This package makes
it extremely simple to test a code analysing tool.

By providing testdata and the expected messages from funcheck in a structured way, all
that is needed to test funcheck is:

\begin{code}
    \captionof{listing}{Testing a code analyser with the `analysistest' package}
\begin{gocode}
package assigncheck

import (
        "testing"

        "golang.org/x/tools/go/analysis/analysistest"
)

func TestRun(t *testing.T) {
        analysistest.Run(t, analysistest.TestData(), Analyzer)
}
\end{gocode}
\end{code}

The library expects the testdata in the current directory in a folder named `testdata' and
then spawns and executes the analyser on the files in that folder. Comments in those files
are used to describe the expected message:

\begin{gocode}
x := 5
fmt.Println(x)
x = 6 // want `^reassignment of x$`
fmt.Println(x)
\end{gocode}

This will ensure that on the line \mintinline{go}|x = 6| an error message is reported that says
`reassignment of x'.


%\subsection{Required Steps}

%Adding a builtin function to the Go language requires a few more steps than just
%adding support within the compiler. While it would technically be enough to
%support the translation between Go code and the compiled binary, there would be
%no visibility for a developer that there is a function that could be used.
%For a complete implementation, the following steps are necessary:
%\begin{itemize}
    %\item Adding the GoDoc\autocite{godoc} that describes the function and it's usage
    %\item Adding type-checking support in external packages for tools like
    %Gopls\autocite{gopls}
    %\item Adding the implementation within the internal\footnote{
            %``An import of
            %a path containing the element “internal” is disallowed if the
            %importing code is outside the tree rooted at the parent of the
            %`internal' directory.''\autocite{internal-packages}
        %}
        %package of the compiler
        %\begin{itemize}
            %\item Adding the \gls{ast} node type
            %\item Adding type-checking for that node type
            %\item Adding the AST traversal for that node type, translating it
                %to AST nodes that the compiler already knows and can translate
                %to builtin runtime-calls or \gls{ssa}
        %\end{itemize}
%\end{itemize}

%The go source code that is relevant for this thesis can be classified into three different
%types. One is the godoc - the documentation for the new built-in functions. The
%other two are the `public' and the `private' implementation of these builtins.

%% TODO: this could need a rewrite
%The `private' implementation is everything that is located within the
%\textit{src/cmd/compile/internal} package\autocite{internal-packages}. It can only
%be used by the package in \textit{src/cmd/compile}, which contains the
%implementation of the compiler itself. When calling \mintinline{shell}|go build .|,
%the compiler is invoked indirectly. To directly invoke the compiler,
%\mintinline{shell}|go tool compile| can be used. The compile tool gets compiled
%from the main package located in \textit{src/cmd/compile}, which in turn
%uses the internal package.

%\newglossaryentry{gopls}{name=Gopls,description={Gopls is go's official language
%server implementation\autocite{gopls}}}

%Everything that is not in \textit{src/cmd/compile} is referred to as the `public'
%part of the compiler in this thesis. The `public' parts are used by external
%tools, for example \gls{gopls}, for type-checking, source code validation and
%analysis.

%\subsection{Adding the GoDoc}
%In Go, documentation is generated directly from comments within the source code
%\autocite{godoc}. This also applies to builtin functions in the compiler, which
%have a function stub to document their behaviour\autocite{godoc-builtin}, but
%no implementation, as that is done in the compiler\autocite{builtin-impl}.

%The documentation for builtins should be as short and precise as possible.
%The usage of `Type' and `Type1' has been decided based on other builtins
%like `append' and 'delete'.
%The function headers are derived from their Haskell counterparts, adjusted
%to the Go nomenclature.

%\begin{code}
    %\captionof{listing}{Godoc for the new built-in functions}
    %\gofilerange{../work/go/src/builtin/builtin.go}{begin-newbuiltins}{end-newbuiltins}
%\end{code}

%\subsection{Public packages}

%\begin{quote}
%Note that the `go/*` family of packages, such as `go/parser` and `go/types`,
%have no relation to the compiler. Since the compiler was initially written in C,
%the `go/*` packages were developed to enable writing tools working with Go code,
%such as `gofmt` and `vet`.\autocite{compiler-readme}
%\end{quote}

%To enable tooling support for the new built-in functions, they have to be
%registered in the `go/*' packages. The only package that is affected by new
%builtins is `go/types'.

%In the `types' package, the builtins have to be registered as such and as
%`predeclared' functions:

%\begin{code}
    %\captionof{listing}{Registering new built-in functions}
    %\gofilerange{../work/go/src/go/types/universe.go}{start-builtin}{end-builtin}
    %\gofilerange{../work/go/src/go/types/universe.go}{start-predeclared}{end-predeclared}
%\end{code}

%This registration defines the type of the built-in - they are all expressions,
%as they return a value - and the number of arguments.
%After that, the type-checking and its associated tests can be implemented.
%% TODO: reference the implementation

%This concludes the type-checking for external tools and makes `gopls' return errors
%Once type-checking the new built-in functions is implemented, `gopls' can be
%compiled against the new public packages.\footnote{
%This means pointing the go toolchain to the correct directory by setting
%the value of `GOROOT' with \mintinline{shell}|go env -w GOROOT=<path>|.
%}
%It will then return errors if the wrong types are used. For example, when trying
%to prepend an integer to a string slice:

%\begin{gocode}
%package main

%import "fmt"

%func main() {
    %fmt.Println(prepend(3, []string{"hello", "world"}))
%}
%\end{gocode}

%Gopls will report a type-checking error:
%\begin{bashcode}
%$ gopls check main.go
%/tmp/playground/main.go:6:22-23: cannot convert 3 (untyped int constant) to string
%\end{bashcode}

%\subsection{Private packages}

%In the private packages - the actual compiler - the expressions have to be
%type-checked, ordered and transformed.

%The type-checking process is similar to the one executed for external tools.
%It should also check the node's child nodes, meaning an operations
%arguments, body and init statements. Furthermore, during the type-checking
%process, the built-in function's return types are set and node types
%may be converted, if possible and necessary.
%An operation may expect it's arguments to be in \mintinline{go}|node.Left|
%and \mintinline{go}|node.Right|, which means type-checking will also need
%to move the argument nodes from their default location in
%\mintinline{go}|node.List| to \mintinline{go}|node.Left| and
%\mintinline{go}|node.Right|.

%Ordering ensures the evaluation order and re-orders expressions. All of
%the new built-in functions will be evaluated left-to-right and there are now
%special cases to handle.

%Transforming means changing the AST nodes from the built-in operation to
%nodes that the compiler knows how to translate to SSA. The actual algorithm
%that these functions use cannot be implemented in normal Go code, they have to be
%translated directly to AST nodes and statements.

%There are more steps to compiling Go code, for example escape-checking,
%SSA conversion and a lot of optimisations. These are not necessary to
%implement and do not have a direct relation to the new built-ins, which
%is why these steps are elided in this paper.

%The actual algorithms and part of the implementations for the builtin
%functions are covered in the following chapters.
%\footnote{
    %The full implementations can be viewed by diff-ing the git repository
    %between the references `bachelor-thesis' and `go1.14'\autocite{ba-go1-14-thesis-diff}.
%}

%\subsubsection{fmap}

%To make the implementation in the AST easier, the algorithm will first be
%developed in Go, and then translated. Implementing fmap in Go is relatively
%simple:

%\begin{code}
    %\captionof{listing}{Fmap implementation in Go}
    %\label{code:fmap-go}
    %\begin{gocode}
%func fmap(fn func(Type) Type1, src []Type) (dest []Type1) {
    %for _, elem := range src {
        %dest = append(dest, fn(elem))
    %}
    %return dest
%}
%\end{gocode}
%\end{code}

%However, there is room for improvement within that function. Instead
%of calling \mintinline{go}|append| at every iteration of the loop, the slice can
%be allocated with \mintinline{go}|make| at the beginning of the function. Thus,
%calls to grow the slice at runtime can be saved.

%\begin{code}
    %\captionof{listing}{Improved implementation of fmap}
    %\label{code:fmap-go-improved}
    %\begin{gocode}
%func fmap(fn func(Type) Type1, src []Type) []Type1 {
    %dest := make([]Type1, len(src))
    %for i, elem := range src {
        %dest[i] = fn(elem)
    %}
    %return dest
%}
    %\end{gocode}
%\end{code}

%This algorithm can be translated to the following AST node:

%\begin{code}
    %\captionof{listing}{fmap AST translation\autocite{fmap-walk-implementation}}
    %\gofilerange{../work/go/src/cmd/compile/internal/gc/walk.go}{start-fmap-header}{end-fmap-header}
%\end{code}

%\subsubsection{prepend}

%The general algorithm for `prepend' is:
%\begin{code}
    %\captionof{listing}{prepend implementation in Go}
    %\begin{gocode}
%func prepend(elem Type, slice []Type) []Type {
    %dest := make([]Type, 1, len(src)+1)
    %dest[0] = elem
    %return append(dest, slice...)
%}
    %\end{gocode}
%\end{code}

%The call to \mintinline{go}|make(...)| creates a slice with the length of 1 and the capacity
%to hold all elements of the source slice, plus one. By allocating the slice with the full
%length, another slice allocation within the call to \mintinline{go}|append(...)| is saved.
%The element to prepend is added as the first element of the slice, and append will then
%copy the `src' slice into `dest'.

%The implementation within `walkprepend' reflects these lines of Go code, but
%as AST nodes:

%\begin{code}
    %\captionof{listing}{prepend AST translation\autocite{prepend-walk-implementation}}
    %\gofilerange{../work/go/src/cmd/compile/internal/gc/walk.go}{start-prepend-header}{end-prepend-header}
%\end{code}

%\subsubsection{foldr and foldl}

%As outlined in Chapter~\ref{sec:fold}, there will be two fold functions;
%foldr and foldl. foldr behaves exactly like its Haskell counterpart,
%while foldl behaves like foldl' in Haskell.

%While the fold algorithms are most obvious when using recursion, due to
%performance considerations, an imperative implementation has been chosen:

%\begin{code}
    %\captionof{listing}{fold implementation in Go}
    %\begin{gocode}
%func foldr(fn func(Type, Type1) Type1, acc Type1, slice []Type) Type1 {
    %for i := len(s) - 1; i >= 0; i-- {
        %acc = fn(s[i], acc)
    %}
    %return acc
%}

%func foldl(fn func(Type1, Type) Type1, acc Type1, slice []Type) Type1 {
    %for i := 0; i < len(s); i++ {
        %acc = f(acc, s[i])
    %}
    %return acc
%}
%\end{gocode}
%\end{code}

%The code further clarifies the differences between the two different folds;
%the slice is processed in reverse order for foldr (as it would be if this
%algorithm would have been implemented with recursion), and the order of
%arguments to the fold function is switched.

%The AST walk translates fold to:
%\begin{code}
    %\captionof{listing}{fold AST translation\autocite{fold-walk-implementation}}
    %\gofilerange{../work/go/src/cmd/compile/internal/gc/walk.go}{start-fold-header}{end-fold-header}
%\end{code}

%\subsubsection{filter}

%Being a slice-manipulating function, filter also needs to traverse the whole
%slice in a for-loop. However, compared to the other newly built-in functions,
%the size for the target slice is unknown until all items have been traversed,
%which is why filter does not allow for the same optimisations as the other
%functions.

%\begin{code}
    %\captionof{listing}{filter implementation in Go}
    %\begin{gocode}
%func filter(f func(Type) bool, s []Type) []Type {
    %var dst []Type
    %for i := range s {
            %if f(s) {
                %dst = append(dst, s[i])
            %}
    %}
%}
    %\end{gocode}
%\end{code}

%And the same algorithm, but translated to AST statements:

%\begin{code}
    %\captionof{listing}{filter AST translation\autocite{filter-walk-implementation}}
    %\gofilerange{../work/go/src/cmd/compile/internal/gc/walk.go}{start-filter-header}{end-filter-header}
%\end{code}

%\section{Functional Check}


%Chapter~\ref{sec:fold}
