% -*- mode: latex; coding: utf-8; TeX-master: ../thesis -*-
% !TEX TS-program = pdflatexmk
% !TEX encoding = UTF-8 Unicode
% !TEX root = ../thesis.tex

\todo[inline]{%
  \quad -- {
    (Beschreibt die Grundüberlegungen der realisierten Lösung
    (Konstruktion/Entwurf) und die Realisierung als Simulation, als Prototyp
    oder als Software-Komponente)
  } \\
  \quad -- {
    (Definiert Messgrössen, beschreibt Mess- oder Versuchsaufbau,
    beschreibt und dokumentiert Durchführung der Messungen/Versuche)
  } \\
  \quad -- (Experimente) \\
  \quad -- (Lösungsweg) \\
  \quad -- (Modell) \\
  \quad -- (Tests und Validierung) \\
  \quad -- (Theoretische Herleitung der Lösung)
}

\section{Slice Helper Functions}

\subsection{Choosing the functions}

The first task is to implement some helper functions for slices, as they are present for lists in Haskell.
To decide on which functions will be implemented, popular Haskell repositories on Github have been analysed. The
popularity of repositories was decided to be based on their number of stars. Out of all Haskell projects
on Github, the most popular are\cite{github-popular-haskell}:

\begin{itemize}
    \item Shellcheck (koalaman/shellcheck\cite{github-shellcheck}): A static analysis tool for shell scripts
    \item Pandoc (jgm/pandoc\cite{github-pandoc}): A universal markup converter
    \item Postgrest (PostgREST/postgrest\cite{github-postgrest}): REST API for any Postgres database
    \item Semantic (github/semantic\cite{github-semantic}): Parsing, analyzing, and comparing source code across many languages
    \item Purescript (purescript/purescript\cite{github-purescript}): A strongly-typed language that compiles to JavaScript
    \item Compiler (elm/compiler\cite{github-elmcompiler}): Compiler for Elm, a functional language for reliable webapps
    \item Haxl (facebook/haxl\cite{github-haxl}): A Haskell library that simplifies access to remote data, such as databases or web-based services
\end{itemize}

In these repositories, the number of occurrences of popular list functions have been counted. The analysis does not
differentiate between different kind of functions. For example, `fold' includes all occurrences of `foldr', `foldl' and `foldl\''.
Also, the analysis has not been done with any kind of AST-parsing. Rather, a simple `grep' has been used to find matches. This means
that it is likely to contain some mismatches, for example in code comments. All in all, this analysis should only be
an indicator of what functions are used most.

 % TODO: replace 'prepend' with cons, where applicable (talking about Haskell)
Running the analysis on the 7 repositories listed above, searching for a number of pre-selected list functions, indicates
that the most used functions are `:' (cons), `map' and `fold', as shown in table~\ref{tab:occurrences-list-funcs}.

\begin{table}[htb]
\centering
\caption{Occurrences of list functions\ref{appendix:function-occurrences}}\label{tab:occurrences-list-funcs}
\begin{tabular}{ll}
\toprule
`:' (cons) & 2912 \\
\midrule
map & 1241 \\
\midrule
fold & 610 \\
\midrule
filter & 262 \\
\midrule
reverse & 154 \\
\midrule
take & 104 \\
\midrule
drop & 81 \\
\midrule
maximum & 53 \\
\midrule
sum & 44 \\
\midrule
zip & 38 \\
\midrule
product & 15 \\
\midrule
minimum & 10 \\
\midrule
reduce & 8
\end{tabular}
\end{table}

Based on this information, it has been decided to implement the cons, map and fold functions into the
Go compiler\footnote{The implementation can be found at `work/go' from the root of this git repository\cite{git-repo}. The work
is based upon Go version 1.14, commit ID `20a838ab94178c55bc4dc23ddc332fce8545a493'.
% TODO: remove this files-ref?
All files referenced in this chapter are based from the root of the go repository, unless noted otherwise.}

\subsection{Required Steps}
\newglossaryentry{ast}{name=AST,description={Abstract Syntax Tree, an abstract representation of source code as a tree}}
\newglossaryentry{ssa}{name=SSA,description={Single Static Assignment, an intermediate representation between the AST and the compiled binary that simplifies and improves compiler optimisations}}

Adding a builtin function to the Go language requires a few more steps than just adding support
within the compiler. While it would technically be enough to support the translation between
Go code and the compiled binary, there would be no visibility for a developer that there is a
function that could be used.
For a complete implementation, the following steps are necessary:
\begin{itemize}
    \item Adding the GoDoc\cite{godoc} that describes the function and it's usage
    \item Adding type-checking support in external packages for tools like Gopls\cite{gopls}
    \item Adding the implementation within the internal\footnote{
            ``An import of
            a path containing the element “internal” is disallowed if the importing code is
            outside the tree rooted at the parent of the “internal” directory.''\cite{internal-packages}
        }
        package of the compiler
        \begin{itemize}
            \item Adding the \gls{ast} node type
            \item Adding type-checking for that node type
            \item Adding the AST traversal for that node type, translating it
                to AST nodes that the compiler already knows and can translate
                to builtin runtime-calls or \gls{ssa}
        \end{itemize}
\end{itemize}

\subsection{Cons}

Being the most commonly used function in Haskell code, Go needs an easier to use implementation for the cons operator
% TODO: wording
than the currently existing \mintinline{go}|append([]<T>{elem}, src...)|. Keeping the naming scheme of go, the cons
function will be named `prepend', cohering with `append'.

In regular Go code, a more efficient implementation compared to the example is:
\begin{code}
\captionof{listing}{Prepend implementation in pseudo-Go code}
\label{code:prepend-raw-go}
\begin{gocode}
func prepend(elem <T>, slice []<T>) {
    dest := make([]<T>, 1, len(slice)+1)
    dest[0] = elem
    return append(dest, slice...)
}
\end{gocode}
\end{code}

The call to \mintinline{go}|make(...)| creates a slice with the length of 1 and the capacity
to hold all elements of the source slice, plus one. By allocating the slice with the full
length, another slice allocation within the call to \mintinline{go}|append(...)| is saved.
The element to prepend is added as the first element of the slice, and append will then
copy the `src' slice into `dest'.

This means that in the AST, the implementation should translate to:
\begin{code}
    \captionof{listing}{Prepend implementation in AST traversal}
\begin{gocode}
//   init {
//     dest := make([]<T>, 1, len(src)+1)
//     dest[0] = x
//     append(dest, src...)
//   }
//   dest
\end{gocode}
\end{code}

The type-checking for prepend is fairly simple. As seen in the Go implementation\ref{code:prepend-raw-go},
prepend accepts an element with type \mintinline{go}|T| and a slice with the element type \mintinline{go}|T|,
and will return a slice of element type \mintinline{go}|T|. The AST node's type, once translated,
will thus be \mintinline{go}|[]T|.

\subsection{Fmap}

Map is the second most-commonly used function in Haskell code. As `map' is already a registered keyword
within go's parser, the `map' function will be called `fmap' instead. This also improves readability
and makes it easier to distinguish the `map' type from the `map' function.

A naive implementation of `fmap', and it's improved version, is shown
in \ref{code:fmap-raw-go}

\begin{code}
    \captionof{listing}{Fmap implementation in pseudo-go code}
    \label{code:fmap-raw-go}
    \begin{gocode}
func fmapNaive(fn func(Type) Type1, src []Type) (dest []Type1) {
    for _, elem := range src {
        dest = append(dest, fn(elem))
    }
    return dest
}

func fmapImproved(fn func(Type) Type1, src []Type) []Type1 {
    dest := make([]Type1, len(src))
    for i, elem := range src {
        dest[i] = fn(elem)
    }
    return dest
}
    \end{gocode}
\end{code}

Again, by using \mintinline{go}|make| to allocate the slice with it's full length at the beginning
of the function, the calls to \mintinline{go}|append| and thus calls to grow the
slice at runtime can be saved.

Similarily to prepend, the AST walk should translate the call to fmap to:
\begin{code}
    \captionof{listing}{Fmap AST translation}
    \begin{gocode}
//   init {
//     dest := make([]out, len(src))
//     for i, e := range src {
//       dest[i] = f(e)
//     }
//   }
//   dest
    \end{gocode}
\end{code}

Type-checking fmap should ensure that the given function's argument type is the same
as the given slice's element type. Fmap returns a slice with the element type of
the given function return type.

\subsection{Fold}

The third and last function that is implemented is fold.
In Haskell, there are three distinct implementations of fold, `foldr', `foldl' and `foldl\''.
`foldr' allows, amongst other things, to transform infinite lists. `foldl' and `foldl\''
differ regarding their strictness properties. `foldl\'' effectively reserves the given list,
which means it returns the same result as `foldr' only if the given list is finite and the
function commutative. It does however have better memory space properties than `foldr'.

In Go, there are no infinite lists and no lazy evaluation. This means that a distinction
between the different fold types in Go is not necessary, and the implementation will
be based on the right fold, as:

\begin{quote}
    `foldr' is not only the right fold, it is also most commonly the right fold to use,
\end{quote}\cite{fold-types}

A pseudo-go implementation of `fold' looks like this:
\begin{code}
    \captionof{listing}{Fold implementation in pseudo-go code}
    \begin{gocode}
func fold(f func(Type, Type1) Type1, acc Type1, src []Type) Type1 {
    for i := len(src) - 1; i >= 0; i-- {
        acc = f(src[i], acc)
    }
    return acc
}
    \end{gocode}
\end{code}

This walks the given slice backwards, calling the function and updating the acculumator with
the new value.

The AST walk translates fold to:
\begin{code}
    \captionof{listing}{Fold AST translation}
    \begin{gocode}
//   init {
//     for i := len(s) - 1; i >= 0; i-- {
//       acc = f(s[i], acc)
//     }
//   }
//   acc
    \end{gocode}
\end{code}

The type-check for fold will need to ensure that the given function's first argument type corresponds
to the slice's element type, that the second argument is the same as the function's return type and
that the accumulator is of the same type as well.

\section{Functional Check}
