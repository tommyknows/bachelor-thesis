% -*- mode: latex; coding: utf-8; TeX-master: ../thesis -*-
% !TEX TS-program = pdflatexmk
% !TEX encoding = UTF-8 Unicode
% !TEX root = ../thesis.tex

\todo[inline]{%
  \quad -- {
    (Beschreibt die Grundüberlegungen der realisierten Lösung
    (Konstruktion/Entwurf) und die Realisierung als Simulation, als Prototyp
    oder als Software-Komponente)
  } \\
  \quad -- {
    (Definiert Messgrössen, beschreibt Mess- oder Versuchsaufbau,
    beschreibt und dokumentiert Durchführung der Messungen/Versuche)
  } \\
  \quad -- (Experimente) \\
  \quad -- (Lösungsweg) \\
  \quad -- (Modell) \\
  \quad -- (Tests und Validierung) \\
  \quad -- (Theoretische Herleitung der Lösung)
}

\section{Slice Helper Functions}

\subsection{Choosing the functions}

The first task is to implement some helper functions for slices, as they are present for lists in Haskell.
To decide on which functions will be implemented, popular Haskell repositories on Github have been analysed. The
popularity of repositories was decided to be based on their number of stars. Out of all Haskell projects
on Github, the most popular are\cite{github-popular-haskell}:

\begin{itemize}
    \item Shellcheck (koalaman/shellcheck\cite{github-shellcheck}): A static analysis tool for shell scripts
    \item Pandoc (jgm/pandoc\cite{github-pandoc}): A universal markup converter
    \item Postgrest (PostgREST/postgrest\cite{github-postgrest}): REST API for any Postgres database
    \item Semantic (github/semantic\cite{github-semantic}): Parsing, analyzing, and comparing source code across many languages
    \item Purescript (purescript/purescript\cite{github-purescript}): A strongly-typed language that compiles to JavaScript
    \item Compiler (elm/compiler\cite{github-elmcompiler}): Compiler for Elm, a functional language for reliable webapps
    \item Haxl (facebook/haxl\cite{github-haxl}): A Haskell library that simplifies access to remote data, such as databases or web-based services
\end{itemize}

In these repositories, the number of occurrences of popular list functions have been counted. Some caveats with
the analysis:

\begin{itemize}
    \item The analysis did not differentiate between different kinds of functions. For example, `fold' includes
    occurrences of `foldl', `foldr', `foldl\'' and `foldMap'.
    \item The analysis has been done with a simple `grep'. This means it is likely to contain mismatches, for example
    in code comments.
    \item The analysis should only be an \textit{indicator} of what functions are used most.
\end{itemize}

Running the analysis on the 7 repositories listed above, searching for a number of pre-selected list functions, indicates
that the most used functions are `:' (prepend), `map' and `fold', as shown in table~\ref{tab:occurrences-list-funcs}.

\begin{table}[htb]
\centering
\caption{Occurrences of list functions\ref{appendix:function-occurrences}}\label{tab:occurrences-list-funcs}
\begin{tabular}{ll}
\toprule
`:' (prepend) & 2912 \\
\midrule
map & 1241 \\
\midrule
fold & 610 \\
\midrule
filter & 262 \\
\midrule
reverse & 154 \\
\midrule
take & 104 \\
\midrule
drop & 81 \\
\midrule
maximum & 53 \\
\midrule
sum & 44 \\
\midrule
zip & 38 \\
\midrule
product & 15 \\
\midrule
minimum & 10 \\
\midrule
reduce & 8
\end{tabular}
\end{table}

Based on this information, it has been decided to implement the prepend, map and fold functions into the
Go compiler.
% TODO
The source code can be found at `work/go' from the root of this git repository\cite{git-repo}, the work
is done upon Go version 1.14, commit ID `20a838ab94178c55bc4dc23ddc332fce8545a493'.
All files referenced in this chapter are based from the root of the go repository, unless noted otherwise.

\subsection{Implementing Prepend}

\subsubsection{Adding the GoDoc}
In Go, documentation is usually generated directly from comments within the source code\cite{godoc}. This also applies to
builtin functions in the compiler, which have a function stub to document their behaviour\cite{godoc-builtin},
but no implementation, as that is done in the compiler\cite{builtin-impl}.

The append function, for example, is declared as
\begin{gocode}
func append(slice []Type, elems ...Type) []Type
\end{gocode}

So the signature of the prepend function is
\begin{gocode}
func prepend(elem Type, slice []Type) []Type
\end{gocode}

The file containing the documentation is in `\textit{src/builtin/builtin.go}'.
This is already everything that is needed to add the documentation for prepend.

\subsubsection{Adding prepend to the public packages}

The second step is to add prepend to the `\textit{src/go/types}' package.
\begin{quote}
    Package types declares the data types and implements the algorithms for
    type-checking of Go packages
\end{quote}\cite{godoc-types}

Though it will not have a direct impact on the compiler, adding the prepend function
to the \textit{types} package allows external tools like `gopls' (Go's official language server)\cite{gopls} to type-check `.go' files and support the programmer
at writing code.

\subsubsection{Registering the function as a builtin}

To allow type-checking prepend, the function has to be registered as a builtin.
This is done in `\textit{src/go/types/universe.go}':

\begin{gocode}
const (
	// universe scope
	_Append builtinId = iota
	_Prepend
	_Cap
    // ...omitted
)

var predeclaredFuncs = [...]struct {
	name     string
	nargs    int
	variadic bool
	kind     exprKind
}{
	_Append:  {"append", 1, true, expression},
	_Prepend: {"prepend", 2, false, expression},
	_Cap:     {"cap", 1, false, expression},
    // ...omitted
}
\end{gocode}

Prepend takes two arguments and is not a variadic function. The expression kind can either be a `conversion',
`expression' or a `statement'. prepend is an expression, like append, as it returns a value.


The type-checking that is implemented here is not related, in any way, to compiling the source code to machine code, which will do its own type-checking. It is only used for external tools.
To add the type-checking to prepend, a test has to be added too (a separate
test ensures that all builtins are type-checked).

\begin{gocode}
var builtinCalls = []struct {
	name, src, sig string
}{
	// ...omitted
	{"prepend", `var s []int; _ = prepend(0, s)`, `func(int, []int) []int`},
	{"prepend", `type T int; var s []T; var n T; _ = prepend(n, s)`, `func(p.T, []p.T) []p.T`},
	{"prepend", `var s []int; _ = (prepend)(0, s)`, `func(int, []int) []int`},
    // ...omitted
}
\end{gocode}

The type-checking itself is straightforward. The implementation starts by checking that the second
argument is a slice, and then extracts the type and checks that the first argument is of the same type.

% TODO: line length?
\textit{src/go/types/builtins.go}
\inputminted[
    linenos,
    numberfirstline,
    stepnumber=5,
    firstline=135,
    lastline=172,
    breaklines,
    breakanywhere,
    tabsize=4,
    gobble=1,
    bgcolor=bg,
]{go}{../work/go/src/go/types/builtins.go}
\begin{gocode}
func (check *Checker) builtin(x *operand, call *ast.CallExpr, id builtinId) (_ bool) {
	// ...omitted
	switch id {
	case _Prepend:
		// prepend(x T, s S) S, where T is the element type
		// of S.
		// spec: prepend is like append, but adds to the
		// beginning instead of the end of the slice. The
		// values x are passed to a parameter of type T where
		// T is the element type of S and the respective
		// parameter passing rules apply."

		// the second argument is the slice, so we start by
		// getting that type.
		arg(x, 1)
		S := x.typ
		var T Type
		if s, _ := S.Underlying().(*Slice); s != nil {
			T = s.elem
		} else {
			check.invalidArg(x.pos(), "%s is not a slice", x)
			return
		}

		// save the already evaulated argument
		arg1 := *x
		// reset to the first argument
		arg(x, 0)

		// check general case by creating custom signature
		sig := makeSig(S, T, S)
		check.arguments(x, call, sig, func(x *operand, i int) {
			// only evaluate arguments that have not been
			// evaluated before
			if i == 1 {
				*x = arg1
				return
			}
			arg(x, i)
		}, nargs)

		x.mode = value
		x.typ = S
		if check.Types != nil {
			check.recordBuiltinType(call.Fun, sig)
		}
	}
	// ...omitted
}
\end{gocode}

This concludes the type-checking for external tools and makes `gopls' return errors
if the wrong types are used. For example, when trying to prepend an integer to
a string slice.

\begin{gocode}
package main

import "fmt"

func main() {
	fmt.Println(prepend(3, []string{"hello", "world"}))
}
\end{gocode}

Gopls then reports the type-checking error:
\begin{bashcode}
$ gopls check main.go
/tmp/playground/main.go:6:22-23: cannot convert 3 (untyped int constant) to string
\end{bashcode}

\subsubsection{Implementing the function in the compiler}

The compiler is implemented at \textit{src/cmd/compile/internal/gc},
so all mentioned files are located within this directory.

\newglossaryentry{ast}{name=AST, description={Abstract Syntax Tree}}
\newglossaryentry{dag}{name=DAG, description={Directed Acyclic Graph}}

First, prepend needs to be registered as a builtin to parse prepend into
the\gls{ast} (technically, the syntax tree is a syntax \gls{dag}\cite{ast-node-dag},
but this is an implementation detail):


\textit{universe.go}
\begin{gocode}
// ...omitted
var builtinFuncs = [...]struct {
	name string
	op   Op
}{
	{"append", OAPPEND},
	{"prepend", OPREPEND},
	// ...omitted
}
\end{gocode}

`OPREPEND' is an operation that has to be defined in \textit{syntax.go}:

\begin{gocode}
// Node ops.
const (
	OXXX Op = iota
	// ...omitted
	OAPPEND       // append(List); after walk, Left may contain elem type descriptor
	OPREPEND      // prepend(Left, Right)
	OBYTES2STR    // Type(Left) (Type is string, Left is a []byte)
	// ...omitted
)
\end{gocode}

This also shows that an `OPREPEND' node expects its arguments to be in `node.Left' and `node.Right', and not, like `OAPPEND', in `node.List'. This is achieved in
\textit{typecheck.go}. Before typechecking, a node's arguments are always
defined in `node.List`. In the implementation for `OPREPEND', the arguments are typechecked and parsed
into `node.Left' and `node.Right' by

\begin{gocode}
// typecheck the arguments, expand function arguments etc.
typecheckargs(n)
// twoarg ensures there's exactly two arguments and adds
// them to n.Left and n.Right
if !twoarg(n) ...
\end{gocode}

After `node.Left' and `node.Right' are populated, `node.Right'\'s type is checked to be
a slice, and `node.Left' is of the same element type as the slice.
The full typecheck is added as case within \textit{typecheck.go}:

\inputminted[
    linenos,
    numberfirstline,
    stepnumber=5,
    firstline=1579,
    lastline=1613,
    breaklines,
    breakanywhere,
    tabsize=2,
    gobble=1,
    bgcolor=bg,
]{go}{../work/go/src/cmd/compile/internal/gc/typecheck.go}

With this, type-checking prepend in the AST is implemented.
The next step in the Go compiler is escape analysis. This is similar
to the `OAPPEND' node, though `OPREPEND' always has only two arguments:

\textit{escape.go}
\inputminted[
    linenos,
    numberfirstline,
    stepnumber=5,
    firstline=814,
    lastline=820,
    breaklines,
    breakanywhere,
    tabsize=2,
    gobble=2,
    bgcolor=bg,
]{go}{../work/go/src/cmd/compile/internal/gc/escape.go}

The next phase within the compiler are AST transformations to
lower the AST to a more easily compilable form. The first call
is made to `order'. `order' reorders expressions and enforces the
evaluation order. For prepend, this is done by adding another `case'
expression within \mintinline{go}|func (o *Order) expr(n, lhs *Node) *Node|.
Again, this can be implemented similarly to append:

\textit{order.go}
\inputminted[
    linenos,
    numberfirstline,
    stepnumber=5,
    firstline=1182,
    lastline=1188,
    breaklines,
    breakanywhere,
    tabsize=2,
    gobble=1,
    bgcolor=bg,
]{go}{../work/go/src/cmd/compile/internal/gc/order.go}

The compiler now calls `walk' to do more AST transformations, for example replacing
nodes like `OAPPEND' with the actual implementation of the algorithm, in AST form.
This is where the logic for `OPREPEND' needs to be added too.
The general algorithm for `prepend' is:
\begin{gocode}
s := make([]<T>, 1, len(dst)+1)
s[0] = x
append(s, dst...)
\end{gocode}

By allocating the slice with the full length, another slice allocation within
the call to append is saved. The element to prepend is added as the first element
of the slice, and append will then copy the `dst' slice into `s'.
The implementation within `walkprepend' reflects these lines of Go code, but
as AST nodes:

\textit{walk.go}
\inputminted[
    linenos,
    numberfirstline,
    stepnumber=5,
    firstline=2987,
    lastline=3017,
    breaklines,
    breakanywhere,
    tabsize=2,
    bgcolor=bg,
]{go}{../work/go/src/cmd/compile/internal/gc/walk.go}

This `walkprepend' function is called from `walkexpr'. Within `walkepr', it is called
only if the parent node is of the operation `OAS' or `OASOP', as prepend is an expression that always
returns a value, meaning it must be assigned back.

This concludes the implementation of prepend, though all changes can be found by comparing the
changes between the git tag `go1.14'\cite{ba-go-1-14} and `ba-added-prepend'\cite{ba-added-prepend}.

\section{Functional Check}
