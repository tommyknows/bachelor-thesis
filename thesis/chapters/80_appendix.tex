\section{Example for Functional Options}\label{appendix:funcopts}
\begin{code}
    \captionof{listing}{Functional Options for a simple Webserver}
    \gofile{../work/examples/functional-options/main.go}
\end{code}

\section{Analysis of function occurrences in Haskell code}\label{appendix:function-occurrences}
The results of the analysis have been aquired by running the following command
from the root of the git repository\cite{git-repo}:
\begin{bashcode}
./work/common-list-functions/count-function.sh "map " " : " "fold" "filter " "reverse " "take " "drop " "maximum" "sum " "zip " "product " "minimum " "reduce "
\end{bashcode}

\section{Mutating variables in Go}\label{appendix:mutation}
\begin{code}
	\captionof{listing}{Example on how to mutate complex types in Go}
	\gofile{../work/examples/mutate/main.go}
\end{code}

\section{Shadowing variables in Go}\label{appendix:shadowing}
\begin{code}
	\captionof{listing}{Example on how shadowing works on block scopes}
	\gofile{../work/examples/shadowing/main.go}
\end{code}

\section{Workaround for the missing foldl' implementation in Go}\label{appendix:foldl-go}
\begin{code}
	\captionof{listing}{Working around the missing foldl implementation in Go}
	\label{code:foldl-go}
	\gofile{../work/examples/foldl-workaround/main.go}
	\begin{bashcode}
$> fgo run .
0
panic: runtime error: invalid memory address or nil pointer dereference
[signal SIGSEGV: segmentation violation code=0x1 addr=0x0 pc=0x109e945]

goroutine 1 [running]:
main.what(0x2, 0x0, 0x2)
		/tmp/map/main.go:16 +0x5
main.main()
		/tmp/map/main.go:12 +0x187
exit status 2
	\end{bashcode}
\end{code}

\section{Prettyprint implementation}\label{appendix:prettyprint-func}
\begin{code}
	\captionof{listing}{The original prettyprint implementation}
	\gofile{../work/funcheck/prettyprint/prettyprint.go}
\end{code}
\begin{code}
	\captionof{listing}{The refactored, functional prettyprint implementation}
	\begin{gocode}
package prettyprint

import (
	"bytes"
	"fmt"
	"go/ast"
	"go/printer"
	"go/token"

	"golang.org/x/tools/go/analysis"
)

var Analyzer = &analysis.Analyzer{
	Name: "prettyprint",
	Doc:  "prints positions",
	Run:  run,
}

type null struct{}

func checkDecl(as *ast.DeclStmt, fset *token.FileSet) {
	fmt.Printf("Declaration %q: %v\n", render(fset, as), as.Pos())

	check := func(_ null, spec ast.Spec) (n null) {
		val, ok := spec.(*ast.ValueSpec)
		if !ok {
			return
		}
		if val.Values != nil {
			return
		}
		if _, ok := val.Type.(*ast.FuncType); !ok {
			return
		}
		fmt.Printf("\tIdent %q: %v\n", render(fset, val), val.Names[0].Pos())
		return
	}

	if decl, ok := as.Decl.(*ast.GenDecl); ok {
		_ = foldl(check, null{}, decl.Specs)
	}
}

func checkAssign(as *ast.AssignStmt, fset *token.FileSet) {
	fmt.Printf("Assignment %q: %v\n", render(fset, as), as.Pos())

	check := func(_ null, expr ast.Expr) (n null) {
		ident, ok := expr.(*ast.Ident) // Lhs always is an "IdentifierList"
		if !ok {
			return
		}

		fmt.Printf("\tIdent %q: %v\n", ident.String(), ident.Pos())

		switch {
		case ident.Name == "_":
			fmt.Printf("\t\tBlank Identifier!\n")
		case ident.Obj == nil:
			fmt.Printf("\t\tDecl is not in the same file!\n")
		default:
			// make sure the declaration has a Pos func and get it
			declPos := ident.Obj.Decl.(ast.Node).Pos()
			fmt.Printf("\t\tDecl %q: %v\n", render(fset, ident.Obj.Decl), declPos)
		}

		return
	}
	_ = foldl(check, null{}, as.Lhs)
}

func run(pass *analysis.Pass) (interface{}, error) {
	inspect := func(_ null, file *ast.File) (n null) {
		ast.Inspect(file, func(n ast.Node) bool {
			switch as := n.(type) {
			case *ast.DeclStmt:
				checkDecl(as, pass.Fset)
			case *ast.AssignStmt:
				checkAssign(as, pass.Fset)
			}
			return true
		})
		return
	}
	_ = foldl(inspect, null{}, pass.Files)

	return nil, nil
}

// render returns the pretty-print of the given node
func render(fset *token.FileSet, x interface{}) string {
	var buf bytes.Buffer
	if err := printer.Fprint(&buf, fset, x); err != nil {
		panic(err)
	}
	return buf.String()
}
	\end{gocode}
\end{code}

\section{Compiling and using functional Go}\label{appendix:install-fgo}

To compile and use the changes to the Go compiler that have been implemented in
this thesis, these instructions should be followed.

First, check out the Go source code:

\begin{bashcode}
$> git clone https://github.com/tommyknows/go.git
$> cd go
$> git checkout bachelor-thesis
\end{bashcode}

\subsection{With Docker}

If Docker is installed on your system, you can follow these steps from within
the checked out `go' git repository on the branch `bachelor-thesis'.
The downside of this approach is that it complicates building and sharing binaries.
To compile your own project, the directory has to be mounted into the container.
If your Guest OS is not Linux, cross-compilation is required so that the executable
can be ran on the host.

These steps have been wrapped inside a script that prints out the necessary commands
to configure your environment.

\begin{bashcode}
$> eval "$(./setup-docker.sh)"
\end{bashcode}
The commands within the shell script will
build Go in the container and print commands to configure the environment. The `eval'
command then executes these printed commands. Executing this command may take a while,
as the Go compiler is compiled within this process.

To build projects with the functional Go installation, simply use `fgo' on the command line.
An alias has been created that mounts the current directory and executes the `fgo'
command within the container.

Note that if this only configures the `fgo' command in the current shell session. To
persist it across shell-sessions, execute the script without eval:
\begin{bashcode}
$> ./setup-docker.sh
\end{bashcode}

And add the printed commands to your `.bashrc' (or equivalent). Further, you may
also need to change the binary path from `/tmp/fgo/bin' to a path which is not
cleaned up regularly.

\subsection{With a working Go installation}

If you already have a working Go installation on your system, the following
steps provide a way to get functional Go up and running in the same way
a normal go installation does.

These steps need to be executed from within the checked out `go' git
repository on the branch `bachelor-thesis'.

Build the functional Go binary and configure the environment:
\begin{bashcode}
$> cd ./src
$> ./make.bash
$> ln -s $(realpath $(pwd)/../bin/go) /usr/local/bin/fgo
$> go env -w GOROOT=$(realpath $(pwd)/..)
\end{bashcode}

The `go env' command sets the GOROOT to point to the newly compiled tools
and source code and is valid for the current shell session only.

\subsection{Using the installation}

After these steps, the binary (or alias) `fgo' can be used to test and build
functional Go code. `fgo' is not different to the normal `go' command, so
all commands that work with the normal `go' command also work with
the `fgo' command.

\begin{bashcode}
$> cd <code directory>
$> fgo test ./...
$> fgo build ./...
\end{bashcode}


\section{Building Funcheck}

Funcheck needs to be built against functional Go to properly detect the builtin functions.
If you have not done so already, install `fgo' as shown in~\ref{appendix:install-fgo}.

Then, funcheck can be installed directly with `go get' (or rather, `fgo get'). `go get'
downloads the source code to the Go modules directory (usually in \mintinline{bash}|$GOPATH/pkg/mod|),
compiles the specified package and moves the binary to \mintinline{bash}|$GOPATH/bin|.

\begin{bashcode}
$> # go get should not be called from within a go module
$> cd /tmp
$> fgo get github.com/tommyknows/funcheck
$> funcheck -h
\end{bashcode}

This installs `funcheck' into \mintinline{bash}|$GOBIN| or, if \mintinline{bash}|$GOBIN|
is not set, into \mintinline{bash}|$GOPATH/bin|.

You can also clone the git repository and use `fgo build' to build `funcheck':
\begin{bashcode}
$> git clone https://github.com/tommyknows/funcheck.git
$> cd funcheck
$> fgo build .
$> mv ./funcheck /usr/local/bin/funcheck
$> funcheck -h
\end{bashcode}

To run funcheck against the current directory / package, simply run
\begin{bashcode}
$> funcheck .
\end{bashcode}

\section{Building Gopls}\label{appendix:build-gopls}

Gopls is the official language server for Go. Similar to funcheck, there are two options
to install it on your local machine.

Installing with `go get':
\begin{bashcode}
$> # go get should not be called from within a go module
$> cd /tmp
$> fgo get golang.org/x/tools/gopls
$> gopls -h
\end{bashcode}

Or by downloading the source manually:
\begin{bashcode}
$> git clone https://github.com/golang/tools.git
$> cd ./tools/gopls
$> fgo build .
$> mv ./gopls /usr/local/bin/gopls
$> gopls -h
\end{bashcode}



% - Add your appendix here:

\todo[inline]{
  Anhang/Appendix:

  \quad -- Projektmanagement: \\ % chktex 8
  \qquad -- Offizielle Aufgabenstellung, Projektauftrag \\ % chktex 8
  \qquad -- (Zeitplan) \\ % chktex 8
  \qquad -- (Besprechungsprotokolle oder Journals) % chktex 8

  \quad -- Weiteres: \\ % chktex 8
  \qquad -- CD/USB-Stick mit dem vollständigen Bericht als PDF-File inklusive Film- und Fotomaterial \\ % chktex 8
  \qquad -- (Schaltpläne und Ablaufschemata) \\ % chktex 8
  \qquad -- (Spezifikation u. Datenblätter der verwendeten Messgeräte und/oder Komponenten) \\ % chktex 8
  \qquad -- (Berechnungen, Messwerte, Simulationsresultate) \\ % chktex 8
  \qquad -- (Stoffdaten) \\ % chktex 8
  \qquad -- (Fehlerrechnungen mit Messunsicherheiten) \\ % chktex 8
  \qquad -- (Grafische Darstellungen, Fotos) \\ % chktex 8
  \qquad -- (Datenträger mit weiteren Daten (z. B. Software-Komponenten) inkl. Verzeichnis der auf diesem Datenträger abgelegten Dateien) \\ % chktex 8
  \qquad -- (Softwarecode) % chktex 8
}
