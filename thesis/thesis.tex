% -*- mode: latex; coding: utf-8 -*-
% !TEX TS-program = pdflatexmk
% !TEX encoding = UTF-8 Unicode

\documentclass[%
  a4paper,
  twoside,
  numbers=noenddot,
  parskip=half+,
  open=any,
  headsepline,
  english, % german, english
  ba  % ba, pa
]{zhawthesis}

\usepackage{etoolbox}


%%%%%%%%%%%%%%%%%%%%%%%%%%%%%%%%%%%%%%%%
% Parameters
% - Adjust these to your needs:

\title{Functional Go}
\subtitle{an easier introduction to functional programming}
\author{% Komma getrennt
    Ramon Rüttimann
}
\newcommand\twodigits[1]{\ifnum#1<10 0#1\else #1\fi}
\date{\twodigits{\the\day}.\twodigits{\number\month}.\the\year}

\major{Informatik}  % Studiengang
\zhawsemester{Spring 2020}
\zhawinstitute{init}
\zhawlogocolour{pantone2945}  % pantone2945, cmyk, sw
\mainsupervisor{Prof. G. Burkert}
\subsupervisor{Prof. K. Rege}

%%%%%%%%%%%%%%%%%%%%%%%%%%%%%%%%%%%%%%%%
% Base packages used by the template (any commonly used packages)

\usepackage{float}
\usepackage{graphicx}
\graphicspath{{figures/}}
\DeclareGraphicsExtensions{.pdf,.png,.jpg,.gif}

\usepackage{tabularx}
\usepackage{longtable}
\usepackage{booktabs}
\usepackage{todonotes}

%%%%%%%%%%%%%%%%%%%%%%%%%%%%%%%%%%%%%%%%
% Custom packages
% - Add packages used by your thesis here:

\usepackage{minted}
\definecolor{bg}{rgb}{0.95,0.95,0.95}
\newminted{go}{breaklines,breakbytoken,tabsize=2,bgcolor=bg}
\newminted{bash}{breaklines,breakbytoken,tabsize=2,bgcolor=bg}
\newminted{haskell}{breaklines,breakbytoken,tabsize=2,bgcolor=bg}
\newmintedfile{go}{breaklines,breakbytoken,tabsize=2,bgcolor=bg}



\usepackage[
    backend=biber,
    style=ieee,
]{biblatex}
\usepackage[toc,page]{appendix}
\usepackage{glossaries}
\makeglossaries
%\usepackage{csquotes}
\AtBeginEnvironment{quote}{\itshape}
%\bibliography{thesis}
\addbibresource{thesis.bib}

\begin{document}

\frontmatter

\maketitle

\cleardoublepage % chktex 1


%%%%%%%%%%%%%%%%%%%%%%%%%%%%%%%%%%%%%%%%
% Declaration of Originality

\makedeclarationoforiginality % chktex 1


\cleardoublepage % chktex 1


%%%%%%%%%%%%%%%%%%%%%%%%%%%%%%%%%%%%%%%%

\IfLanguageName{nswissgerman}{\chapter{Zusammenfassung}}{\chapter{Summary}}
\label{ch:summary} % chktex 24
% -*- mode: latex; coding: utf-8; TeX-master: ../thesis -*-
% !TEX TS-program = pdflatexmk
% !TEX encoding = UTF-8 Unicode
% !TEX root = ../thesis.tex

Innerhalb der letzten zehn Jahre haben Konzepte und Ideen aus dem funktionalen
Programmieren im Alltag von vielen Entwicklern Fuss gefasst. Häufig wird
empfohlen, eine pur funktionale Programmiersprache wie zum Beispiel Haskell
zu lernen, um sich mit diesen Konzepten vertraut zu machen. Viele haben jedoch
Mühe, eine neue Syntax und ein neues Paradigma gleichzeitig zu lernen. Das Ziel
dieser Arbeit ist deswegen, einen einfacheren Einstieg in funktionales Programmieren
zu ermöglichen, dies mit Hilfe einer multiparadigmatischen Programmiersprache mit bekannter
Syntax.

Um dieses Ziel zu erreichen, wurde die Programmiersprache Go aufgrund ihrer
syntaktischen Simplizität und Vertrautheit gewählt.
Da Listen jedoch oft eine zentrale Rolle im funktionalen Programmieren einnehmen, ist ein
Nachteil dieser Wahl, dass Go keinen eingebauten List Datentyp besitzt. Zwar wird
dieser Nachteil durch Go's `Slices' gemildert, jedoch fehlen viele sogenannte `higher-order'
Funktionen um mit Listen zu arbeiten --- `map', `filter' und `reduce', um einige zu nennen.
Da Go's Typensystem keinen Polymorphismus bietet, müssen diese Funktionen im Compiler
implementiert werden, um eine möglichst benutzerfreundliche Verwendung zu ermöglichen.

Zusätzlich dazu wird die Bedeutung von `pure functional Programming' im Kontext dieser Arbeit
festgelegt und auf Basis dieser Definition das Code-Analyse Tool `funcheck' entwickelt, welches
nicht-funktionale Konstrukte im Programmcode meldet.

Mit den neuen built-in Funktionen `fmap', `filter', `foldr', `foldl' und `prepend',
sowie dem Linter `funcheck' erweist sich Go als geeignete Programmiersprache um
einen einfachen Einstieg in funktionales Programmieren zu ermöglichen. Der primäre Grund
spiegelt sich auch im Go Idiom `clear is better than clever' wider. Obwohl funktionaler
Go Code länger ist als in funktionalen Sprachen, ist dieser auch einfacher nachzuvollziehen.
Des Weiteren zeigt die Arbeit aber auch, dass es keinen Weg um eine pure funktionale Sprache
wie Haskell gibt, um sich funktionales Programmieren vollständig anzueignen.
Haskell's zwar ungewöhnliche, aber prägnante Syntax sowie das Design
der Sprache --- das Typensystem, Pattern Matching, die Purity Guarantees und vieles mehr ---
bilden hierfür eine solide und oft verwendete Grundlage.


\chapter{Abstract}
\label{ch:abstract} % chktex 24
% -*- mode: latex; coding: utf-8; TeX-master: ../thesis -*-
% !TEX TS-program = pdflatexmk
% !TEX encoding = UTF-8 Unicode
% !TEX root = ../thesis.tex

In the last decade, concepts from functional programming have grown in
importance within the wider, non-functional programming community.
Often it is recommended to learn a purely functional programming language
like Haskell to become familiar with these concepts.
However, many programmers struggle with the double duty
of learning a new paradigm and a new syntax at the same time. Because of
this, this paper explores the idea of learning purely functional programming % TODO: explores?
with a multi-paradigm programming language with a familiar syntax. The Go
programming language has been the choice for this due to its syntactical
simplicity and familiarity.

The absence of a list datatype in Go is remediated by Go's slices.
However, Go is missing the typical higher-order functions ---
`map', `filter' and `fold' to name a few --- that are present in every
functional programming language and many other programming languages too. Due to
this, the most popular higher-order functions have been determined and, because of the
absence of polymorphism in Go, implemented as built-in functions in the compiler.

Furthermore, this paper specifies a definition of what pure functional programming is and
introduces `funcheck', a static code analysis tool that has been designed and implemented to
report constructs that are non-functional.

With the help of the developed tools and extensions, Go proves itself to be
a suitable language for getting started with functional programming.
At the same time, it also shows why there is no way around learning a
language like Haskell if fluency with functional programming concepts
is desired.


\IfLanguageName{nswissgerman}{\chapter{Vorwort}}{\chapter{Preface}}
\label{ch:preface} % chktex 24
% -*- mode: latex; coding: utf-8; TeX-master: ../thesis -*-
% !TEX TS-program = pdflatexmk
% !TEX encoding = UTF-8 Unicode
% !TEX root = ../thesis.tex

\todo[inline]{Stellt den persönlichen Bezug zur Arbeit dar und spricht Dank aus.}


\cleardoublepage % chktex 1


%%%%%%%%%%%%%%%%%%%%%%%%%%%%%%%%%%%%%%%%
\mainmatter % chktex 1

\tableofcontents

\IfLanguageName{nswissgerman}{\chapter{Einleitung}}{\chapter{Introduction}}
\label{ch:introduction} % chktex 24
% -*- mode: latex; coding: utf-8; TeX-master: ../thesis -*-
% !TEX TS-program = pdflatexmk
% !TEX encoding = UTF-8 Unicode
% !TEX root = ../thesis.tex

\section{Learning Functional Programming}

In 2007, C\# 3.0 was released. Two years later, Ryan Dhal published the initial version
of NodeJS, eliminating JavaScript's ties to the browser and introducing it as a server-side
programming language. In 2013, Java 8 was released. Within the same timeframe, Python
has been rapidly growing in popularity\autocite{python-popularity}.

What all these events have in common is that they brought along concepts from functional
programming, so far mainly used in academia, into the daily life of many programmers.

Further, many new multi-paradigm programming languages have been introduced,
including Rust, Kotlin, Go and Dart. They all have functions as first-class citizens in
the language since their initial release.

Functional programming has landed in the wider programming-community, emerging from niche use-cases
and academia.
For example Rust, the `most popular programming language' for 4 years in a row (2016--2019)
according to the StackOverflow Developer survey\autocite{rust-loved}, has been significantly
influenced by functional programming languages\autocite{rust-functional}. Further, in idiomatic
Rust code, a functional style can be clearly observed\footnote{A simple example for this may be
that variables are immutable by default}.

Learning a purely functional programming language increases fluency with these concepts and
teaches a different way to think and approach problems when programming. Due to this, many
people recommend learning a functional programming
language\autocite{blog1-funcprog}\autocite{blog2-funcprog}\autocite{blog3-funcprog}\autocite{blog4-funcprog},
even if one may not end up using that language at all\autocite{quora-funcprog}.

Even though the exact definition of what a \textit{purely} functional language is remains a
controversy\autocite{functional-controversy}, most literature about functional programming,
including academia and online resources like blogs, contain code examples written in Haskell.
Further, according to the Tiobe Index\autocite{tiobe-index}, Haskell is also the most popular
purely functional programming language\autocite{comparison-functional-languages}.

\section{Haskell}

Haskell, the \textit{lingua franca} amongst functional programmers, is a lazely-evaluated, purely functional programming
language. While Haskell's strengths stem from all it's features like type classes, type polymorphism, purity and more,
these features are also what makes Haskell famously hard to learn\autocite{haskell-hard-one}\autocite{haskell-hard-two}\autocite{haskell-hard-three}\autocite{haskell-hard-four}.

Beginner Haskell programmers face a very distinctive challenge in contrast to learning a new, non-functional programming language:
Not only do they need to learn a new language with an unusual syntax (compared to imperative or object-oriented languages), they
also need to change their way of thinking and reasoning about problems.
For example, the renowned quicksort-implementation from the Haskell Introduction Page\autocite{haskell-quicksort}:

\label{code:haskell-quicksort}
\begin{haskellcode}
quicksort :: Ord a => [a] -> [a]
quicksort []     = []
quicksort (p:xs) = (quicksort lesser) ++ [p] ++ (quicksort greater)
    where
        lesser  = filter (< p) xs
        greater = filter (>= p) xs
\end{haskellcode}

While this is only a very short and clean piece of code, these 6 lines already pose many challenges to non-experienced Haskellers;

\begin{itemize}
    \item The function's signature with no `fn' or `func' statement as they often appear in imperative languages
    \item The pattern matching, which would be a `switch' statement or a chain of `if / else' conditions
    \item The deconstruction of the list within the pattern matching
    \item The functional nature of the program, passing `(< p)' (a function returning a function) to another function
    \item The function call to `filter' without paranthesised arguments and no clear indicator at which arguments
        it takes and which types are returned
\end{itemize}

Though some of these points are also available to programmers in imperative or object-oriented languages, the cumulative difference
is not to underestimate and adds to Haskell's steep learning curve.

\section{Goals}

Learning a new paradigm and syntax at the same time can be daunting and discouraging for novices.
By using a modern, multi-paradigm language with a clear and familiar syntax, the functional
programming beginner should be able to focus on the paradigm first, and then change to a language
like Haskell to fully get into functional programming.

To ease the learning curve of functional programming, this thesis will consist of two parts.
In the first part, writing functional code should be made as easy as possible. This means that
a language with an easy and familiar syntax should be chosen. Further, this programming language
should already support functions as first-class citizens. Additionally, it should be statically
typed, as a static type system makes it easier to reason about a program and can support the
programmer while writing code.
In the second part, a linter is created to check code for non-functional statements. To achieve
this, functional purity has to be defined, a ruleset has to be worked out and implemented into
a static analysis tool.

\section{Why Go}\label{sec:why-go}

The language of choice for this task is Go, a statically typed, garbage-collected programming language
designed at Google in 2009\autocite{golang-publish}. With its strong syntactic similarity to C, it should
be familiar to most programmers.

Go strives for simplicity and its syntax is extremely small and easy to learn. For example, the
language consists only of 25 keywords and purposefully omits constructs like the ternary operator
(<bool> ? <then> : <else>) as a replacement for the longer `if <bool> \{ <then> \} else \{ <else> \}' due
to clarity. `A language needs only one conditional control flow construct'\autocite{go-ternary},
and this also holds true for many other constructs. In Go, there is usually only one way
to express something, improving the clarity of code.

Due to this clarity and unambiguity, the language is a perfect fit to grasp the concepts and trace
the inner workings of functional programming. It should be easy to read code and understand what
it does without years of experience with the language.

There are however a few downsides of using Go. So far, Go does not have polymorphism, which means
that functions always have to be written with specific types. Due to this, Go also does not include
common list processing functions like `map', `filter', `reduce' and more\footnote{Although Go does
	have some polymorphic functions like `append', these are specified as built-in functions in the
language and not user-defined}. Further, Go does not have a built-in `list' datatype. However, Go's
`slices' cover a lot of use cases for lists already. Section~\ref{sec:go-slices} goes into more
details on slices.

\section{Existing Work}

With Go's support of some functional aspects, patterns and best practices have emerged that relate
to functional programming.
For example, in the \textit{net/http} package of the standard library, the function
\begin{gocode}
func HandleFunc(pattern string, handler func(ResponseWriter, *Request))
\end{gocode}
is used to register functions for http server handling:

\begin{gocode}
func myHandler(w http.ResponseWriter, r *http.Request) {
    // Handle the given HTTP request
}

func main() {
    // register myHandler in the default ServeMux
    http.HandleFunc("/", myHandler)
    http.ListenAndServe(":8080", nil)
}
\end{gocode}
\autocite{go-http-doc}

Using functions as function parameters or return types is a commonly used feature in Go, not just
within the standard library.

\subsection{Functional Options}

A software design pattern that has gained popularity within the Go community is `functional options'.
The pattern has been outlined in Dave Cheney's blog post `Functional options for friendly APIs'
and is a great example on how to use the support for multiple paradigms.
The basic idea with functional options is that a type constructor receives an unknown (0-n) amount
of options:
\begin{gocode}
func New(requiredSetting string, opts ...option) *MyType {
	t := &MyType{
		setting: requiredSetting,
	}

	for _, opt := range opts {
		opt(t)
	}

	return t
}

type option func(t *MyType)
\end{gocode}

These options can then access the instance of \mintinline{go}|MyType| to modify it accordingly,
for example:

\begin{gocode}
func EnableFeatureX() option {
	return func(t *MyType) {
		t.featureX = true
	}
}
\end{gocode}

To enable feature X, `New' can be called with that option:
\begin{gocode}
t := New("required", EnableFeatureX())
\end{gocode}

With this pattern, it is easy to introduce new options without breaking old usages of the API.
Furthermore, the typical `config struct' pattern can be avoided and meaningful zero values
can be set.

A more extensive example on how functional options are implemented and used can be found in
appendix~\ref{appendix:funcopts}.

\begin{quote}
    In summary
    \begin{itemize}
        \item Functional options let you write APIs that can grow over time.
        \item They enable the default use case to be the simplest.
        \item They provide meaningful configuration parameters.
        \item Finally they give you access to the entire power of the language to initialize complex values.
    \end{itemize}\autocite{functional-options}
\end{quote}

While this is a great example of what can be done with support for functional concepts, a purely functional approach to
Go has so far been discouraged by the core Go team, which is understandable for a multi-paradigm programming language.
However, multiple developers have already researched and tested Go's ability to do functional programming.

\subsection{Functional Go?}

In his talk `Functional Go'\autocite{func-go-talk}, Francesc Campoy Flores analysed some commonly used functional
language features in Haskell and how they can be ported with Go. Ignoring speed and stackoverflows due to non-existent
tail call optimisation\autocite{go-tco}, the main issue was with the type system and the missing polymorphism.

\subsection{go-functional}

In July 2017, Aaron Schlesinger, a Go programmer for Micosoft Azure, gave a talk on functional programming wit Go.
He released a repository\autocite{go-functional} that contains `core utilities for functional Programming in Go'.
The project is currently unmaintained, but showcases functional programming concepts like currying, functors and
monoids in Go.
In the `README' file of the repository, he also states that:
\begin{quote}
    Note that the types herein are hard-coded for specific types, but you could
    use code generation to produce these FP constructs for any type you please!
    \autocite{go-functional-readme}
\end{quote}

\section{Conclusion}

The aforementioned projects showcase the main issue with functional programming in Go: the missing
helper functions that are prevalent in functional languages and that they currently cannot be implemented
in a generic way.

To make functional programming more accessible in Go, this thesis will research what the most used
higher-order functions are and implement them with a focus on usability.
Furthermore, a list of rules for pure functional should be curated and implmented in a linter. This
linter can then be used to check existing code and report constructs which are not functional.


% DISABLE RELATED WORK AS I PUT THAT INTO THE INTRO
%\IfLanguageName{nswissgerman}{\chapter{Verwandte Arbeit}}{\chapter{Related Work}}
%\label{ch:related-work} % chktex 24
%\input{chapters/30_related_work.tex}

\IfLanguageName{nswissgerman}{\chapter{Methoden}}{\chapter{Methodology}}
\label{ch:methodology} % chktex 24
% -*- mode: latex; coding: utf-8; TeX-master: ../thesis -*-
% !TEX TS-program = pdflatexmk
% !TEX encoding = UTF-8 Unicode
% !TEX root = ../thesis.tex

\todo[inline]{%
  \quad -- {
    (Beschreibt die Grundüberlegungen der realisierten Lösung
    (Konstruktion/Entwurf) und die Realisierung als Simulation, als Prototyp
    oder als Software-Komponente)
  } \\
  \quad -- {
    (Definiert Messgrössen, beschreibt Mess- oder Versuchsaufbau,
    beschreibt und dokumentiert Durchführung der Messungen/Versuche)
  } \\
  \quad -- (Experimente) \\
  \quad -- (Lösungsweg) \\
  \quad -- (Modell) \\
  \quad -- (Tests und Validierung) \\
  \quad -- (Theoretische Herleitung der Lösung)
}


\chapter{Application}
\label{ch:application} % chktex 24
% -*- mode: latex; coding: utf-8; TeX-master: ../thesis -*-
% !TEX TS-program = pdflatexmk
% !TEX encoding = UTF-8 Unicode
% !TEX root = ../thesis.tex

\section{Refactoring the Prettyprint Package}

\newglossaryentry{stdout}{name=stdout, description={Standard Output, the default
output stream for programs}}

The code blocks~\ref{code:assign-pos} and~\ref{code:func-reassign} have been
generated by a small package `prettyprint' contained in the funcheck repository.

To see how the newly bult-in functions and funcheck can be used, we refactor `prettyprint'
to a purely functional version.
The current version of the package is written in what could be considered idiomatic
Go\footnote{
	Although there is no exact definition of what idiomatic Go is, so this interpretation
	could be challenged. It is idiomatic Go code to the author of this thesis.
}. %TODO?


The prettyprinter is based on the same framework as assigncheck\footnote{Assigncheck
is the main package for funcheck and checks the reassignments}, but instead
of reporting anything, it prints AST information to \gls{stdout}.

Similarily to assigncheck, the main logic of the package is within a
function literal that is being passed to the \mintinline{go}|ast.Inspect|
function.

Prettyprint only checks two AST node types, \mintinline{go}|*ast.DeclStmt|
(declarations) and \mintinline{go}|*ast.AssignStmt| (assignments).

For example, for the program
\begin{gocode}
package main

import "fmt"

func main() {
	x, y := 1, 2
	y = 3
	fmt.Println(x, y)
}
\end{gocode}
the following AST information is printed:

\begin{gocode}
Assignment "x, y := 1, 2": 2958101
		Ident "x": 2958101
				Decl "x, y := 1, 2": 2958101
		Ident "y": 2958104
				Decl "x, y := 1, 2": 2958101
Assignment "y = 3": 2958115
		Ident "y": 2958115
				Decl "x, y := 1, 2": 2958101
\end{gocode}

To refactor it to a purely functional version, funcheck can be used to
list statements that are not functional:

\begin{bashcode}
$> funcheck .
prettyprint.go:20:2: internal reassignment (for loop) in "for _, file := range pass.Files { ... }"
prettyprint.go:30:5: internal reassignment (for loop) in "for i := range decl.Specs { ... }"
prettyprint.go:53:5: internal reassignment (for loop) in "for _, expr := range as.Lhs { ... }"
\end{bashcode}
As can be seen in the output, the package uses 3 `for' loops to range over
slices. However, there are no other re-assignments of variables in the code.

The code to print declarations is as shown in~\ref{code:decl-printing}.

\begin{code}
	\captionof{listing}{Pretty-printing declarations in idiomatic Go}
	\label{code:decl-printing}
\begin{gocode}
func checkDecl(as *ast.DeclStmt, fset *token.FileSet) {
	fmt.Printf("Declaration %q: %v\n", render(pass.Fset, as), as.Pos())
	decl, ok := as.Decl.(*ast.GenDecl)
	if !ok {
		break
	}

	for i := range decl.Specs {
		val, ok := decl.Specs[i].(*ast.ValueSpec)
		if !ok {
			continue
		}

		if val.Values != nil {
			continue
		}

		if _, ok := val.Type.(*ast.FuncType); !ok {
			continue
		}

		fmt.Printf("\tIdent %q: %v\n", render(pass.Fset, val), val.Names[0].Pos())
	}
}
\end{gocode}
\end{code}
To convert this for-loop appropriately, the new built-in `foldl' can be used.
To recapitulate, the `foldl' function is being defined as:
\begin{gocode}
func foldl(fn func(Type1, Type) Type1, acc Type1, slice []Type) Type1
\end{gocode}
As `foldl' requires a return type, we introduce a dummy type `null', which
is just an empty struct:
\begin{gocode}
type null struct{}
\end{gocode}
Now the code within the foor loop can be used to create a function literal:
\begin{gocode}
check := func(_ null, spec ast.Spec) (n null) {
	// implementation
}
\end{gocode}
There are two subtleties in regards to the introduced null type:
First, the null value that is being passed as an argument is being discarded
by the use of an empty identifier.
Secondly, the return value is `named', which means the variable `n' is
already declared in the function block. Because of this, `naked returns' can
be used, so there is no need to specify which variable is being returned.

The snippet~\ref{code:decl-printing} thus can be translated to

\begin{code}
	\captionof{listing}{Pretty-printing declarations in functional Go}
	\begin{gocode}
func checkDecl(as *ast.DeclStmt, fset *token.FileSet) {
	fmt.Printf("Declaration %q: %v\n", render(fset, as), as.Pos())

	check := func(_ null, spec ast.Spec) (n null) {
		val, ok := spec.(*ast.ValueSpec)
		if !ok {
			return
		}

		if val.Values != nil {
			return
		}

		if _, ok := val.Type.(*ast.FuncType); !ok {
			return
		}

		fmt.Printf("\tIdent %q: %v\n", render(fset, val), val.Names[0].Pos())
		return
	}

	if decl, ok := as.Decl.(*ast.GenDecl); ok {
		_ = foldl(check, null{}, decl.Specs)
	}
}
\end{gocode}
\end{code}
The for-loop has been replaced by a `foldl', where we pass a function closure
that contains the actual processing.

While this still looks similar to the original example, this is mostly due to
the `if' statements. In Haskell, pattern matching would be used and nil checks
could be omitted entirely. Also, as Haskell's type system is more advanced, the
handling of those would be different too.

However, the goal of this thesis is to make functional code look more familiar
to programmers that are used to imperative code.
And while it may not look like it, the code does not use any mutation of
variables\footnote{Libraries may do, but the scope is not to rewrite any existing
libraries.}, for loops or global state. Therefore, it can be concluded that this
snippet is purely functional as per the definition from Chapter~\ref{sec:func-purity}.

\section{Quicksort}

In Chapter~\ref{code:haskell-quicksort}, a naive implementation of the Quicksort sorting
algorithm has been introduced.
Implementing this algorithm in Go is now straightforward and the similarities between
the Haskell implementation and the functional Go implementation are striking:

\begin{code}
	\captionof{listing}{Quicksort Implementations compared}
	\begin{gocode}
func quicksort(p []int) []int {
	if len(p) == 0 {
		return []int{}
	}

	lesser := filter(func(x int) bool { return p[0] > x }, p[1:])
	greater := filter(func(x int) bool { return p[0] <= x }, p[1:])

	return append(quicksort(lesser), prepend(p[0], quicksort(greater))...)
}
\end{gocode}
\begin{haskellcode}
quicksort :: Ord a => [a] -> [a]
quicksort []     = []
quicksort (p:xs) = (quicksort lesser) ++ [p] ++ (quicksort greater)
    where
        lesser  = filter (< p) xs
        greater = filter (>= p) xs
\end{haskellcode}
\end{code}

Again, the Go implementation bridges the gap between being imperative and functional,
while still being obvious about the algorithm.
Furthermore, as expected, when inspecting the code with funcheck, no non-functional
constructs are reported.

\section{Comparison to Java Streams}

In Java 8, concepts from functional programing have been introduced to the language.
The major new feature was Lambda Expressions --- anonymous function literals --- and
streams. Streams are an abstract layer to process data in a functional way, with `map',
`filter', `reduce' and more.

It is similar to the new built-in functions in this thesis:

\begin{code}
	\captionof{listing}{Comparision Java Streams and Functional Go}
	\begin{javacode}
List<Integer> even = list.stream()
	.filter(x -> x % 2 == 0)
	.collect(Collectors.toList());
	\end{javacode}
	\begin{gocode}
even := filter(
	func(x int) bool { return x%2 == 0 },
	list)
	\end{gocode}
\end{code}

The lambda-syntax in Java is more concise than Go's function literals, where the
complete function header has to be provided\footnote{There is an open proposal
	to add a lightweight anonymous function syntax to Go 2, which, if implemented,
would resolve the verbosity\autocite{go-lambdas}}. Java also has a more verbose way
to declare anonymous functions, which is creating an anonymous class:

\begin{javacode}
printPersons(
    roster,
    new CheckPerson() {
        public boolean test(Person p) {
            return p.getGender() == Person.Sex.MALE
                && p.getAge() >= 18
                && p.getAge() <= 25;
        }
    }
);
\end{javacode}
\autocite{java-lambda-expressions}

If this is used, Go's function literal syntax has an edge over Java's when it comes to
readability and conciseness.

Furthermore, the conversion to a stream and back to a list
(\mintinline{java}|list.stream()| and \mintinline{java}|.collect(Collectors.toList())|)
is not required in Go, as the operations all work on slices. Here, only having a single
list-like type built into the language is an advantage, as the mental overhead to convert
the list only to run a `filter' function can be avoided.

Apart from syntactical differences, Java Streams contain all the functions that
have been added as built-ins to Go too, and more.

However, Java's Syntax is arguably more complex than Go. An indicator for this might be
the language specification; Go's Language Specification is roughly 110 pages, while
Java's specification is more than 700 pages\footnote{
	The Java 8 Specification is 724\autocite{java-8-spec}, the Java 14
	Specification 774\autocite{java-14-spec} pages. An interesting sidenote may
	be that Go's specification grew by roughly 15 pages in 8 years (Go 1.0.0, 2012 - Go
	1.14.0, 2020), Java's by almost 170 pages between Java 7 (2013) and Java 14 (2020).
},
more than 6 times the size.


\IfLanguageName{nswissgerman}{\chapter{Resultate}}{\chapter{Experiments and Results}}
\label{ch:results} % chktex 24
% -*- mode: latex; coding: utf-8; TeX-master: ../thesis -*-
% !TEX TS-program = pdflatexmk
% !TEX encoding = UTF-8 Unicode
% !TEX root = ../thesis.tex

\todo[inline]{(Zusammenfassung der Resultate)}


\IfLanguageName{nswissgerman}{\chapter{Diskussion}}{\chapter{Discussion}}
\label{ch:discussion} % chktex 24
% -*- mode: latex; coding: utf-8; TeX-master: ../thesis -*-
% !TEX TS-program = pdflatexmk
% !TEX encoding = UTF-8 Unicode
% !TEX root = ../thesis.tex

\todo[inline]{%
  \quad --{
    Bespricht die erzielten Ergebnisse bezüglich ihrer Erwartbarkeit,
    Aussagekraft und Relevanz
  } \\
  \quad -- Interpretation und Validierung der Resultate \\
  \quad -- Rückblick auf Aufgabenstellung, erreicht bzw. nicht erreicht \\
  \quad -- {
    Legt dar, wie an die Resultate (konkret vom Industriepartner oder
    weiteren Forschungsarbeiten; allgemein) angeschlossen werden kann; legt
    dar, welche Chancen die Resultate bieten
  }
}



%%%%%%%%%%%%%%%%%%%%%%%%%%%%%%%%%%%%%%%%
\backmatter % chktex 1

% \let\clearpage\relax
% \vspace{-4em}
\printbibliography
% \endgroup


% \begingroup
% \let\clearpage\relax
% \vspace{-4em}
\listoffigures
% \endgroup

% \begingroup
% \let\clearpage\relax
% \vspace{-4em}
\listoftables
\printglossaries
% \endgroup
\cleardoublepage % chktex 1


%%%%%%%%%%%%%%%%%%%%%%%%%%%%%%%%%%%%%%%%
%\appendix
\begin{appendices}

\section{Analysis of function occurrences in Haskell code}\label{appendix:function-occurrences}
The results of the analysis have been aquired by running the following command
from the root of the git repository\cite{git-repo}:
\begin{minted}{shell}
./work/common-list-functions/count-function.sh ": " "map " "fold" "filter "
"reverse " "take " "drop " "maximum" "sum " "zip " "product " "minimum "
"reduce "
\end{minted}

% - Add your appendix here:

\todo[inline]{
  Anhang/Appendix:

  \quad -- Projektmanagement: \\ % chktex 8
  \qquad -- Offizielle Aufgabenstellung, Projektauftrag \\ % chktex 8
  \qquad -- (Zeitplan) \\ % chktex 8
  \qquad -- (Besprechungsprotokolle oder Journals) % chktex 8

  \quad -- Weiteres: \\ % chktex 8
  \qquad -- CD/USB-Stick mit dem vollständigen Bericht als PDF-File inklusive Film- und Fotomaterial \\ % chktex 8
  \qquad -- (Schaltpläne und Ablaufschemata) \\ % chktex 8
  \qquad -- (Spezifikation u. Datenblätter der verwendeten Messgeräte und/oder Komponenten) \\ % chktex 8
  \qquad -- (Berechnungen, Messwerte, Simulationsresultate) \\ % chktex 8
  \qquad -- (Stoffdaten) \\ % chktex 8
  \qquad -- (Fehlerrechnungen mit Messunsicherheiten) \\ % chktex 8
  \qquad -- (Grafische Darstellungen, Fotos) \\ % chktex 8
  \qquad -- (Datenträger mit weiteren Daten (z. B. Software-Komponenten) inkl. Verzeichnis der auf diesem Datenträger abgelegten Dateien) \\ % chktex 8
  \qquad -- (Softwarecode) % chktex 8
}

\end{appendices}

\end{document}
