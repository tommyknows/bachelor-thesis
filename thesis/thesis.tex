% -*- mode: latex; coding: utf-8 -*-
% !TEX TS-program = pdflatexmk
% !TEX encoding = UTF-8 Unicode

%\RequirePackage[hyphens]{url}
\documentclass[%
  a4paper,
  twoside,
  numbers=noenddot,
  parskip=half,
  open=any,
  headsepline,
  english, % german, english
  ba  % ba, pa
]{zhawthesis}

\usepackage{etoolbox}


%%%%%%%%%%%%%%%%%%%%%%%%%%%%%%%%%%%%%%%%
% Parameters
% - Adjust these to your needs:

\title{Functional Go}
\subtitle{...An Easier Introduction to Functional Programming}
\author{% Komma getrennt
    Ramon Rüttimann
}
\newcommand\twodigits[1]{\ifnum#1<10 0#1\else #1\fi}
\date{\twodigits{\the\day}.\twodigits{\number\month}.\the\year}

\major{Computer Science}  % Studiengang
\zhawsemester{Spring 2020}
\zhawinstitute{init}
\zhawlogocolour{pantone2945}  % pantone2945, cmyk, sw
\mainsupervisor{Dr. G. Burkert}
\subsupervisor{Dr. K. Rege}

%%%%%%%%%%%%%%%%%%%%%%%%%%%%%%%%%%%%%%%%
% Base packages used by the template (any commonly used packages)

%\PassOptionsToPackage{hyphens}{url}\usepackage{hyperref}

\usepackage{float}
\usepackage{graphicx}
\graphicspath{{figures/}}
\DeclareGraphicsExtensions{.pdf,.png,.jpg,.gif}

\usepackage{tabularx}
\usepackage{longtable}
\usepackage{booktabs}
\usepackage{todonotes}

%%%%%%%%%%%%%%%%%%%%%%%%%%%%%%%%%%%%%%%%
% Custom packages
% - Add packages used by your thesis here:

%\usepackage{hyperref}
%\PassOptionsToPackage{hyphens}{url}\usepackage{hyperref}

\hypersetup{
  colorlinks   = true, %Colours links instead of ugly boxes
  %urlcolor     = blue, %Colour for external hyperlinks
  linkcolor    = blue, %Colour of internal links
  citecolor   = blue %Colour of citations
}
\usepackage[newfloat]{minted}
\usepackage{listings}
\newenvironment{code}{\captionsetup{type=listing}}{}
\SetupFloatingEnvironment{listing}{name=Source Code,placement=H}
\definecolor{bg}{rgb}{0.95,0.95,0.95}
\newminted{bash}{breaklines,breakbytoken,tabsize=2,bgcolor=bg}
\newminted{bnf}{breaklines,breakbytoken,tabsize=2,bgcolor=bg}
\newminted{c}{breaklines,breakbytoken,tabsize=2,bgcolor=bg}
\newminted{go}{breaklines,breakbytoken,tabsize=2,bgcolor=bg}
\newminted{haskell}{breaklines,breakbytoken,tabsize=2,bgcolor=bg}
\newminted{java}{breaklines,breakbytoken,tabsize=2,bgcolor=bg}
\newmintedfile{go}{breaklines,breakanywhere,tabsize=2,bgcolor=bg,linenos,stepnumber=5,numberfirstline}


\newcommand{\gofilerange}[4][]{%
    \immediate\write18{./utils/delim -file="#2" -start="#3" -end="#4"}%
    \IfFileExists{code.lineno}%
      {\CatchFileEdef{\linenumber}{./code.lineno}{\endlinechar=-1 }}%
      {\def\linenumber{0}}%
    \edef\flags{firstnumber=\linenumber,#1}%
    \expandafter\gofile\expandafter[\flags]{./code.snippet}}

%\sloppy

\usepackage[
    backend=biber,
    hyperref=true,
    style=ieee,
    dashed=false,
]{biblatex}
%\usepackage{xurl}
\usepackage[skip=0pt]{caption}
\usepackage[toc,page]{appendix}
\usepackage{hyperref}
\usepackage{glossaries}
%\usepackage{listings}
\makeglossaries
%\usepackage{csquotes}
\AtBeginEnvironment{quote}{\itshape}
%\bibliography{thesis}
\addbibresource{thesis.bib}

\begin{document}

\frontmatter

\maketitle

\cleardoublepage % chktex 1


%%%%%%%%%%%%%%%%%%%%%%%%%%%%%%%%%%%%%%%%
% Declaration of Originality

\makedeclarationoforiginality % chktex 1


\cleardoublepage % chktex 1


%%%%%%%%%%%%%%%%%%%%%%%%%%%%%%%%%%%%%%%%


\chapter{Abstract}
\label{ch:abstract} % chktex 24
% -*- mode: latex; coding: utf-8; TeX-master: ../thesis -*-
% !TEX TS-program = pdflatexmk
% !TEX encoding = UTF-8 Unicode
% !TEX root = ../thesis.tex

In the last decade, concepts from functional programming have grown in
importance within the wider, non-functional programming community.
Often it is recommended to learn a purely functional programming language
such as Haskell to become familiar with these concepts.
However, many programmers struggle with the double duty
of learning a new paradigm and a new syntax at the same time.
This paper proposes that by learning functional programming with a
multi-paradigm programming language and a familiar syntax it is possible
to lower this effort.

To achieve this goal, the programming language Go has been chosen due to
its syntactical simplicity and familiarity.
However, a downside of Go is the lack of a built-in list type, as lists take a
central role in functional programming. Although this is remediated by Go's slices,
they are not accompanied by any higher-order list processing functions --- `map', `filter', and `fold' to name a
few --- that are present in every functional programming language (and many
other languages too).
Due to the absence of polymorphism, in order to provide these higher-order functions in
a user-friendly way it is necessary to build these functions into the compiler.

Furthermore, this paper adopts a definition of pure functional programming and
introduces `funcheck', a static code analysis tool that is designed to
report constructs that are non-functional.

In conclusion, with the help of the newly built-in functions `fmap', `filter', `foldr', `foldl' and
`prepend', as well as `funcheck' to lint code, Go proves itself to be a
suitable language for getting started with functional programming.
The primary factor for this is reflected in the Go idiom `clear is better than clever'.
While functional Go code is more verbose when compared to functional languages, it
is also more obvious about its inner workings.
At the same time, it also illustrates why there is no way around learning a
language such as Haskell if fluency with functional programming concepts
is desired. The main reasons are that, although it may be unusual at first, Haskell's
syntax is extremely concise, and that the language's design --- the type system,
pattern matching, the purity guarantees and more --- provides a very effective toolset
for purely functional programming.


\chapter{Zusammenfassung}
\label{ch:summary} % chktex 24
% -*- mode: latex; coding: utf-8; TeX-master: ../thesis -*-
% !TEX TS-program = pdflatexmk
% !TEX encoding = UTF-8 Unicode
% !TEX root = ../thesis.tex

Innerhalb der letzten zehn Jahre haben Konzepte und Ideen aus dem funktionalen
Programmieren im Alltag von vielen Entwicklern Fuss gefasst. Häufig wird
empfohlen, eine rein funktionale Programmiersprache wie zum Beispiel Haskell
zu lernen, um sich mit diesen Konzepten vertraut zu machen. Viele haben jedoch
Mühe, eine neue Syntax und ein neues Paradigma gleichzeitig zu lernen. Das Ziel
dieser Arbeit ist deswegen, mit Hilfe einer multiparadigmatischen Programmiersprache mit
bekannter Syntax einen einfacheren Einstieg in funktionales Programmieren zu ermöglichen.

Um dieses Ziel zu erreichen, wurde die Programmiersprache Go aufgrund ihrer
syntaktischen Simplizität und Vertrautheit gewählt.
Da Listen jedoch oft eine zentrale Rolle im funktionalen Programmieren einnehmen, ist ein
Nachteil dieser Wahl, dass Go keinen eingebauten Datentyp für Listen besitzt. Zwar wird
dieser Nachteil durch Go's `Slices' gemildert, jedoch fehlen viele Funktionen höherer
Ordnung um mit Listen zu arbeiten --- `map', `filter' und `reduce', um einige zu nennen.
Da Go's Typensystem keinen Polymorphismus bietet, müssen diese Funktionen im Compiler
implementiert werden, um eine möglichst benutzerfreundliche Verwendung zu ermöglichen.

Zusätzlich dazu wird die Bedeutung von rein funktionalem Programmieren im Kontext dieser Arbeit
festgelegt und auf Basis dieser Definition das Code-Analyse Tool `funcheck' entwickelt, welches
nicht-funktionale Konstrukte im Programmcode meldet.

Mit den neuen eingebauten Funktionen `fmap', `filter', `foldr', `foldl' und `prepend',
sowie dem Linter `funcheck' erweist sich Go als geeignete Programmiersprache um
einen einfachen Einstieg in funktionales Programmieren zu ermöglichen. Der primäre Grund
spiegelt sich auch im Go Idiom `clear is better than clever' wider. Obwohl funktionaler
Go Code länger ist als in funktionalen Sprachen, ist dieser auch einfacher nachzuvollziehen.
Des Weiteren zeigt die Arbeit aber auch, dass es keine Alternative zu einer rein funktionalen
Sprache wie Haskell gibt, um sich funktionales Programmieren vollständig anzueignen.
Haskell's zwar ungewöhnliche, aber prägnante Syntax sowie das Design
der Sprache --- das Typensystem, Pattern Matching, die Reinheitsgarantien und vieles mehr ---
bilden hierfür eine solide und oft verwendete Grundlage.


\IfLanguageName{nswissgerman}{\chapter{Vorwort}}{\chapter{Preface}}
\label{ch:preface} % chktex 24
% -*- mode: latex; coding: utf-8; TeX-master: ../thesis -*-
% !TEX TS-program = pdflatexmk
% !TEX encoding = UTF-8 Unicode
% !TEX root = ../thesis.tex

\todo[inline]{Stellt den persönlichen Bezug zur Arbeit dar und spricht Dank aus.}


\cleardoublepage % chktex 1


%%%%%%%%%%%%%%%%%%%%%%%%%%%%%%%%%%%%%%%%
\mainmatter % chktex 1

\tableofcontents

\IfLanguageName{nswissgerman}{\chapter{Einleitung}}{\chapter{Introduction}}
\label{ch:introduction} % chktex 24
% -*- mode: latex; coding: utf-8; TeX-master: ../thesis -*-
% !TEX TS-program = pdflatexmk
% !TEX encoding = UTF-8 Unicode
% !TEX root = ../thesis.tex

\section{Learning Functional Programming}

In 2007, C\# 3.0 was released. Two years later, Ryan Dhal published the initial version
of NodeJS, eliminating JavaScript's ties to the browser and introducing it as a server-side
programming language. In 2013, Java 8 was released. Within the same timeframe, Python
has been rapidly growing in popularity\autocite{python-popularity}.

What all these events have in common is that they brought along concepts from functional
programming, so far mainly used in academia, into the daily life of many programmers.

Further, many new multi-paradigm programming languages have been introduced,
including Rust, Kotlin, Go and Dart. They all have functions as first-class citizens in
the language since their initial release.

Functional programming has landed in the wider programming-community, emerging from niche use-cases
and academia.
For example Rust, the `most popular programming language' for 4 years in a row (2016--2019)
according to the StackOverflow Developer survey\autocite{rust-loved}, has been significantly
influenced by functional programming languages\autocite{rust-functional}. Further, in idiomatic
Rust code, a functional style can be clearly observed\footnote{A simple example for this may be
that variables are immutable by default}.

Learning a purely functional programming language increases fluency with these concepts and
teaches a different way to think and approach problems when programming. Due to this, many
people recommend learning a functional programming
language\autocite{blog1-funcprog}\autocite{blog2-funcprog}\autocite{blog3-funcprog}\autocite{blog4-funcprog},
even if one may not end up using that language at all\autocite{quora-funcprog}.

Even though the exact definition of what a \textit{purely} functional language is remains a
controversy\autocite{functional-controversy}, most literature about functional programming,
including academia and online resources like blogs, contain code examples written in Haskell.
Further, according to the Tiobe Index\autocite{tiobe-index}, Haskell is also the most popular
purely functional programming language\autocite{comparison-functional-languages}.

\section{Haskell}

Haskell, the \textit{lingua franca} amongst functional programmers, is a lazely-evaluated, purely functional programming
language. While Haskell's strengths stem from all it's features like type classes, type polymorphism, purity and more,
these features are also what makes Haskell famously hard to learn\autocite{haskell-hard-one}\autocite{haskell-hard-two}\autocite{haskell-hard-three}\autocite{haskell-hard-four}.

Beginner Haskell programmers face a very distinctive challenge in contrast to learning a new, non-functional programming language:
Not only do they need to learn a new language with an unusual syntax (compared to imperative or object-oriented languages), they
also need to change their way of thinking and reasoning about problems.
For example, the renowned quicksort-implementation from the Haskell Introduction Page\autocite{haskell-quicksort}:

\label{code:haskell-quicksort}
\begin{haskellcode}
quicksort :: Ord a => [a] -> [a]
quicksort []     = []
quicksort (p:xs) = (quicksort lesser) ++ [p] ++ (quicksort greater)
    where
        lesser  = filter (< p) xs
        greater = filter (>= p) xs
\end{haskellcode}

While this is only a very short and clean piece of code, these 6 lines already pose many challenges to non-experienced Haskellers;

\begin{itemize}
    \item The function's signature with no `fn' or `func' statement as they often appear in imperative languages
    \item The pattern matching, which would be a `switch' statement or a chain of `if / else' conditions
    \item The deconstruction of the list within the pattern matching
    \item The functional nature of the program, passing `(< p)' (a function returning a function) to another function
    \item The function call to `filter' without paranthesised arguments and no clear indicator at which arguments
        it takes and which types are returned
\end{itemize}

Though some of these points are also available to programmers in imperative or object-oriented languages, the cumulative difference
is not to underestimate and adds to Haskell's steep learning curve.

\section{Goals}

Learning a new paradigm and syntax at the same time can be daunting and discouraging for novices.
By using a modern, multi-paradigm language with a clear and familiar syntax, the functional
programming beginner should be able to focus on the paradigm first, and then change to a language
like Haskell to fully get into functional programming.

To ease the learning curve of functional programming, this thesis will consist of two parts.
In the first part, writing functional code should be made as easy as possible. This means that
a language with an easy and familiar syntax should be chosen. Further, this programming language
should already support functions as first-class citizens. Additionally, it should be statically
typed, as a static type system makes it easier to reason about a program and can support the
programmer while writing code.
In the second part, a linter is created to check code for non-functional statements. To achieve
this, functional purity has to be defined, a ruleset has to be worked out and implemented into
a static analysis tool.

\section{Why Go}\label{sec:why-go}

The language of choice for this task is Go, a statically typed, garbage-collected programming language
designed at Google in 2009\autocite{golang-publish}. With its strong syntactic similarity to C, it should
be familiar to most programmers.

Go strives for simplicity and its syntax is extremely small and easy to learn. For example, the
language consists only of 25 keywords and purposefully omits constructs like the ternary operator
(<bool> ? <then> : <else>) as a replacement for the longer `if <bool> \{ <then> \} else \{ <else> \}' due
to clarity. `A language needs only one conditional control flow construct'\autocite{go-ternary},
and this also holds true for many other constructs. In Go, there is usually only one way
to express something, improving the clarity of code.

Due to this clarity and unambiguity, the language is a perfect fit to grasp the concepts and trace
the inner workings of functional programming. It should be easy to read code and understand what
it does without years of experience with the language.

There are however a few downsides of using Go. So far, Go does not have polymorphism, which means
that functions always have to be written with specific types. Due to this, Go also does not include
common list processing functions like `map', `filter', `reduce' and more\footnote{Although Go does
	have some polymorphic functions like `append', these are specified as built-in functions in the
language and not user-defined}. Further, Go does not have a built-in `list' datatype. However, Go's
`slices' cover a lot of use cases for lists already. Section~\ref{sec:go-slices} goes into more
details on slices.

\section{Existing Work}

With Go's support of some functional aspects, patterns and best practices have emerged that relate
to functional programming.
For example, in the \textit{net/http} package of the standard library, the function
\begin{gocode}
func HandleFunc(pattern string, handler func(ResponseWriter, *Request))
\end{gocode}
is used to register functions for http server handling:

\begin{gocode}
func myHandler(w http.ResponseWriter, r *http.Request) {
    // Handle the given HTTP request
}

func main() {
    // register myHandler in the default ServeMux
    http.HandleFunc("/", myHandler)
    http.ListenAndServe(":8080", nil)
}
\end{gocode}
\autocite{go-http-doc}

Using functions as function parameters or return types is a commonly used feature in Go, not just
within the standard library.

\subsection{Functional Options}

A software design pattern that has gained popularity within the Go community is `functional options'.
The pattern has been outlined in Dave Cheney's blog post `Functional options for friendly APIs'
and is a great example on how to use the support for multiple paradigms.
The basic idea with functional options is that a type constructor receives an unknown (0-n) amount
of options:
\begin{gocode}
func New(requiredSetting string, opts ...option) *MyType {
	t := &MyType{
		setting: requiredSetting,
	}

	for _, opt := range opts {
		opt(t)
	}

	return t
}

type option func(t *MyType)
\end{gocode}

These options can then access the instance of \mintinline{go}|MyType| to modify it accordingly,
for example:

\begin{gocode}
func EnableFeatureX() option {
	return func(t *MyType) {
		t.featureX = true
	}
}
\end{gocode}

To enable feature X, `New' can be called with that option:
\begin{gocode}
t := New("required", EnableFeatureX())
\end{gocode}

With this pattern, it is easy to introduce new options without breaking old usages of the API.
Furthermore, the typical `config struct' pattern can be avoided and meaningful zero values
can be set.

A more extensive example on how functional options are implemented and used can be found in
appendix~\ref{appendix:funcopts}.

\begin{quote}
    In summary
    \begin{itemize}
        \item Functional options let you write APIs that can grow over time.
        \item They enable the default use case to be the simplest.
        \item They provide meaningful configuration parameters.
        \item Finally they give you access to the entire power of the language to initialize complex values.
    \end{itemize}\autocite{functional-options}
\end{quote}

While this is a great example of what can be done with support for functional concepts, a purely functional approach to
Go has so far been discouraged by the core Go team, which is understandable for a multi-paradigm programming language.
However, multiple developers have already researched and tested Go's ability to do functional programming.

\subsection{Functional Go?}

In his talk `Functional Go'\autocite{func-go-talk}, Francesc Campoy Flores analysed some commonly used functional
language features in Haskell and how they can be ported with Go. Ignoring speed and stackoverflows due to non-existent
tail call optimisation\autocite{go-tco}, the main issue was with the type system and the missing polymorphism.

\subsection{go-functional}

In July 2017, Aaron Schlesinger, a Go programmer for Micosoft Azure, gave a talk on functional programming wit Go.
He released a repository\autocite{go-functional} that contains `core utilities for functional Programming in Go'.
The project is currently unmaintained, but showcases functional programming concepts like currying, functors and
monoids in Go.
In the `README' file of the repository, he also states that:
\begin{quote}
    Note that the types herein are hard-coded for specific types, but you could
    use code generation to produce these FP constructs for any type you please!
    \autocite{go-functional-readme}
\end{quote}

\section{Conclusion}

The aforementioned projects showcase the main issue with functional programming in Go: the missing
helper functions that are prevalent in functional languages and that they currently cannot be implemented
in a generic way.

To make functional programming more accessible in Go, this thesis will research what the most used
higher-order functions are and implement them with a focus on usability.
Furthermore, a list of rules for pure functional should be curated and implmented in a linter. This
linter can then be used to check existing code and report constructs which are not functional.


\chapter{About Go}
\label{ch:about-go}
This chapter introduces the core concepts in Go that are needed to follow this paper.

\section{Go Slices}\label{sec:go-slices}

As mentioned in Section~\ref{sec:why-go}, Go does not have a `list' implementation and lists are rarely used.
The reason for this
is twofold. First, as Go does not have polymorphism, it is not possible for users to implement a generic
`list' type that would work with any underlying type. Second, the Go authors added `slices' as a core type
to the language. From a usage perspective, lists would not add anything compared to slices.

Go's Slices can be viewed as an abstraction over arrays, to mitigate some of the weaknesses of arrays
when compared to lists.

\begin{quote}
    Arrays have their place, but they're a bit inflexible, so you don't see them too often in Go code.
    Slices, though, are everywhere. They build on arrays to provide great power and convenience.\autocite{golang-slices}
\end{quote}

Slices can be visualised as a `struct' over an array:

\begin{gocode}
// NOTE: this type does not really exist, it
// is just to visualise how they are implemented.
type Slice struct {
    // the underlying "backing store" array
    array *[]T
    // the length of the slice / view on the array
    len int
    // the capacity of the array from the
    // starting index of the slice
    cap int
}
\end{gocode}

With the `append' function, elements can be added to a slice. Should the underlying array not have enough
capacity left to store the new elements, a new array will be created and the data from the old array will
be copied into the new one. This happens transparently to the user.

\subsection{Using Slices}

`head', `tail' and `last' operations can be done with index expressions:

\begin{gocode}
// []<T> initialises a slice, while [n]<T> initialises an
// array, which is of fixed length n. One can also use `...'
// instead of a natural number, to let the compiler count
// the number of elements.
s := []string{"first", "second", "third"}
head := s[0]
tail := s[1:]
last := s[len(s)-1]
\end{gocode}

Adding elements or joining slices is achieved with `append':

\begin{gocode}
s := []string{"first", "second"}
s = append(s, "third", "fourth")
t := []string{"fifth", "seventh"}
s = append(s, t...)
// to prepend an element, one has to create a
// slice out of that element
s = append([]string{"zeroth"}, s...)
\end{gocode}

Append is a variadic function, meaning it takes \textit{n} elements. If the slice is of type \textit{[]<T>},
the appended elements have to be of type \textit{<T>}.

To join two lists, the second list is expanded into
variadic arguments.

More complex operations like removing elements, inserting elements in the middle or finding
elements in a slice require helper functions, which have also been documented in Go's
Slice Tricks\autocite{slice-tricks}.

\subsection{What is missing from Slices}

This quick glance at slices should clarify that, though the runtime characteristics of lists and slices
can differ, from a usage standpoint, what is possible with lists is also possible with slices.

In a typical program written in a functional language, lists take a central role\footnote{Interestingly,
	there is no clear and direct answer as to why they do. One reason may be because they are recursively
	defined and trivially to implement functionally. Further, they are easier to use than arrays, where
	the programmer would need to track the index and bound of the array (imagine keeping track of the
indices in a recursive function)\autocite{why-lists}}. Because of this, functional languages have
a number of helper functions like `map', `filter' and `fold'\autocite{haskell-list-funcs} to modify and
work on lists. These so callend `higher order functions' currently do
not exist in Go and would need to be implemented by the programmer. With no support for polymorphism, a
different implementation would need to be written for every slice-type that is used. The type \mintinline{go}|[]int|
(read: a slice of integers) differs from \mintinline{go}|[]string|, which means that a possible
`map' function would have to be written once to support slices of integers, once to support slices
of strings, and a combination of these two:

\begin{gocode}
func mapIntToInt(f func(int) int, []int) []int
func mapIntToString(f func(int) string, []int) []string
func mapStringToInt(f func(string) int, []string) []int
func mapStringToString(f func(string) string, []string) []string
\end{gocode}

With 7 base types (eliding the different `int' types like `int8', `uint16`, `int16', etc.), this would
mean $7^{2} = 49$ map functions just to cover the base types. Counting the different numeric
types into that equation (totally 19 distinct types\autocite{go-basetypes}), would grow that number to $19^{2} = 361$ functions.

Though this code could be generated, it misses user-defined types which would still
need to be generated separately in a pre-compile step.

Another option, instead of having a function per type, would be that `map' takes and returns empty interfaces
(\mintinline{go}|interface{}|). However, `the empty interface says
nothing'\autocite{empty-interface}. The declaration of `map' would be:
\begin{gocode}
	func map(f func(interface{}) interface{}, []interface{}) interface{}
\end{gocode}

This function header does not say anything about it's types, which would
mean that they would need to be checked at runtime and handled gracefully. It
would also require the caller to do a type assertion after every call. Further,
a slice of type \mintinline{go}|T| cannot simply be converted or asserted to a slice of another type
\mintinline{go}|T2|\autocite{go-interface-slice-conv}\autocite{go-interface-slice-conv2}.
Because of this limitation, a usage pattern of map would be\footnote{There are more
	issues with this code sample, for example the returned \mintinline{go}|interface{}|. If
	it was \mintinline{go}|[]interface{}|, the conversion would need to be applied again. With
	\mintinline{go}|interface{}|, a type assertion would suffice. However, the implementation
would be extremely tedious}:
\begin{gocode}
s := []string{"hello", "world"}
var i []interface{}
for _, e := range s {
	i = append(i, e)
}
// i is now populated and can be used.
r := map(someFunc, f)
// to convert it back to []string:
s = r.([]string)
\end{gocode}

This exemplifies why using the empty interface is not an option. Further, the
function could not really be type-checked at compile type, as there is no indication
of which argument's type must be equal.

\section{Built-in functions}

To mitigate these issues, the most common list-operations (in Go slice-operations) will
be added as built-ins to the compiler, so that the programmer can use these functions
on every slice-type without any conversion or code generation being necessary.

The language specification defines what built-in functions are and which built-in
functions should exist:
\begin{quote}
    Built-in functions are predeclared. They are called like any other function
    but some of them accept a type instead of an expression as the first argument.

    The built-in functions do not have standard Go types, so they can only appear
    in call expressions; they cannot be used as function values.\autocite{go-spec-builtins}
\end{quote}

For example, the documentation for the built-in \mintinline{go}|append|:
\begin{gocode}
// The append built-in function appends elements to the end of a slice.
// ...
func append(slice []Type, elems ...Type) []Type
\end{gocode}

The documentation shows that the types supplied to append are not specified upfront.
Instead, they are resolved and checked during compilation of the program.

Thus, in order to have generic list processing functions, these functions need to
be implemented as built-ins in the compiler.

\section{The Go Compiler}

The Go programming language is defined by its specification\autocite{go-spec}, and not
it's implementation. As of Go 1.14, there are two major implementations of that
specification; Google's self-hosting compiler toolchain `gc', which is written in
Go, and `gccgo', a frontend for GCC, writen in C++.

When talking about the Go compiler, what's mostly referred to is `gc'\footnote{`gc' stands
for `go compiler', and not `garbage collection' (which is abbreviated as `GC').}.

A famous, although not completely correct story tells about Go being designed
while a C++ program was compiling\autocite{less-is-more}.
This is why one of the main goals when designing Go was fast compilation times:
\begin{quote}
    Finally, working with Go is intended to be fast: it should take at most a few
    seconds to build a large executable on a single computer. To meet these goals
    required addressing a number of linguistic issues: an expressive but lightweight
    type system; concurrency and garbage collection; rigid dependency specification;
    and so on. These cannot be addressed well by libraries or tools; a new language
    was called for.\autocite{go-faq}
\end{quote}

Go has taken some measures to to combat slow compilation times. In general, Go's dependency resolution is simpler
compared to other languages, for example by not allowing circular dependencies.
Furthermore, compilation is not even attempted if there are unused
imports or unused declarations of variables, types and functions.
This leads to less code to compile and in turn shorter compilation times.
Another reason is that `the `gc' compiler is simpler code compared to most
recent compilers'\autocite{nuts-compiler}. However, according
to Rob Pike, one of the creators of Go, Go's compiler is not notably fast, but
most other compilers are slow:

\begin{quote}
    The compiler hasn't even been properly tuned for speed. A truly fast compiler
    would generate the same quality code much faster.\autocite{nuts-compiler}
\end{quote}

How Go generates code with the `gc' compiler can be split into four phases:

\subsection{Parsing Go programs}

\newglossaryentry{dag}{name=DAG, description={Directed Acyclic Graph}}
\newglossaryentry{ast}{name=AST,description={Abstract Syntax Tree, an abstract representation of source code as a tree}}

The first phase of compilation is parsing Go programs into a syntax tree. This
is done by tokenising the code (`lexical analysis' - the `lexer'), parsing
(`syntax analysis' - the `parser') it and then constructing a syntax tree
(\gls{ast})\footnote{Technically, the syntax tree is a syntax \gls{dag}\autocite{ast-node-dag}}
for each source file.

In comparison to many other programming languages, Go can be parsed without a
symbol table. It has been designed to be easy to parse
and analyse, which makes the Go parser simple in it's design\autocite{go-faq-symbol}.

\subsection{Type-checking and AST-transformation}\label{sec:comp-type}

The second phase of compilation starts by converting the `syntax' package's
AST, created in the first phase, to the compiler's AST representation. This
is due to historical reasons, as `gc's AST definition was carried over
from the C implementation.

After the conversion, the AST is type-checked. Within the type-checking, there
are also some additional steps included like checking for `declared and not used'
variables and determining whether a function terminates.

After type-checking, some transformations are applied on the AST. This includes,
but is not limited to, eliminating dead code, inlining function calls and escape
analysis. What is also done in the transformation phase is rewriting built-in function
calls, replacing for example a call to the built-in `append' with the necessary
AST structure and runtime-calls to implement its functionality.

\subsection{SSA}

\newglossaryentry{ssa}{name=SSA,description={Single Static Assignment, an intermediate representation between the AST and the compiled binary that simplifies and improves compiler optimisations}}
In the third phase, the AST is converted to \gls{ssa} form. SSA  is `a
lower-level intermediate representation with specific properties that make it
easier to implement optimizations and to eventually generate machine code from
it'\autocite{compiler-readme}.

The conversion consists of multiple `passes' through the SSA that
apply machine-independent rules to optimise code. These generic
rewrite rules are applied on every architecture and thus mostly
concern expressions (for example replacing expressions with constant values and
optimising multiplications), dead code elimination and removal of unneeded
nil-checks.

\subsection{Generating machine code}

Lastly, in the fourth phase of the compilation, machine-specific
SSA optimisations are applied. These may include:
\begin{itemize}
    \item Rewriting generic values into their machine-specific variants
        (for example, on amd64, combining load-store operations)
    \item Dead-code elimination pass
    \item Pointer liveness analysis
    \item Removing unused local variables
\end{itemize}

After generating and optimising the SSA form, the code is passed to the
assembler, which replaces the so far generic instructions with
architecture-specific machine code and writes out the final object file\autocite{compiler-readme}.


% DISABLE RELATED WORK AS I PUT THAT INTO THE INTRO
%\IfLanguageName{nswissgerman}{\chapter{Verwandte Arbeit}}{\chapter{Related Work}}
%\label{ch:related-work} % chktex 24
%\input{chapters/30_related_work.tex}

\IfLanguageName{nswissgerman}{\chapter{Methoden}}{\chapter{Methodology}}
\label{ch:methodology} % chktex 24
% -*- mode: latex; coding: utf-8; TeX-master: ../thesis -*-
% !TEX TS-program = pdflatexmk
% !TEX encoding = UTF-8 Unicode
% !TEX root = ../thesis.tex

\todo[inline]{%
  \quad -- {
    (Beschreibt die Grundüberlegungen der realisierten Lösung
    (Konstruktion/Entwurf) und die Realisierung als Simulation, als Prototyp
    oder als Software-Komponente)
  } \\
  \quad -- {
    (Definiert Messgrössen, beschreibt Mess- oder Versuchsaufbau,
    beschreibt und dokumentiert Durchführung der Messungen/Versuche)
  } \\
  \quad -- (Experimente) \\
  \quad -- (Lösungsweg) \\
  \quad -- (Modell) \\
  \quad -- (Tests und Validierung) \\
  \quad -- (Theoretische Herleitung der Lösung)
}

% - Kapitel 3.1.2 ff: Sie starten mit einer Implementierung in Pseudo-Go. Vorschlag:
% beschreiben Sie erst einmal prepend. Wozu dient die Funktion? Beispiele? Wie
% heisst sie in anderen funktionalen Sprachen? Wie könnte sie in Go umgesetzt werden?
% Damit meine ich nicht: wie wird sie implementiert, sondern wie soll sie später
% genutzt werden können. Das ist dann das Ziel der Umsetzung.
\section{Slice Helper Functions}

\subsection{Choosing the functions}

The first task is to implement some helper functions for slices, as they are present for lists in Haskell.
To decide on which functions will be implemented, popular Haskell repositories on Github have been analysed. The
popularity of repositories was decided to be based on their number of stars. Out of all Haskell projects
on Github, the most popular are\cite{github-popular-haskell}:

\begin{itemize}
    \item Shellcheck (koalaman/shellcheck\cite{github-shellcheck}): A static analysis tool for shell scripts
    \item Pandoc (jgm/pandoc\cite{github-pandoc}): A universal markup converter
    \item Postgrest (PostgREST/postgrest\cite{github-postgrest}): REST API for any Postgres database
    \item Semantic (github/semantic\cite{github-semantic}): Parsing, analyzing, and comparing source code across many languages
    \item Purescript (purescript/purescript\cite{github-purescript}): A strongly-typed language that compiles to JavaScript
    \item Compiler (elm/compiler\cite{github-elmcompiler}): Compiler for Elm, a functional language for reliable webapps
    \item Haxl (facebook/haxl\cite{github-haxl}): A Haskell library that simplifies access to remote data, such as databases or web-based services
\end{itemize}

In these repositories, the number of occurrences of popular list functions have been counted. The analysis does not
differentiate between different kind of functions. For example, `fold' includes all occurrences of `foldr', `foldl' and `foldl\''.
Also, the analysis has not been done with any kind of AST-parsing. Rather, a simple `grep' has been used to find matches. This means
that it is likely to contain some mismatches, for example in code comments. All in all, this analysis should only be
an indicator of what functions are used most.

 % TODO: replace 'prepend' with cons, where applicable (talking about Haskell)
Running the analysis on the 7 repositories listed above, searching for a number of pre-selected list functions, indicates
that the most used functions are `:' (cons), `map' and `fold', as shown in table~\ref{tab:occurrences-list-funcs}.

\begin{table}[htb]
\centering
\caption{Occurrences of list functions\ref{appendix:function-occurrences}}\label{tab:occurrences-list-funcs}
\begin{tabular}{ll}
\toprule
`:' (cons) & 2912 \\
\midrule
map & 1241 \\
\midrule
fold & 610 \\
\midrule
filter & 262 \\
\midrule
reverse & 154 \\
\midrule
take & 104 \\
\midrule
drop & 81 \\
\midrule
maximum & 53 \\
\midrule
sum & 44 \\
\midrule
zip & 38 \\
\midrule
product & 15 \\
\midrule
minimum & 10 \\
\midrule
reduce & 8
\end{tabular}
\end{table}

Based on this information, it has been decided to implement the cons, map and fold functions into the
Go compiler\footnote{The implementation can be found at `work/go' from the root of this git repository\cite{git-repo}. The work
is based upon Go version 1.14, commit ID `20a838ab94178c55bc4dc23ddc332fce8545a493'.
% TODO: remove this files-ref?
All files referenced in this chapter are based from the root of the go repository, unless noted otherwise.}

\subsection{Required Steps}
\newglossaryentry{ast}{name=AST,description={Abstract Syntax Tree, an abstract representation of source code as a tree}}
\newglossaryentry{ssa}{name=SSA,description={Single Static Assignment, an intermediate representation between the AST and the compiled binary that simplifies and improves compiler optimisations}}

Adding a builtin function to the Go language requires a few more steps than just adding support
within the compiler. While it would technically be enough to support the translation between
Go code and the compiled binary, there would be no visibility for a developer that there is a
function that could be used.
For a complete implementation, the following steps are necessary:
\begin{itemize}
    \item Adding the GoDoc\cite{godoc} that describes the function and it's usage
    \item Adding type-checking support in external packages for tools like Gopls\cite{gopls}
    \item Adding the implementation within the internal\footnote{
            ``An import of
            a path containing the element “internal” is disallowed if the importing code is
            outside the tree rooted at the parent of the “internal” directory.''\cite{internal-packages}
        }
        package of the compiler
        \begin{itemize}
            \item Adding the \gls{ast} node type
            \item Adding type-checking for that node type
            \item Adding the AST traversal for that node type, translating it
                to AST nodes that the compiler already knows and can translate
                to builtin runtime-calls or \gls{ssa}
        \end{itemize}
\end{itemize}

\subsection{Cons}

Being the most commonly used function in Haskell code, Go needs an easier to use implementation for the cons operator
% TODO: wording
than the currently existing \mintinline{go}|append([]<T>{elem}, src...)|. Keeping the naming scheme of go, the cons
function will be named `prepend', cohering with `append'.

In regular Go code, a more efficient implementation compared to the example is:
\begin{code}
\captionof{listing}{Prepend implementation in pseudo-Go code}
\label{code:prepend-raw-go}
\begin{gocode}
func prepend(elem <T>, slice []<T>) {
    dest := make([]<T>, 1, len(slice)+1)
    dest[0] = elem
    return append(dest, slice...)
}
\end{gocode}
\end{code}

The call to \mintinline{go}|make(...)| creates a slice with the length of 1 and the capacity
to hold all elements of the source slice, plus one. By allocating the slice with the full
length, another slice allocation within the call to \mintinline{go}|append(...)| is saved.
The element to prepend is added as the first element of the slice, and append will then
copy the `src' slice into `dest'.

This means that in the AST, the implementation should translate to:
\begin{code}
    \captionof{listing}{Prepend implementation in AST traversal}
\begin{gocode}
//   init {
//     dest := make([]<T>, 1, len(src)+1)
//     dest[0] = x
//     append(dest, src...)
//   }
//   dest
\end{gocode}
\end{code}

The type-checking for prepend is fairly simple. As seen in the Go implementation\ref{code:prepend-raw-go},
prepend accepts an element with type \mintinline{go}|T| and a slice with the element type \mintinline{go}|T|,
and will return a slice of element type \mintinline{go}|T|. The AST node's type, once translated,
will thus be \mintinline{go}|[]T|.

\subsection{Fmap}

Map is the second most-commonly used function in Haskell code. As `map' is already a registered keyword
within go's parser, the `map' function will be called `fmap' instead. This also improves readability
and makes it easier to distinguish the `map' type from the `map' function.

A naive implementation of `fmap', and it's improved version, is shown
in \ref{code:fmap-raw-go}

\begin{code}
    \captionof{listing}{Fmap implementation in pseudo-go code}
    \label{code:fmap-raw-go}
    \begin{gocode}
func fmapNaive(fn func(Type) Type1, src []Type) (dest []Type1) {
    for _, elem := range src {
        dest = append(dest, fn(elem))
    }
    return dest
}

func fmapImproved(fn func(Type) Type1, src []Type) []Type1 {
    dest := make([]Type1, len(src))
    for i, elem := range src {
        dest[i] = fn(elem)
    }
    return dest
}
    \end{gocode}
\end{code}

Again, by using \mintinline{go}|make| to allocate the slice with it's full length at the beginning
of the function, the calls to \mintinline{go}|append| and thus calls to grow the
slice at runtime can be saved.

Similarily to prepend, the AST walk should translate the call to fmap to:
\begin{code}
    \captionof{listing}{Fmap AST translation}
    \begin{gocode}
//   init {
//     dest := make([]out, len(src))
//     for i, e := range src {
//       dest[i] = f(e)
//     }
//   }
//   dest
    \end{gocode}
\end{code}

Type-checking fmap should ensure that the given function's argument type is the same
as the given slice's element type. Fmap returns a slice with the element type of
the given function return type.

\subsection{Fold}

The third and last function that is implemented is fold.
In Haskell, there are three distinct implementations of fold, `foldr', `foldl' and `foldl\''.
`foldr' allows, amongst other things, to transform infinite lists. `foldl' and `foldl\''
differ regarding their strictness properties. `foldl\'' effectively reserves the given list,
which means it returns the same result as `foldr' only if the given list is finite and the
function commutative. It does however have better memory space properties than `foldr'.

In Go, there are no infinite lists and no lazy evaluation. This means that a distinction
between the different fold types in Go is not necessary, and the implementation will
be based on the right fold, as:

\begin{quote}
    `foldr' is not only the right fold, it is also most commonly the right fold to use,
\end{quote}\cite{fold-types}

A pseudo-go implementation of `fold' looks like this:
\begin{code}
    \captionof{listing}{Fold implementation in pseudo-go code}
    \begin{gocode}
func fold(f func(Type, Type1) Type1, acc Type1, src []Type) Type1 {
    for i := len(src) - 1; i >= 0; i-- {
        acc = f(src[i], acc)
    }
    return acc
}
    \end{gocode}
\end{code}

This walks the given slice backwards, calling the function and updating the acculumator with
the new value.

The AST walk translates fold to:
\begin{code}
    \captionof{listing}{Fold AST translation}
    \begin{gocode}
//   init {
//     for i := len(s) - 1; i >= 0; i-- {
//       acc = f(s[i], acc)
//     }
//   }
//   acc
    \end{gocode}
\end{code}

The type-check for fold will need to ensure that the given function's first argument type corresponds
to the slice's element type, that the second argument is the same as the function's return type and
that the accumulator is of the same type as well.

\section{Functional Check}


\chapter{Implementation}
\label{ch:implementation} % chktex 24
\section{Implementing the new built-in functions}

\subsection{Required Steps}

Adding a built-in function to the Go language requires a few more steps than just
adding support within the compiler. While it would technically be enough to
support the translation between Go code and the compiled binary, there would be
no visibility for a developer that there is a function which could be used.
For a complete implementation, the following steps are necessary:
\begin{itemize}
	\item Adding the Godoc\autocite{godoc} that describes the function and it's usage
	\item Adding type-checking support in external packages for tools like
		Gopls\footnote{Gopls is Go's official language server implementation\autocite{gopls}.}
	\item Adding the implementation within the internal\footnote{
			`internal' packages can only be imported by other packages that
			are rooted at the parent of the `internal' directory. It is used to
			make specific packages not importable  in order to decrease potential API surface\autocite{internal-packages}.
		}
		package of the compiler
		\begin{itemize}
			\item Adding the \gls{ast} node type
			\item Adding type-checking for that node type
			\item Adding the AST traversal (`walk') for that node type, translating it
				to AST nodes that the compiler already knows and can translate
				to built-in runtime-calls or \gls{ssa}
		\end{itemize}
\end{itemize}

The Go source code that is relevant for this thesis can be classified into three different
types. One is the Godoc --- the documentation for the new built-in functions. The
other two are the `public' and the `private' implementation of these built-ins.

The `private' implementation is located within the
\textit{src/cmd/compile/internal} package\autocite{internal-packages}. Because it
is an internal package, it can only
be used by the packages in \textit{src/cmd/compile}, which contain the
implementation of the compiler itself.

When calling
\begin{bashcode}
$> go build .
\end{bashcode}
the compiler is invoked indirectly
through the main `go' binary. To directly invoke the compiler,
\begin{bashcode}
$> go tool compile
\end{bashcode}
can be used.

Everything that is not in \textit{src/cmd/compile} is referred to as the `public'
part of the compiler in this thesis. The `public' parts are used by external
tools, for example Gopls, for type-checking, source code validation and
analysis.

\subsection{Adding the Godoc}
In Go, documentation is generated directly from comments within the source code
\autocite{godoc}. This also applies to built-in functions in the compiler, which
have a function stub to document their behaviour\autocite{godoc-builtin}, but
no implementation, as that is done in the compiler\autocite{builtin-impl}.

The documentation for built-ins should be as short and precise as possible.
The usage of `Type' and `Type1' has been decided based on other built-ins
like `append' and 'delete'.
The function headers are derived from their Haskell counterparts, adjusted
to the Go nomenclature.

\begin{code}
	\gofilerange{../work/go/src/builtin/builtin.go}{begin-newbuiltins}{end-newbuiltins}%
	\caption{Godoc for the new built-in functions\autocite{new-builtins-godoc}}
\end{code}
\subsection{Public packages}

To enable tooling support for the new built-in functions, they have to be
registered in the `go/*' packages. The only package that is affected by new
built-ins is `go/types'.

\begin{quote}
Note that the `go/*` family of packages, such as `go/parser` and `go/types`,
have no relation to the compiler. Since the compiler was initially written in C,
the `go/*` packages were developed to enable writing tools working with Go code,
such as `gofmt` and `vet`.\autocite{compiler-readme}
\end{quote}

In the `types' package, the built-ins have to be registered as such and as
`predeclared' functions:

\begin{code}
	\gofilerange{../work/go/src/go/types/universe.go}{start-builtin}{end-builtin}%
	\gofilerange{../work/go/src/go/types/universe.go}{start-predeclared}{end-predeclared}%
	\caption{Registering new built-in functions\autocite{new-builtins-universe}}
\end{code}
This registration defines the type of the built-in --- they are all expressions,
as they return a value --- and the number of arguments.
After that, the type-checking and its associated tests have been implemented, but
are not shown here. The implementation can be located in the `src/go/types' package
in the files `builtins.go', `builtins\_test.go' and `universe.go' See the git
diff\autocite{ba-go1-14-thesis-diff} to view the changes that have been made.

This concludes the type-checking for external tools.
`gopls' can be compiled against these changed public packages\footnote{
See Appendix~\ref{appendix:build-gopls} for instructions on how to build Gopls.
} and will then return errors if the wrong types are used. For example, when trying
to prepend an integer to a string slice:

\begin{gocode}
package main

import "fmt"

func main() {
	fmt.Println(prepend(3, []string{"hello", "world"}))
}
\end{gocode}

Gopls will report a type-checking error:
\begin{bashcode}
$ gopls check main.go
/tmp/playground/main.go:6:22-23: cannot convert 3 (untyped int constant) to string
\end{bashcode}

\subsection{Private packages}
In the private packages - the actual compiler - the expressions have to be
type-checked, ordered and transformed.

The type-checking process is similar to the one executed for external tools.
Furthermore, during the type-checking
process, the built-in function's return types are set and node types
may be converted, if possible and necessary.
An operation may expect it's arguments to be in \mintinline{go}|node.Left|
and \mintinline{go}|node.Right|, which means type-checking will also need
to move the argument nodes from their default location in
\mintinline{go}|node.List| to \mintinline{go}|node.Left| and
\mintinline{go}|node.Right|.

Ordering ensures the evaluation order and re-orders expressions. All of
the new built-in functions will be evaluated left-to-right and there are no
special cases to handle.

Transforming means changing the AST nodes from the built-in operation to
nodes that the compiler knows how to translate to SSA. The actual algorithm
that these functions use cannot be implemented in normal Go code, they have to be
translated directly to AST nodes and statements.

There are more steps to compiling Go code, for example escape-checking,
SSA conversion and a lot of optimisations. These are not necessary to
implement and do not have a direct relation to the new built-ins.

The algorithms and part of the implementations for the built-in
functions are covered in the following chapters\footnote{
	To see the full implementation, the git diff can be viewed\autocite{ba-go1-14-thesis-diff}.
}.

\subsubsection{fmap}\label{ch:impl-fmap}

To make the implementation in the AST easier, the algorithm will first be
developed in Go, and then translated. Implementing fmap in Go is relatively
simple:

\begin{listing}
	\begin{gocode}
func fmap(fn func(Type) Type1, src []Type) (dest []Type1) {
	for _, elem := range src {
		dest = append(dest, fn(elem))
	}
	return dest
}
\end{gocode}
	\caption{fmap implementation in Go}\label{code:fmap-go}
\end{listing}
However, there is room for improvement within that function. Instead
of calling \mintinline{go}|append| at every iteration of the loop, the slice can
be allocated with \mintinline{go}|make| at the beginning of the function. Thus,
calls to grow the slice at runtime can be saved.

\begin{listing}
	\begin{gocode}
func fmap(fn func(Type) Type1, src []Type) []Type1 {
	dest := make([]Type1, len(src))
	for i, elem := range src {
		dest[i] = fn(elem)
	}
	return dest
}
	\end{gocode}
	\caption{Improved implementation of fmap}\label{code:fmap-go-improved}
\end{listing}
This algorithm can be translated to the following AST node:

\begin{code}
	\gofilerange{../work/go/src/cmd/compile/internal/gc/walk.go}{start-fmap-header}{end-fmap-header}
	\caption{fmap AST translation\autocite{fmap-walk-implementation}}
\end{code}
\subsubsection{prepend}

The general algorithm for `prepend' is:
\begin{listing}
	\begin{gocode}
func prepend(elem Type, slice []Type) []Type {
	dest := make([]Type, 1, len(src)+1)
	dest[0] = elem
	return append(dest, slice...)
}
	\end{gocode}
	\caption{prepend implementation in Go}
\end{listing}
The call to \mintinline{go}|make(...)| creates a slice with the length of 1 and the capacity
to hold all elements of the source slice, plus one. By allocating the slice with the full
length, another slice allocation within the call to \mintinline{go}|append(...)| is saved.
The element to prepend is added as the first element of the slice, and append will then
copy the `src' slice into `dest'.

The implementation within `walkprepend' reflects these lines of Go code, but
as AST nodes:

\begin{code}
	\gofilerange{../work/go/src/cmd/compile/internal/gc/walk.go}{start-prepend-header}{end-prepend-header}
	\caption{prepend AST translation\autocite{prepend-walk-implementation}}
\end{code}
\subsubsection{foldr and foldl}

As outlined in Chapter~\ref{sec:fold}, there will be two fold functions;
foldr and foldl. foldr behaves exactly like its Haskell counterpart,
while foldl behaves like foldl' in Haskell.

While the fold algorithms are most obvious when using recursion, due to
performance considerations, an imperative implementation has been chosen:

\begin{listing}
	\begin{gocode}
func foldr(fn func(Type, Type1) Type1, acc Type1, slice []Type) Type1 {
	for i := len(s) - 1; i >= 0; i-- {
		acc = fn(s[i], acc)
	}
	return acc
}

func foldl(fn func(Type1, Type) Type1, acc Type1, slice []Type) Type1 {
	for i := 0; i < len(s); i++ {
		acc = f(acc, s[i])
	}
	return acc
}
\end{gocode}
	\caption{fold implementation in Go}
\end{listing}
The code further clarifies the differences between the two different folds;
the slice is processed in reverse order for foldr (as it would be if this
algorithm would have been implemented with recursion), and the order of
arguments to the fold function is switched.

The AST walk translates fold to:
\begin{code}
	\gofilerange{../work/go/src/cmd/compile/internal/gc/walk.go}{start-fold-header}{end-fold-header}
	\caption{fold AST translation\autocite{fold-walk-implementation}}
\end{code}
\subsubsection{filter}\label{ch:impl-filter}

Being a slice-manipulating function, filter also needs to traverse the whole
slice in a for-loop. However, compared to the other newly built-in functions,
the size for the target slice is unknown until all items have been traversed,
which is why filter does not allow for the same optimisations as the other
functions.

\begin{listing}
	\begin{gocode}
func filter(f func(Type) bool, s []Type) []Type {
	var dst []Type
	for i := range s {
			if f(s) {
				dst = append(dst, s[i])
			}
	}
}
	\end{gocode}
	\caption{filter implementation in Go}
\end{listing}
And the same algorithm, but translated to AST statements:

\begin{code}
	\gofilerange{../work/go/src/cmd/compile/internal/gc/walk.go}{start-filter-header}{end-filter-header}
	\caption{filter AST translation\autocite{filter-walk-implementation}}
\end{code}

\subsubsection{Writing the AST traversal}

The previous chapters have all shown the function headers of the `walk' functions
that are used to traverse and rewrite the new built-ins. To illustrate how the
actual implementation of such an algorithm looks like in these functions, we
provide a small example here. The full implementation of these algorithms can
be viewed at the git diff\autocite{ba-go1-14-thesis-diff}.

\begin{listing}
	\begin{gocode}
// the type here is not specified, as it is dynamic
filtered := make([]T, 0)

// this call translates to:
// construct the first argument to make, the type
makeType := nod(OTYPE, nil, nil)
makeType.Type = source.Type // copy the type of the slice

// create the make(...) call
makeDest := nod(OMAKE, nil, nil)
// add the arguments (the type and an int constant 0
makeDest.List.Append(makeType, nodintconst(0)) // make([]<T>, 0))

// create the "variable" where the result of make will be stored
filtered := temp(source.Type)
l = append(l, nod(OAS, filtered, makeDest)) // filtered = make([]<T>, 0)
\end{gocode}
	\caption{Illustrating the difference between Go code and it's AST code}
\end{listing}



\section{Functional Check}

As discussed in Chapter~\ref{sec:funcheck-theory}, to assist writing purely
functional code, a linter needs to be written that detects reassignments within
a Go program.

To get a grasp about the issues this linter should report, the first step
is to capture some examples, cases that should be matched against.

\subsection{Examples}

The simplest cases are standalone reassignments and assignment operators:
\begin{gocode}
x := 5
x = 6 // forbidden
// or
var y = 5
y = 6   // forbidden
y += 6  // forbidden
y <<= 2 // forbidden
y++     // forbidden
\end{gocode}

Where the statements with a \mintinline{go}|// forbidden
| comment should be reported.

Adding block scoping to this, shadowing the old variable needs to be allowed:
\begin{gocode}
x := 5
{
	x = 6  // forbidden, changing the old value
	x := 6 // allowed, as this shadows the old variable
}
\end{gocode}

What should be illegal is to declare the variable first and then assign a
value to it:
\begin{gocode}
var x int
x = 6 // forbidden
\end{gocode}

The exception here are functions, as they need to be declared first in order
to recursively call them:
\begin{gocode}
var f func()
f = func() {
	f()
}
\end{gocode}

Furthermore, the linter also needs to be able to handle multiple variables
at once:
\begin{gocode}
var f func()
x, f, y := 1, func() { f() }, 2
\end{gocode}

Loops should be reported too, as they are using reassignments internally:
\begin{gocode}
for i := 0; i < 5; i++ { // forbidden
	for i != 3 { // forbidden
		for { // allowed
			// ...
		}
	}
}
\end{gocode}

All the aforementioned examples and more can be found in the testcases for funcheck\autocite{funcheck-examples}.

\subsection{Building a linter}

The Go ecosystem already provides an official library for building code analysis tools,
the `analysis' package from the Go Tools repository\autocite{go-analysis}. With this package,
implementing a static code analyser is being reduced to writing the actual AST node analysis.

To define an analysis, a variable of type \mintinline{go}|*analysis.Analyzer| has to be declared:

\begin{gocode}
var Analyzer = &analysis.Analyzer{
	Name: "assigncheck",
	Doc:  "reports reassignments",
	Run:  func(*analysis.Pass) (interface{}, error)
}
\end{gocode}

The necessary steps are now adding the `Run' function and registering the analyser
in the \mintinline{go}|main()| function.

The `Run' function takes an \mintinline{go}|*analysis.Pass| type. The Pass provides
information about the package that is being analysed and some helper-functions to report
diagnostics.

With `analysis.Pass.Files` and the help of the `go/ast` package, traversing the syntax
tree of every file in a package is made extremely convenient:

\begin{gocode}
for _, file := range pass.Files {
	ast.Inspect(file, func(n ast.Node) bool {
		// node analysis here
	})
}
\end{gocode}

To implement funcheck as described, five different AST node types need to be
taken care of. The simpler ones are
\mintinline{go}|*ast.IncDecStmt|, \mintinline{go}|*ast.ForStmt| and \mintinline{go}|*ast.RangeStmt|.
An `IncDecStmt' node is a \mintinline{go}|x++| or \mintinline{go}|x--|
expression and should always be reported.
`ForStmt' and `RangeStmt' are similar; a `RangeStmt' is a `for' loop with the
\mintinline{go}|range| keyword instead of an init-, condition- and post-stametent.

Both of these loop-types need to be reported explicitly as they do not show up
as reassignments in the AST.
The basic building blocks for our is the following \mintinline{go}|switch|
statement:
\begin{code}
	\gofilerange{../work/funcheck/assigncheck/assigncheck.go}{start-basictypes}{end-basictypes}
	\caption{Handling the basic AST types in funcheck}
\end{code}
The remaining two node types are \mintinline{go}|*ast.DeclStmt| and \mintinline{go}|*ast.AssignStmt|.
They are not as simple to handle, which is why they are covered in their own chapters.

\subsection{Detecting reassignments}

To recapitulate, the goal of this step is to detect all assignments except blank identifiers
(discarded values cannot be mutated) and function literals, if the function is declared in the
last statement\footnote{This rule is to simplify the logic of the checker and make it easier
    for developers to read the code. It means that no code may be between \mintinline{go}|var f func|
and \mintinline{go}|f = func() { ... }|.}.

To detect such reassignments, funcheck iterates over all identifiers on the left-hand side
of an assignment statement.

On the left-hand side of an assignment is a list of expressions. These expressions can be
identifiers, index expressions (\mintinline{go}|*ast.IndexExpr|, for map and slice access),
a `star expression' (\mintinline{go}|*ast.StarExpr|\footnote{star expressions
    are expressions that are prefixed by an asterisk, dereferencing a pointer. For example
\mintinline{go}|*x = 5|, if \mintinline{go}|x| is of type \mintinline{go}|*int|.}) or others.

If the expression is not an identifier, the assignment must be a reassignment, as all non-identifier
expressions contain an already declared identifier. For example, the slice index expression
\mintinline{go}|s[5]| is of type \mintinline{go}|*ast.IndexExpr|:
\begin{gocode}
// An IndexExpr node represents an expression followed by an index.
IndexExpr struct {
	X      Expr      // expression
	Lbrack token.Pos // position of "["
	Index  Expr      // index expression
	Rbrack token.Pos // position of "]"
}
\end{gocode}

Where \mintinline{go}|IndexExpr.X| is our identifier `s' (of type \mintinline{go}|*ast.Ident|)
and a \mintinline{go}|IndexExpr.Index| is \mintinline{go}|5| (of type \mintinline{go}|*ast.BasicLit|).

As these nested identifiers already need to be declared beforehand (else they could not be used
in the expression), all expressions on the left-hand side of an assignment that are not identifiers
are reassignments.

Identifiers are the only expressions that can occur in declarations and reassignments. A naive
approach would be to check for the colon in a short variable declaration (\mintinline{go}|:=|).
However, as touched upon in Chapter~\ref{sec:multi-assign}, even short variable declarations may
contain redeclarations, if at least one variable is new.

Thus, another approach is needed.

Every identifier (an AST node with type \mintinline{go}|*ast.Ident|) contains an object\footnote{`An
    object describes a named language entity such as a package, constant, type, variable,
function (incl. methods), or a label'\autocite{go-ast-object}.} that links to the declaration.

This declaration, of whatever type it may be, always has a position (and a corresponding function
to retrieve that position) in the source file.

A reassignment is detected if an identifier's declaration position does not match the assignment's
position (indicating that the variable is being assigned at a different place to where it is
declared).

This is illustrated in the code block~\ref{code:assign-pos}. What can be clearly seen is that in
the assignment \mintinline{go}|y = 3|, \mintinline{go}|y|'s declaration refers to the position
of the first assignment \mintinline{go}|x, y := 1, 2|, the position where \mintinline{go}|y| has
been declared.
\begin{listing}
    \begin{gocode}
Assignment "x, y := 1, 2": 2958101
        Ident "x": 2958101
                Decl "x, y := 1, 2": 2958101
        Ident "y": 2958104
                Decl "x, y := 1, 2": 2958101
Assignment "y = 3": 2958115
        Ident "y": 2958115
                Decl "x, y := 1, 2": 2958101
    \end{gocode}
\caption{Illustration of an assignment node and corresponding positions\autocite{ast-positions}\label{code:assign-pos}}
\end{listing}

As this technique works on an identifier level, multi-variable declarations or assignments
can be verified without any additional effort.

If a variable in a short variable declaration is being reassigned, the variable's `Declaration'
field will point to the original position of its declaration, which can be easily detected
(as shown in code block~\ref{code:assign-pos}).

\subsection{Handling function declarations}

In contrast to all other variable types, function variables may be `reassigned' once.
As discussed in Chapter~\ref{sec:func-reassign}, this is to allow recursive function
literals. Detecting and not reporting these assignments is a two-step process, as two
consecutive AST nodes need to be inspected.

The first step is to detect function declarations; statements of the form
\mintinline{go}|var f func() |. Should such a statement be encountered,
its position is saved for the following AST node.

In the consecutive AST node it is ensured that, if the node is an assignment and
the assignee identifier is of type function literal, the position matches the
previously saved one.

The position of the declaration and AST node structure can be seen in~\ref{code:func-reassign}

\begin{listing}
    \begin{gocode}
Declaration "var f func() int": 2958142
        Ident "f func() int": 2958146
Assignment "f = func() int { return y }": 2958160
        Ident "f": 2958160
                Decl "f func() int": 2958146
\end{gocode}
	\caption{Illustration of a function literal assignment\autocite{ast-positions}\label{code:func-reassign}}
\end{listing}

With this technique it is possible to excempt functions from the reassignment rule.

\subsection{Testing Funcheck}

The analysis-package is distributed with a subpackage `analysistest'. This package makes
it extremely simple to test a code analysis tool.

By providing testdata and the expected messages from funcheck in a structured way, all
that is needed to test funcheck is:

\begin{code}
\begin{gocode}
package assigncheck

import (
        "testing"

        "golang.org/x/tools/go/analysis/analysistest"
)

func TestRun(t *testing.T) {
        analysistest.Run(t, analysistest.TestData(), Analyzer)
}
\end{gocode}
    \caption{Testing a code analyser with the `analysistest' package}
\end{code}

The library expects the testdata in the current directory in a folder named `testdata' and
then spawns and executes the analyser on the files in that folder. Comments in those files
are used to describe the expected message:

\begin{gocode}
x := 5
fmt.Println(x)
x = 6 // want `^reassignment of x$`
fmt.Println(x)
\end{gocode}

This will ensure that on the line \mintinline{go}|x = 6| an error message is reported that says
`reassignment of x'.



\chapter{Application}
\label{ch:application} % chktex 24
% -*- mode: latex; coding: utf-8; TeX-master: ../thesis -*-
% !TEX TS-program = pdflatexmk
% !TEX encoding = UTF-8 Unicode
% !TEX root = ../thesis.tex

\section{Refactoring the Prettyprint Package}

\newglossaryentry{stdout}{name=stdout, description={Standard Output, the default
output stream for programs}}

The code blocks~\ref{code:assign-pos} and~\ref{code:func-reassign} have been
generated by a small package `prettyprint' contained in the funcheck repository.

To see how the newly bult-in functions and funcheck can be used, we refactor `prettyprint'
to a purely functional version.
The current version of the package is written in what could be considered idiomatic
Go\footnote{
	Although there is no exact definition of what idiomatic Go is, so this interpretation
	could be challenged. It is idiomatic Go code to the author of this thesis.
}. %TODO?


The prettyprinter is based on the same framework as assigncheck\footnote{Assigncheck
is the main package for funcheck and checks the reassignments}, but instead
of reporting anything, it prints AST information to \gls{stdout}.

Similarily to assigncheck, the main logic of the package is within a
function literal that is being passed to the \mintinline{go}|ast.Inspect|
function.

Prettyprint only checks two AST node types, \mintinline{go}|*ast.DeclStmt|
(declarations) and \mintinline{go}|*ast.AssignStmt| (assignments).

For example, for the program
\begin{gocode}
package main

import "fmt"

func main() {
	x, y := 1, 2
	y = 3
	fmt.Println(x, y)
}
\end{gocode}
the following AST information is printed:

\begin{gocode}
Assignment "x, y := 1, 2": 2958101
		Ident "x": 2958101
				Decl "x, y := 1, 2": 2958101
		Ident "y": 2958104
				Decl "x, y := 1, 2": 2958101
Assignment "y = 3": 2958115
		Ident "y": 2958115
				Decl "x, y := 1, 2": 2958101
\end{gocode}

To refactor it to a purely functional version, funcheck can be used to
list statements that are not functional:

\begin{bashcode}
$> funcheck .
prettyprint.go:20:2: internal reassignment (for loop) in "for _, file := range pass.Files { ... }"
prettyprint.go:30:5: internal reassignment (for loop) in "for i := range decl.Specs { ... }"
prettyprint.go:53:5: internal reassignment (for loop) in "for _, expr := range as.Lhs { ... }"
\end{bashcode}
As can be seen in the output, the package uses 3 `for' loops to range over
slices. However, there are no other re-assignments of variables in the code.

The code to print declarations is as shown in~\ref{code:decl-printing}.

\begin{code}
	\captionof{listing}{Pretty-printing declarations in idiomatic Go}
	\label{code:decl-printing}
\begin{gocode}
func checkDecl(as *ast.DeclStmt, fset *token.FileSet) {
	fmt.Printf("Declaration %q: %v\n", render(pass.Fset, as), as.Pos())
	decl, ok := as.Decl.(*ast.GenDecl)
	if !ok {
		break
	}

	for i := range decl.Specs {
		val, ok := decl.Specs[i].(*ast.ValueSpec)
		if !ok {
			continue
		}

		if val.Values != nil {
			continue
		}

		if _, ok := val.Type.(*ast.FuncType); !ok {
			continue
		}

		fmt.Printf("\tIdent %q: %v\n", render(pass.Fset, val), val.Names[0].Pos())
	}
}
\end{gocode}
\end{code}
To convert this for-loop appropriately, the new built-in `foldl' can be used.
To recapitulate, the `foldl' function is being defined as:
\begin{gocode}
func foldl(fn func(Type1, Type) Type1, acc Type1, slice []Type) Type1
\end{gocode}
As `foldl' requires a return type, we introduce a dummy type `null', which
is just an empty struct:
\begin{gocode}
type null struct{}
\end{gocode}
Now the code within the foor loop can be used to create a function literal:
\begin{gocode}
check := func(_ null, spec ast.Spec) (n null) {
	// implementation
}
\end{gocode}
There are two subtleties in regards to the introduced null type:
First, the null value that is being passed as an argument is being discarded
by the use of an empty identifier.
Secondly, the return value is `named', which means the variable `n' is
already declared in the function block. Because of this, `naked returns' can
be used, so there is no need to specify which variable is being returned.

The snippet~\ref{code:decl-printing} thus can be translated to

\begin{code}
	\captionof{listing}{Pretty-printing declarations in functional Go}
	\begin{gocode}
func checkDecl(as *ast.DeclStmt, fset *token.FileSet) {
	fmt.Printf("Declaration %q: %v\n", render(fset, as), as.Pos())

	check := func(_ null, spec ast.Spec) (n null) {
		val, ok := spec.(*ast.ValueSpec)
		if !ok {
			return
		}

		if val.Values != nil {
			return
		}

		if _, ok := val.Type.(*ast.FuncType); !ok {
			return
		}

		fmt.Printf("\tIdent %q: %v\n", render(fset, val), val.Names[0].Pos())
		return
	}

	if decl, ok := as.Decl.(*ast.GenDecl); ok {
		_ = foldl(check, null{}, decl.Specs)
	}
}
\end{gocode}
\end{code}
The for-loop has been replaced by a `foldl', where we pass a function closure
that contains the actual processing.

While this still looks similar to the original example, this is mostly due to
the `if' statements. In Haskell, pattern matching would be used and nil checks
could be omitted entirely. Also, as Haskell's type system is more advanced, the
handling of those would be different too.

However, the goal of this thesis is to make functional code look more familiar
to programmers that are used to imperative code.
And while it may not look like it, the code does not use any mutation of
variables\footnote{Libraries may do, but the scope is not to rewrite any existing
libraries.}, for loops or global state. Therefore, it can be concluded that this
snippet is purely functional as per the definition from Chapter~\ref{sec:func-purity}.

\section{Quicksort}

In Chapter~\ref{code:haskell-quicksort}, a naive implementation of the Quicksort sorting
algorithm has been introduced.
Implementing this algorithm in Go is now straightforward and the similarities between
the Haskell implementation and the functional Go implementation are striking:

\begin{code}
	\captionof{listing}{Quicksort Implementations compared}
	\begin{gocode}
func quicksort(p []int) []int {
	if len(p) == 0 {
		return []int{}
	}

	lesser := filter(func(x int) bool { return p[0] > x }, p[1:])
	greater := filter(func(x int) bool { return p[0] <= x }, p[1:])

	return append(quicksort(lesser), prepend(p[0], quicksort(greater))...)
}
\end{gocode}
\begin{haskellcode}
quicksort :: Ord a => [a] -> [a]
quicksort []     = []
quicksort (p:xs) = (quicksort lesser) ++ [p] ++ (quicksort greater)
    where
        lesser  = filter (< p) xs
        greater = filter (>= p) xs
\end{haskellcode}
\end{code}

Again, the Go implementation bridges the gap between being imperative and functional,
while still being obvious about the algorithm.
Furthermore, as expected, when inspecting the code with funcheck, no non-functional
constructs are reported.

\section{Comparison to Java Streams}

In Java 8, concepts from functional programing have been introduced to the language.
The major new feature was Lambda Expressions --- anonymous function literals --- and
streams. Streams are an abstract layer to process data in a functional way, with `map',
`filter', `reduce' and more.

It is similar to the new built-in functions in this thesis:

\begin{code}
	\captionof{listing}{Comparision Java Streams and Functional Go}
	\begin{javacode}
List<Integer> even = list.stream()
	.filter(x -> x % 2 == 0)
	.collect(Collectors.toList());
	\end{javacode}
	\begin{gocode}
even := filter(
	func(x int) bool { return x%2 == 0 },
	list)
	\end{gocode}
\end{code}

The lambda-syntax in Java is more concise than Go's function literals, where the
complete function header has to be provided\footnote{There is an open proposal
	to add a lightweight anonymous function syntax to Go 2, which, if implemented,
would resolve the verbosity\autocite{go-lambdas}}. Java also has a more verbose way
to declare anonymous functions, which is creating an anonymous class:

\begin{javacode}
printPersons(
    roster,
    new CheckPerson() {
        public boolean test(Person p) {
            return p.getGender() == Person.Sex.MALE
                && p.getAge() >= 18
                && p.getAge() <= 25;
        }
    }
);
\end{javacode}
\autocite{java-lambda-expressions}

If this is used, Go's function literal syntax has an edge over Java's when it comes to
readability and conciseness.

Furthermore, the conversion to a stream and back to a list
(\mintinline{java}|list.stream()| and \mintinline{java}|.collect(Collectors.toList())|)
is not required in Go, as the operations all work on slices. Here, only having a single
list-like type built into the language is an advantage, as the mental overhead to convert
the list only to run a `filter' function can be avoided.

Apart from syntactical differences, Java Streams contain all the functions that
have been added as built-ins to Go too, and more.

However, Java's Syntax is arguably more complex than Go. An indicator for this might be
the language specification; Go's Language Specification is roughly 110 pages, while
Java's specification is more than 700 pages\footnote{
	The Java 8 Specification is 724\autocite{java-8-spec}, the Java 14
	Specification 774\autocite{java-14-spec} pages. An interesting sidenote may
	be that Go's specification grew by roughly 15 pages in 8 years (Go 1.0.0, 2012 - Go
	1.14.0, 2020), Java's by almost 170 pages between Java 7 (2013) and Java 14 (2020).
},
more than 6 times the size.


\IfLanguageName{nswissgerman}{\chapter{Resultate}}{\chapter{Results}}
\label{ch:results} % chktex 24
% -*- mode: latex; coding: utf-8; TeX-master: ../thesis -*-
% !TEX TS-program = pdflatexmk
% !TEX encoding = UTF-8 Unicode
% !TEX root = ../thesis.tex

\todo[inline]{(Zusammenfassung der Resultate)}


\IfLanguageName{nswissgerman}{\chapter{Diskussion}}{\chapter{Discussion}}
\label{ch:discussion} % chktex 24
% -*- mode: latex; coding: utf-8; TeX-master: ../thesis -*-
% !TEX TS-program = pdflatexmk
% !TEX encoding = UTF-8 Unicode
% !TEX root = ../thesis.tex

\todo[inline]{%
  \quad --{
    Bespricht die erzielten Ergebnisse bezüglich ihrer Erwartbarkeit,
    Aussagekraft und Relevanz
  } \\
  \quad -- Interpretation und Validierung der Resultate \\
  \quad -- Rückblick auf Aufgabenstellung, erreicht bzw. nicht erreicht \\
  \quad -- {
    Legt dar, wie an die Resultate (konkret vom Industriepartner oder
    weiteren Forschungsarbeiten; allgemein) angeschlossen werden kann; legt
    dar, welche Chancen die Resultate bieten
  }
}



%%%%%%%%%%%%%%%%%%%%%%%%%%%%%%%%%%%%%%%%
\backmatter % chktex 1

% \let\clearpage\relax
% \vspace{-4em}
\printbibliography
% \endgroup

\renewcommand{\lstlistlistingname}{List of source codes}

\lstlistoflistings

% \begingroup
% \let\clearpage\relax
% \vspace{-4em}
\listoffigures
% \endgroup

% \begingroup
% \let\clearpage\relax
% \vspace{-4em}
\listoftables
\printglossaries
% \endgroup
\cleardoublepage % chktex 1


%%%%%%%%%%%%%%%%%%%%%%%%%%%%%%%%%%%%%%%%
%\appendix
\begin{appendices}
\section{Example for Functional Options}\label{appendix:funcopts}
\begin{code}
    \captionof{listing}{Functional Options for a simple Webserver}
    \gofile{../work/examples/functional-options/main.go}
\end{code}

\section{Analysis of function occurrences in Haskell code}\label{appendix:function-occurrences}
The results of the analysis have been aquired by running the following command
from the root of the git repository\cite{git-repo}:
\begin{bashcode}
./work/common-list-functions/count-function.sh "map " " : " "fold" "filter " "reverse " "take " "drop " "maximum" "sum " "zip " "product " "minimum " "reduce "
\end{bashcode}

\section{Mutating variables in Go}\label{appendix:mutation}
\begin{code}
	\captionof{listing}{Example on how to mutate complex types in Go}
	\gofile{../work/examples/mutate/main.go}
\end{code}

\section{Shadowing variables in Go}\label{appendix:shadowing}
\begin{code}
	\captionof{listing}{Example on how shadowing works on block scopes}
	\gofile{../work/examples/shadowing/main.go}
\end{code}

\section{Workaround for the missing foldl' implementation in Go}\label{appendix:foldl-go}
\begin{code}
	\captionof{listing}{Working around the missing foldl implementation in Go}
	\label{code:foldl-go}
	\gofile{../work/examples/foldl-workaround/main.go}
	\begin{bashcode}
$> fgo run .
0
panic: runtime error: invalid memory address or nil pointer dereference
[signal SIGSEGV: segmentation violation code=0x1 addr=0x0 pc=0x109e945]

goroutine 1 [running]:
main.what(0x2, 0x0, 0x2)
		/tmp/map/main.go:16 +0x5
main.main()
		/tmp/map/main.go:12 +0x187
exit status 2
	\end{bashcode}
\end{code}

\section{Prettyprint implementation}\label{appendix:prettyprint-func}
\begin{code}
	\captionof{listing}{The original prettyprint implementation}
	\gofile{../work/funcheck/prettyprint/prettyprint.go}
\end{code}
\begin{code}
	\captionof{listing}{The refactored, functional prettyprint implementation}
	\begin{gocode}
package prettyprint

import (
	"bytes"
	"fmt"
	"go/ast"
	"go/printer"
	"go/token"

	"golang.org/x/tools/go/analysis"
)

var Analyzer = &analysis.Analyzer{
	Name: "prettyprint",
	Doc:  "prints positions",
	Run:  run,
}

type null struct{}

func checkDecl(as *ast.DeclStmt, fset *token.FileSet) {
	fmt.Printf("Declaration %q: %v\n", render(fset, as), as.Pos())

	check := func(_ null, spec ast.Spec) (n null) {
		val, ok := spec.(*ast.ValueSpec)
		if !ok {
			return
		}
		if val.Values != nil {
			return
		}
		if _, ok := val.Type.(*ast.FuncType); !ok {
			return
		}
		fmt.Printf("\tIdent %q: %v\n", render(fset, val), val.Names[0].Pos())
		return
	}

	if decl, ok := as.Decl.(*ast.GenDecl); ok {
		_ = foldl(check, null{}, decl.Specs)
	}
}

func checkAssign(as *ast.AssignStmt, fset *token.FileSet) {
	fmt.Printf("Assignment %q: %v\n", render(fset, as), as.Pos())

	check := func(_ null, expr ast.Expr) (n null) {
		ident, ok := expr.(*ast.Ident) // Lhs always is an "IdentifierList"
		if !ok {
			return
		}

		fmt.Printf("\tIdent %q: %v\n", ident.String(), ident.Pos())

		switch {
		case ident.Name == "_":
			fmt.Printf("\t\tBlank Identifier!\n")
		case ident.Obj == nil:
			fmt.Printf("\t\tDecl is not in the same file!\n")
		default:
			// make sure the declaration has a Pos func and get it
			declPos := ident.Obj.Decl.(ast.Node).Pos()
			fmt.Printf("\t\tDecl %q: %v\n", render(fset, ident.Obj.Decl), declPos)
		}

		return
	}
	_ = foldl(check, null{}, as.Lhs)
}

func run(pass *analysis.Pass) (interface{}, error) {
	inspect := func(_ null, file *ast.File) (n null) {
		ast.Inspect(file, func(n ast.Node) bool {
			switch as := n.(type) {
			case *ast.DeclStmt:
				checkDecl(as, pass.Fset)
			case *ast.AssignStmt:
				checkAssign(as, pass.Fset)
			}
			return true
		})
		return
	}
	_ = foldl(inspect, null{}, pass.Files)

	return nil, nil
}

// render returns the pretty-print of the given node
func render(fset *token.FileSet, x interface{}) string {
	var buf bytes.Buffer
	if err := printer.Fprint(&buf, fset, x); err != nil {
		panic(err)
	}
	return buf.String()
}
	\end{gocode}
\end{code}

\section{Compiling and using functional Go}

To compile and use the changes to the Go compiler that have been implemented in
this thesis, these instructions should be followed.

First, check out the Go source code:

\begin{bashcode}
$> git clone https://github.com/tommyknows/go.git
$> cd go
$> git checkout bachelor-thesis
\end{bashcode}

\subsection{With a working Go installation}

If you already have a working Go installation on your system, the following
steps provide a way to get functional go up and running in the same way
a normal go installation does.

These steps need to be executed from within the checked out `go' git
repository on the branch `bachelor-thesis'.

Build the functional Go binary and configure the environment:
\begin{bashcode}
$> cd ./src
$> ./make.bash
$> ln -s $(realpath $(pwd)/../bin/go) /usr/local/bin/fgo
$> go env -w GOROOT=$(realpath $(pwd)/..)
\end{bashcode}

The `go env' command sets the GOROOT to point to the newly compiled tools
and source code and is valid for the current shell session only.

After these steps, the binary `fgo' can be used to test and build
functional go code. `fgo' is not different to the normal `go' command, so
all commands that work with the normal `go' command should also work with
the `fgo' command.

\begin{bashcode}
$> cd <code directory>
$> fgo test ./...
$> fgo build ./...
\end{bashcode}


\subsection{With Docker}

If Docker is installed on your system, you can follow these steps from within
the checked out `go' git repository on the branch `bachelor-thesis'.
The downside of this approach is that the installation can only be used in
the container, and directories have to be mounted into the container to compile
projects.

\begin{bashcode}
$> CODEDIR=<some directory>
$> docker build . -t fgo
$> docker run --rm --it -v $CODEDIR:/work fgo bash
\end{bashcode}
These commands build Go in the container and setup the environment. To build
projects with the functional Go installation, set \mintinline{bash}|CODEDIR|
to the path where your code resides.

Then, from within the container, the binary `fgo' can be used to test and build
functional go code. `fgo' is not different to the normal `go' command, but is the
go binary that has been compiled with the functional additions.

\begin{bashcode}
$> cd /work
$> fgo test ./...
$> fgo build ./...
\end{bashcode}


% - Add your appendix here:

\todo[inline]{
  Anhang/Appendix:

  \quad -- Projektmanagement: \\ % chktex 8
  \qquad -- Offizielle Aufgabenstellung, Projektauftrag \\ % chktex 8
  \qquad -- (Zeitplan) \\ % chktex 8
  \qquad -- (Besprechungsprotokolle oder Journals) % chktex 8

  \quad -- Weiteres: \\ % chktex 8
  \qquad -- CD/USB-Stick mit dem vollständigen Bericht als PDF-File inklusive Film- und Fotomaterial \\ % chktex 8
  \qquad -- (Schaltpläne und Ablaufschemata) \\ % chktex 8
  \qquad -- (Spezifikation u. Datenblätter der verwendeten Messgeräte und/oder Komponenten) \\ % chktex 8
  \qquad -- (Berechnungen, Messwerte, Simulationsresultate) \\ % chktex 8
  \qquad -- (Stoffdaten) \\ % chktex 8
  \qquad -- (Fehlerrechnungen mit Messunsicherheiten) \\ % chktex 8
  \qquad -- (Grafische Darstellungen, Fotos) \\ % chktex 8
  \qquad -- (Datenträger mit weiteren Daten (z. B. Software-Komponenten) inkl. Verzeichnis der auf diesem Datenträger abgelegten Dateien) \\ % chktex 8
  \qquad -- (Softwarecode) % chktex 8
}


%\section{Example for Functional Options}\label{appendix:funcopts}
%\begin{code}
    %\captionof{listing}{Functional Options for a simple Webserver}
    %\gofile{../work/examples/functional-options/main.go}
%\end{code}

%\section{Analysis of function occurrences in Haskell code}\label{appendix:function-occurrences}
%The results of the analysis have been aquired by running the following command
%from the root of the git repository\cite{git-repo}:
%\begin{bashcode}
%./work/common-list-functions/count-function.sh "map " " : " "fold" "filter " "reverse " "take " "drop " "maximum" "sum " "zip " "product " "minimum " "reduce "
%\end{bashcode}

%\section{Mutating variables in Go}\label{appendix:mutation}
%\begin{code}
	%\captionof{listing}{Example on how to mutate complex types in Go}
	%\gofile{../work/examples/mutate/main.go}
%\end{code}

%\section{Shadowing variables in Go}\label{appendix:shadowing}
%\begin{code}
	%\captionof{listing}{Example on how shadowing works on block scopes}
	%\gofile{../work/examples/shadowing/main.go}
%\end{code}

%\section{Workaround for the missing foldl' implementation in Go}\label{appendix:foldl-go}
%\begin{code}
	%\captionof{listing}{Working around the missing foldl implementation in Go}
	%\label{code:foldl-go}
	%\gofile{../work/examples/foldl-workaround/main.go}
	%\begin{bashcode}
%$> fgo run .
%0
%panic: runtime error: invalid memory address or nil pointer dereference
%[signal SIGSEGV: segmentation violation code=0x1 addr=0x0 pc=0x109e945]

%goroutine 1 [running]:
%main.what(0x2, 0x0, 0x2)
		%/tmp/map/main.go:16 +0x5
%main.main()
		%/tmp/map/main.go:12 +0x187
%exit status 2
	%\end{bashcode}
%\end{code}

%\section{Prettyprint implementation}\label{appendix:prettyprint-func}
%\begin{code}
	%\captionof{listing}{The original prettyprint implementation}
	%\gofile{../work/funcheck/prettyprint/prettyprint.go}
%\end{code}
%\begin{code}
	%\captionof{listing}{The refactored, functional prettyprint implementation}
	%\begin{gocode}
%package prettyprint

%import (
	%"bytes"
	%"fmt"
	%"go/ast"
	%"go/printer"
	%"go/token"

	%"golang.org/x/tools/go/analysis"
%)

%var Analyzer = &analysis.Analyzer{
	%Name: "prettyprint",
	%Doc:  "prints positions",
	%Run:  run,
%}

%type null struct{}

%func checkDecl(as *ast.DeclStmt, fset *token.FileSet) {
	%fmt.Printf("Declaration %q: %v\n", render(fset, as), as.Pos())

	%check := func(_ null, spec ast.Spec) (n null) {
		%val, ok := spec.(*ast.ValueSpec)
		%if !ok {
			%return
		%}
		%if val.Values != nil {
			%return
		%}
		%if _, ok := val.Type.(*ast.FuncType); !ok {
			%return
		%}
		%fmt.Printf("\tIdent %q: %v\n", render(fset, val), val.Names[0].Pos())
		%return
	%}

	%if decl, ok := as.Decl.(*ast.GenDecl); ok {
		%_ = foldl(check, null{}, decl.Specs)
	%}
%}

%func checkAssign(as *ast.AssignStmt, fset *token.FileSet) {
	%fmt.Printf("Assignment %q: %v\n", render(fset, as), as.Pos())

	%check := func(_ null, expr ast.Expr) (n null) {
		%ident, ok := expr.(*ast.Ident) // Lhs always is an "IdentifierList"
		%if !ok {
			%return
		%}

		%fmt.Printf("\tIdent %q: %v\n", ident.String(), ident.Pos())

		%switch {
		%case ident.Name == "_":
			%fmt.Printf("\t\tBlank Identifier!\n")
		%case ident.Obj == nil:
			%fmt.Printf("\t\tDecl is not in the same file!\n")
		%default:
			%// make sure the declaration has a Pos func and get it
			%declPos := ident.Obj.Decl.(ast.Node).Pos()
			%fmt.Printf("\t\tDecl %q: %v\n", render(fset, ident.Obj.Decl), declPos)
		%}

		%return
	%}
	%_ = foldl(check, null{}, as.Lhs)
%}

%func run(pass *analysis.Pass) (interface{}, error) {
	%inspect := func(_ null, file *ast.File) (n null) {
		%ast.Inspect(file, func(n ast.Node) bool {
			%switch as := n.(type) {
			%case *ast.DeclStmt:
				%checkDecl(as, pass.Fset)
			%case *ast.AssignStmt:
				%checkAssign(as, pass.Fset)
			%}
			%return true
		%})
		%return
	%}
	%_ = foldl(inspect, null{}, pass.Files)

	%return nil, nil
%}

%// render returns the pretty-print of the given node
%func render(fset *token.FileSet, x interface{}) string {
	%var buf bytes.Buffer
	%if err := printer.Fprint(&buf, fset, x); err != nil {
		%panic(err)
	%}
	%return buf.String()
%}
	%\end{gocode}
%\end{code}

%% - Add your appendix here:

%\todo[inline]{
  %Anhang/Appendix:

  %\quad -- Projektmanagement: \\ % chktex 8
  %\qquad -- Offizielle Aufgabenstellung, Projektauftrag \\ % chktex 8
  %\qquad -- (Zeitplan) \\ % chktex 8
  %\qquad -- (Besprechungsprotokolle oder Journals) % chktex 8

  %\quad -- Weiteres: \\ % chktex 8
  %\qquad -- CD/USB-Stick mit dem vollständigen Bericht als PDF-File inklusive Film- und Fotomaterial \\ % chktex 8
  %\qquad -- (Schaltpläne und Ablaufschemata) \\ % chktex 8
  %\qquad -- (Spezifikation u. Datenblätter der verwendeten Messgeräte und/oder Komponenten) \\ % chktex 8
  %\qquad -- (Berechnungen, Messwerte, Simulationsresultate) \\ % chktex 8
  %\qquad -- (Stoffdaten) \\ % chktex 8
  %\qquad -- (Fehlerrechnungen mit Messunsicherheiten) \\ % chktex 8
  %\qquad -- (Grafische Darstellungen, Fotos) \\ % chktex 8
  %\qquad -- (Datenträger mit weiteren Daten (z. B. Software-Komponenten) inkl. Verzeichnis der auf diesem Datenträger abgelegten Dateien) \\ % chktex 8
  %\qquad -- (Softwarecode) % chktex 8
%}

\end{appendices}

\end{document}
