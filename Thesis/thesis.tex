% -*- mode: latex; coding: utf-8 -*-
% !TEX TS-program = pdflatexmk
% !TEX encoding = UTF-8 Unicode

\documentclass[%
  a4paper,
  twoside,
  numbers=noenddot,
  parskip=half+,
  open=any,
  headsepline,
  english, % german, english
  ba  % ba, pa
]{zhawthesis}

\usepackage{etoolbox}


%%%%%%%%%%%%%%%%%%%%%%%%%%%%%%%%%%%%%%%%
% Parameters
% - Adjust these to your needs:

\title{Solving all the world's problems}
\subtitle{\dots at once!}
\author{% Komma getrennt
    Ramon Rüttimann
}
\newcommand\twodigits[1]{\ifnum#1<10 0#1\else #1\fi}
\date{\twodigits{\the\day}.\twodigits{\number\month}.\the\year}

\major{Informatik}  % Studiengang
\zhawsemester{Spring 2020}
\zhawinstitute{init}
\zhawlogocolour{pantone2945}  % pantone2945, cmyk, sw
\mainsupervisor{Prof. G. Burkert}
\subsupervisor{Prof. K. Rege}

%%%%%%%%%%%%%%%%%%%%%%%%%%%%%%%%%%%%%%%%
% Base packages used by the template (any commonly used packages)

\usepackage{float}
\usepackage{graphicx}
\graphicspath{{figures/}}
\DeclareGraphicsExtensions{.pdf,.png,.jpg,.gif}

\usepackage{tabularx}
\usepackage{longtable}
\usepackage{booktabs}
\usepackage{todonotes}

%%%%%%%%%%%%%%%%%%%%%%%%%%%%%%%%%%%%%%%%
% Custom packages
% - Add packages used by your thesis here:

\usepackage{minted}
\usepackage[
    backend=biber,
    style=ieee,
]{biblatex}
%\usepackage{csquotes}
\AtBeginEnvironment{quote}{\itshape}
%\bibliography{thesis}
\addbibresource{thesis.bib}

\begin{document}

\frontmatter

\maketitle

\cleardoublepage % chktex 1


%%%%%%%%%%%%%%%%%%%%%%%%%%%%%%%%%%%%%%%%
% Declaration of Originality

\makedeclarationoforiginality % chktex 1


\cleardoublepage % chktex 1


%%%%%%%%%%%%%%%%%%%%%%%%%%%%%%%%%%%%%%%%

\IfLanguageName{nswissgerman}{\chapter{Zusammenfassung}}{\chapter{Summary}}
\label{ch:summary} % chktex 24
% -*- mode: latex; coding: utf-8; TeX-master: ../thesis -*-
% !TEX TS-program = pdflatexmk
% !TEX encoding = UTF-8 Unicode
% !TEX root = ../thesis.tex

Innerhalb der letzten zehn Jahre haben Konzepte und Ideen aus dem funktionalen
Programmieren im Alltag von vielen Entwicklern Fuss gefasst. Häufig wird
empfohlen, eine pur funktionale Programmiersprache wie zum Beispiel Haskell
zu lernen, um sich mit diesen Konzepten vertraut zu machen. Viele haben jedoch
Mühe, eine neue Syntax und ein neues Paradigma gleichzeitig zu lernen. Das Ziel
dieser Arbeit ist deswegen, einen einfacheren Einstieg in funktionales Programmieren
zu ermöglichen, dies mit Hilfe einer multiparadigmatischen Programmiersprache mit bekannter
Syntax.

Um dieses Ziel zu erreichen, wurde die Programmiersprache Go aufgrund ihrer
syntaktischen Simplizität und Vertrautheit gewählt.
Da Listen jedoch oft eine zentrale Rolle im funktionalen Programmieren einnehmen, ist ein
Nachteil dieser Wahl, dass Go keinen eingebauten List Datentyp besitzt. Zwar wird
dieser Nachteil durch Go's `Slices' gemildert, jedoch fehlen viele sogenannte `higher-order'
Funktionen um mit Listen zu arbeiten --- `map', `filter' und `reduce', um einige zu nennen.
Da Go's Typensystem keinen Polymorphismus bietet, müssen diese Funktionen im Compiler
implementiert werden, um eine möglichst benutzerfreundliche Verwendung zu ermöglichen.

Zusätzlich dazu wird die Bedeutung von `pure functional Programming' im Kontext dieser Arbeit
festgelegt und auf Basis dieser Definition das Code-Analyse Tool `funcheck' entwickelt, welches
nicht-funktionale Konstrukte im Programmcode meldet.

Mit den neuen built-in Funktionen `fmap', `filter', `foldr', `foldl' und `prepend',
sowie dem Linter `funcheck' erweist sich Go als geeignete Programmiersprache um
einen einfachen Einstieg in funktionales Programmieren zu ermöglichen. Der primäre Grund
spiegelt sich auch im Go Idiom `clear is better than clever' wider. Obwohl funktionaler
Go Code länger ist als in funktionalen Sprachen, ist dieser auch einfacher nachzuvollziehen.
Des Weiteren zeigt die Arbeit aber auch, dass es keinen Weg um eine pure funktionale Sprache
wie Haskell gibt, um sich funktionales Programmieren vollständig anzueignen.
Haskell's zwar ungewöhnliche, aber prägnante Syntax sowie das Design
der Sprache --- das Typensystem, Pattern Matching, die Purity Guarantees und vieles mehr ---
bilden hierfür eine solide und oft verwendete Grundlage.


\chapter{Abstract}
\label{ch:abstract} % chktex 24
% -*- mode: latex; coding: utf-8; TeX-master: ../thesis -*-
% !TEX TS-program = pdflatexmk
% !TEX encoding = UTF-8 Unicode
% !TEX root = ../thesis.tex

In the last decade, concepts from functional programming have grown in
importance within the wider, non-functional programming community.
Often it is recommended to learn a purely functional programming language
like Haskell to become familiar with these concepts.
However, many programmers struggle with the double duty
of learning a new paradigm and a new syntax at the same time. Because of
this, this paper expands on the idea of learning purely functional programming
with a multi-paradigm programming language with a familiar syntax. The Go
programming language has been the choice for this due to its syntactical
simplicity and familiarity.

The absence of a list datatype in Go is remediated by Go's slices.
However, Go is missing the typical higher-order functions ---
`map', `filter' and `fold' to name a few --- that are present in every
functional programming language and many other programming languages too. Due to
this, the most popular higher-order functions have been determined and, because of the
absence of polymorphism in Go, implemented as built-in functions in the compiler.

Furthermore, this paper specifies a definition of what pure functional programming is and
introduces `funcheck', a static code analysis tool that has been designed and implemented to
report constructs that are non-functional.

With the help of the newly built-in functions `fmap', `filter', `foldr', `foldl' and
`prepend', as well as `funcheck' to lint code, Go proves itself to be a
suitable language for getting started with functional programming.
At the same time, it also shows that there is no way around learning a
language like Haskell if fluency with functional programming concepts
is desired. The primary reasons are that, although it may be unusual, Haskell's
syntax is extremely concise, and that the language's design --- the type system,
pattern matching, the purity guarantees and more --- provides a very effective toolset
for purely functional programming.


\IfLanguageName{nswissgerman}{\chapter{Vorwort}}{\chapter{Preface}}
\label{ch:preface} % chktex 24
% -*- mode: latex; coding: utf-8; TeX-master: ../thesis -*-
% !TEX TS-program = pdflatexmk
% !TEX encoding = UTF-8 Unicode
% !TEX root = ../thesis.tex

As a part of my bachelor studies, I chose to attend a course on functional programming.
Having worked with Go for the last 3 years, first-class and higher-order functions
were not particularly new ideas to me. However, learning Haskell was, at the beginning,
overwhelming.

So I rewrote some exercises that I did not understand in Go.
%So what I did for some exercises that I could not understand completely is that
%I rewrote them in Go.
The result was more verbose; usually
roughly two to three times the lines of code for the same algorithm.
However, after writing the Go version, I understood not only the Go version,
but also the Haskell version.

After doing this a couple of times, I realised that I was constantly rewriting
the same higher-order functions with different types, but more or less the same
implementation. Thus, the idea of adding them as built-ins came up.

`Funcheck' then came into play when I wanted to build something in Go first and
then rewrite it in Haskell. It was hard to tell whether it was purely functional,
but it needed to be in order to be easier to write the implementation in Haskell.

This thesis is written based on my own struggles I had with Haskell, and it is
my hope that someday, someone may benefit from the work done in this thesis.

Special thanks to:

My supervisors Gerrit Burkert and Karl Rege for their support and guidance,
Tom Whiston for proofreading this thesis and improving my English,
Eva Kuske for the consultation on writing and
my employer nine (\href{http://nine.ch}{nine.ch}) for their flexible working times.


\cleardoublepage % chktex 1


%%%%%%%%%%%%%%%%%%%%%%%%%%%%%%%%%%%%%%%%
\mainmatter % chktex 1

\tableofcontents

\IfLanguageName{nswissgerman}{\chapter{Einleitung}}{\chapter{Introduction}}
\label{ch:introduction} % chktex 24
% -*- mode: latex; coding: utf-8; TeX-master: ../thesis -*-
% !TEX TS-program = pdflatexmk
% !TEX encoding = UTF-8 Unicode
% !TEX root = ../thesis.tex

\section{Learning Functional Programming}
\todo[inline]{
    rewrite first sentence. Also, Rust being 'more popular' is dangerous, too.
    Less vague expressions, "seems to be", "and many more".
    Java 8 added a lot of functional features, C\# too.
}
When Javascript and Python started to take off around 2010\autocite{python-popularity},
they also brought rise to a lot of concepts borrowed from functional programming.
Since then, many new multi-paradigm
languages have appeared and gotten more popular, as for example Go, Rust,
Kotlin and Dart.
Most of the languages mentioned support an imperative, object-oriented, as well as a functional programming style.
Rust, being the `most popular programming language'%\autocite{rust-loved}%
for 4 years in a row (2016--2019), has been
significantly influenced by functional programming languages\autocite{rust-functional} and borrows a lot of functional
concepts in idiomatic Rust code.

Learning a functional programming language increases fluency with these concepts and teaches a different
way to think and approach problems when programming. For those reasons, and many more, a lot of programmers
advocate learning a functional programming language.

Though the exact definition of what a \textit{purely} functional language consists of remains a controversy\autocite{functional-controversy},
the most popular, pure functional programming language seems to be Haskell\autocite{comparison-functional-languages}.

\section{Haskell}

Haskell, the \textit{lingua franca} amongst functional programmers, is a lazely-evaluated, purely functional programming
language. While Haskell's strengths stem from all it's features like type classes, type polymorphism, purity and more,
these features are also what makes Haskell famously hard to learn\autocite{haskell-hard-one}\autocite{haskell-hard-two}\autocite{haskell-hard-three}\autocite{haskell-hard-four}.

Beginner Haskell programmers face a very distinctive challenge in contrast to learning a new, non-functional programming language:
Not only do they need to learn a new language with an unusual syntax (compared to imperative or object-oriented languages), they
also need to change their way of thinking and reasoning about problems.
For example, the renowned quicksort-implementation from the Haskell Introduction Page\autocite{haskell-quicksort}:

\begin{haskellcode}
quicksort :: Ord a => [a] -> [a]
quicksort []     = []
quicksort (p:xs) = (quicksort lesser) ++ [p] ++ (quicksort greater)
    where
        lesser  = filter (< p) xs
        greater = filter (>= p) xs
\end{haskellcode}

While this is only a very short and clean piece of code, these 6 lines already pose many challenges to non-experienced Haskellers;

\begin{itemize}
    \item The function's signature with no `fn' or `func' statement as they often appear in imperative languages
    \item The pattern matching, which would be a `switch' statement or a chain of `if / else' conditions
    \item The deconstruction of the list within the pattern matching
    \item The functional nature of the program, passing `(< p)' (a function returning a function) to another function
    \item The function call to `filter' without paranthesised arguments and no clear indicator at which arguments
        it takes and which types are returned
\end{itemize}

Though some of these points are also available to programmers in imperative or object-oriented languages, the cumulative difference
is not to underestimate and adds to Haskell's steep learning curve.

 %TODO: talk about production-ready code and programmers?

\section{Goals}

\todo[inline]{unclear first sentence, not specific enough}

The goal is to solve the issue of the first steps in functional programming.
Learning a new paradigm and syntax at the same time can be daunting and discouraging for novices.
By using a modern, multi-paradigm language with a clear
and familiar syntax, the functional programming beginner should be able to focus on the paradigm
first, and then change to a language like Haskell to fully get into functional programming.

To ease the learning curve of functional programming, this thesis will consist of two parts:

\todo[inline]{should not be a list, for the flow}
\begin{itemize}
    \item Make a multi-paradigm language support functional programming as much as needed.
        The criteria for this language are:
    \begin{itemize}
        \item Easy, familiar syntax
        \item Be statically typed, as this makes it easier to reason about a program
        \item Have support for functional programming features like first class functions, currying
            and partial application
    \end{itemize}
    \item Create a linter that checks code on its functional purity. For this, some rules will have
        to be curated to define what pure functional code is.
\end{itemize}

It s not the goal to create a production-ready functional language, so runtime and performance requirements
can be ignored.

\section{Why Go}

The language of choice for this task is Go, a statically typed, garbage-collected programming language
designed at Google in 2009\autocite{golang-publish}. With its strong syntactic similarity to C, it should
be familiar to most programmers.
Go is an extremely verbose language with almost no syntactic sugar. This makes it a perfect fit to
grasp the concepts and trace the inner workings of functional programming.

There are, however, a few downsides of using Go:

% TODO: should not be a list.
\todo[inline]{shouldn't be a list either}
\begin{itemize}
    \item No polymorphism. Go 2 will likely have support for polymorphism, but at the time of writing,
        there is no implementation available.
    \item Missing implementations for common functions like `map', `filter', `reduce' and more.
    \item No list implementation. Go has `slices', which are `views' on arrays, but
        no list datatype.
\end{itemize}

\subsection{Go Slices}

Go's Slices can be viewed as an abstraction over arrays, to mitigate some of the weaknesses of arrays
compared to lists.

\begin{quote}
    Arrays have their place, but they're a bit inflexible, so you don't see them too often in Go code.
    Slices, though, are everywhere. They build on arrays to provide great power and convenience.\autocite{golang-slices}
\end{quote}

Slices can be visualised as a `struct' over an array:

\begin{gocode}
// NOTE: this type does not really exist, it
// is just to visualise how they are implemented.
type Slice struct {
    // the underlying "backing store" array
    array *[]T
    // the length of the slice / view on the
    //array
    len   int
    // the capacity of the array from the
    // starting index of the slice
    cap   int
}
\end{gocode}

With the `append' function, elements can be added to a slice. Should the underlying array not have enough
capacity left to store the new elements, a new array will be created and the data from the old array will
be copied into the new one. This happens transparently to the user.

\subsubsection{Using Slices}

`head', `tail' and `last' operations can be done with index expressions:

\begin{gocode}
// []<T> initialises a slice, while [n]<T> initialises an
// array, which is of fixed length n. One can also use `...'
// instead of a natural number, to let the compiler count
// the number of elements.
s := []string{"first", "second", "third"}
head := s[0]
tail := s[1:]
last := s[len(s)-1]
\end{gocode}

Adding elements or joining slices is achieved with `append':

\begin{gocode}
s := []string{"first", "second"}
s = append(s, "third", "fourth")
t := []string{"fifth", "seventh"}
s = append(s, t...)
// to prepend an element, one has to create a
// slice out of that element
s = append([]string{"zeroth"}, s...)
\end{gocode}

Append is a variadic function, meaning it takes \textit{n} elements. If the slice is of type \textit{[]<T>},
the appended elements have to be of type \textit{<T>}.

To join two lists, the second list is expanded into
variadic arguments.

More complex operations like removing elements, inserting elements in the middle or finding
elements in a slice require helper functions, which have also been documented in Go's
Slice Tricks\autocite{slice-tricks}.

\subsubsection{What is missing from Slices}

This quick glance at slices should clarify that, though the runtime characteristics of lists and slices
can differ, from a usage standpoint, what is possible with lists is also possible with slices.

However, what is missing from Go's slices are a lot of the classical list `helper' functions. In a typical program written in a functional
language, lists take a central role. This results in a number of helper functions\autocite{haskell-list-funcs}
that currently do not exist in Go and would need to be implemented by the programmer.
With no support for polymorphism, the programmer would need to implement a function for every slice-type
that is used. The type \mintinline{go}|[]int| (read: a slice of integers) differs from \mintinline{go}|[]string|
which means that a possible `map' function would have to be
written once to support slices of integers, once to support slices of strings, and a combination of these two:

\begin{gocode}
func mapIntToInt(f func(int) int, []int) []int
func mapIntToString(f func(int) string, []int) []string
func mapStringToInt(f func(string) int, []string) []int
func mapStringToString(f func(string) string, []string) []string
\end{gocode}

With 7 base types (eliding the different `int' types like `int8', `uint16`, `int16', etc.), this would
mean $7^{2} = 49$ map functions just to cover these base types. Counting the different numeric
types into that equation (totally 19 distinct types\autocite{go-basetypes}), would grow that number to $19^{2} = 361$ functions.

Though this code could be generated, it leaves out custom user-defined types, which would still
need to be generated separately.

To mitigate this point, the most common list-operations (in Go slice-operations) will be added to
the compiler, so that the programmer can use these functions on every slice-type.

\section{Existing Work}

With Go's support of some functional aspects, patterns and best practices have emerged that relate
to functional programming.
For example, in the \textit{net/http} package of the standard library, the function
\begin{gocode}
func HandleFunc(pattern string, handler func(ResponseWriter, *Request))
\end{gocode}
is used to register functions for http server handling:

\begin{gocode}
func myHandler(w http.ResponseWriter, r *http.Request) {
    // Handle the given HTTP request
}

func main() {
    // register myHandler in the default ServeMux
    http.HandleFunc("/", myHandler)
    http.ListenAndServe(":8080", nil)
}
\end{gocode}
\autocite{go-http-doc}

Using functions as function parameters or return types is a commonly used feature in Go, not just
within the standard library.

A software design pattern that has gained popularity is `functional options'. The pattern has been
outlined in Dave Cheney's blog post `Functional options for friendly APIs'
and is a great example on how to use the support for multiple paradigms.

A quick example on how functional options are implemented and used can be found in the
appendix~\ref{appendix:funcopts}

Dave Cheney's summary on functional options is thus:
\begin{quote}
    In summary
    \begin{itemize}
        \item Functional options let you write APIs that can grow over time.
        \item They enable the default use case to be the simplest.
        \item They provide meaningful configuration parameters.
        \item Finally they give you access to the entire power of the language to initialize complex values.
    \end{itemize}\autocite{functional-options}
\end{quote}

While this is a great example of what can be done with support for functional concepts, a purely functional approach to
Go has so far been discouraged by the core Go team, which is understandable for a multi-paradigm programming language.
However, multiple developers have already researched and tested Go's ability to do functional programming.

\subsubsection{Functional Go?}

In his talk `Functional Go'\autocite{func-go-talk}, Francesc Campoy Flores analysed some commonly used functional
language features in Haskell and how they can be ported with Go. Ignoring speed and stackoverflows due to non-existent
tail call optimisation\autocite{go-tco}, the main issue was with the type system and the missing polymorphism.

\subsubsection{go-functional}

In July 2017, Aaron Schlesinger, a Go programmer for Micosoft Azure, gave a talk on functional programming wit Go.
He released a repository\autocite{go-functional} that contains `core utilities for functional Programming in Go'.
The project is currently unmaintained, but showcases functional programming concepts like currying, functors and
monoids in Go.
In the `README' file of the repository, he also states that:
\begin{quote}
    Note that the types herein are hard-coded for specific types, but you could
    use code generation to produce these FP constructs for any type you please!
    \autocite{go-functional-readme}
\end{quote}

\section{Work to be done} % TODO: rename...

To conclude, the work that has to be done for this thesis is:

\begin{itemize}
    \item Define which `standard' list functions are most commonly used in functional programming
    \item Implement some of them into the compiler
    \begin{itemize}
        \item at least one of these functions needs to have complete type-checking
        \item demonstrate some usage examples for these functions
    \end{itemize}
    \item Research `rules' for pure functional programming
    \begin{itemize}
        \item for example, this could be exact definitions of immutability and purity
    \end{itemize}
    \item Add these rules to a checker. This could be an already available tool like `go vet', or a newly
    developed, purpose-built utility
\end{itemize}


% DISABLE RELATED WORK AS I PUT THAT INTO THE INTRO
%\IfLanguageName{nswissgerman}{\chapter{Verwandte Arbeit}}{\chapter{Related Work}}
%\label{ch:related-work} % chktex 24
%% -*- mode: latex; coding: utf-8; TeX-master: ../thesis -*-
% !TEX TS-program = pdflatexmk
% !TEX encoding = UTF-8 Unicode
% !TEX root = ../thesis.tex

\todo[inline]{\dots}


\IfLanguageName{nswissgerman}{\chapter{Methoden}}{\chapter{Methodology}}
\label{ch:methodology} % chktex 24
% -*- mode: latex; coding: utf-8; TeX-master: ../thesis -*-
% !TEX TS-program = pdflatexmk
% !TEX encoding = UTF-8 Unicode
% !TEX root = ../thesis.tex

\todo[inline]{%
  \quad -- {
    (Beschreibt die Grundüberlegungen der realisierten Lösung
    (Konstruktion/Entwurf) und die Realisierung als Simulation, als Prototyp
    oder als Software-Komponente)
  } \\
  \quad -- {
    (Definiert Messgrössen, beschreibt Mess- oder Versuchsaufbau,
    beschreibt und dokumentiert Durchführung der Messungen/Versuche)
  } \\
  \quad -- (Experimente) \\
  \quad -- (Lösungsweg) \\
  \quad -- (Modell) \\
  \quad -- (Tests und Validierung) \\
  \quad -- (Theoretische Herleitung der Lösung)
}

\section{Slice Helper Functions}

\subsection{Choosing the functions}

The first task is to implement some helper functions for slices, as they are present for lists in Haskell.
To decide on which functions will be implemented, popular Haskell repositories on Github have been analysed. The
popularity of repositories was decided to be based on their number of stars. Out of all Haskell projects
on Github, the most popular are\cite{github-popular-haskell}:

\begin{itemize}
    \item Shellcheck (koalaman/shellcheck\cite{github-shellcheck}): A static analysis tool for shell scripts
    \item Pandoc (jgm/pandoc\cite{github-pandoc}): A universal markup converter
    \item Postgrest (PostgREST/postgrest\cite{github-postgrest}): REST API for any Postgres database
    \item Semantic (github/semantic\cite{github-semantic}): Parsing, analyzing, and comparing source code across many languages
    \item Purescript (purescript/purescript\cite{github-purescript}): A strongly-typed language that compiles to JavaScript
    \item Compiler (elm/compiler\cite{github-elmcompiler}): Compiler for Elm, a functional language for reliable webapps
    \item Haxl (facebook/haxl\cite{github-haxl}): A Haskell library that simplifies access to remote data, such as databases or web-based services
\end{itemize}

In these repositories, the number of occurrences of popular list functions have been counted. Some caveats with
the analysis:

\begin{itemize}
    \item The analysis did not differentiate between different kinds of functions. For example, `fold' includes
    occurrences of `foldl', `foldr', `foldl\'' and `foldMap'.
    \item The analysis has been done with a simple `grep'. This means it is likely to contain mismatches, for example
    in code comments.
    \item The analysis should only be an \textit{indicator} of what functions are used most.
\end{itemize}

Running the analysis on the 7 repositories listed above, searching for a number of pre-selected list functions, indicates
that the most used functions are `:' (prepend), `map' and `fold', as shown in table~\ref{tab:occurrences-list-funcs}.

\begin{table}[htb]
\centering
\caption{Occurrences of list functions\ref{appendix:function-occurrences}}\label{tab:occurrences-list-funcs}
\begin{tabular}{ll}
\toprule
`:' (prepend) & 2912 \\
\midrule
map & 1241 \\
\midrule
fold & 610 \\
\midrule
filter & 262 \\
\midrule
reverse & 154 \\
\midrule
take & 104 \\
\midrule
drop & 81 \\
\midrule
maximum & 53 \\
\midrule
sum & 44 \\
\midrule
zip & 38 \\
\midrule
product & 15 \\
\midrule
minimum & 10 \\
\midrule
reduce & 8
\end{tabular}
\end{table}

Based on this information, it has been decided to implement the prepend, map and fold functions into the
Go compiler.
% TODO
The source code can be found at `work/go' from the root of this git repository\cite{git-repo}, the work
is done upon Go version 1.14, commit ID `20a838ab94178c55bc4dc23ddc332fce8545a493'.
All files referenced in this chapter are based from the root of the go repository, unless noted otherwise.

\subsection{Implementing Prepend}

\subsubsection{Adding the GoDoc}
In Go, documentation is usually generated directly from comments within the source code\cite{godoc}. This also applies to
builtin functions in the compiler, which have a function stub to document their behaviour\cite{godoc-builtin},
but no implementation, as that is done in the compiler\cite{builtin-impl}.

The append function, for example, is declared as
\begin{gocode}
func append(slice []Type, elems ...Type) []Type
\end{gocode}

So the signature of the prepend function is
\begin{gocode}
func prepend(elem Type, slice []Type) []Type
\end{gocode}

The file containing the documentation is in `\textit{src/builtin/builtin.go}'.
This is already everything that is needed to add the documentation for prepend.

\subsubsection{Adding prepend to the public packages}

The second step is to add prepend to the `\textit{src/go/types}' package.
\begin{quote}
    Package types declares the data types and implements the algorithms for
    type-checking of Go packages
\end{quote}\cite{godoc-types}

Though it will not have a direct impact on the compiler, adding the prepend function
to the \textit{types} package allows external tools like `gopls' (Go's official language server)\cite{gopls} to type-check `.go' files and support the programmer
at writing code.

\subsubsection{Registering the function as a builtin}

To allow type-checking prepend, the function has to be registered as a builtin.
This is done in `\textit{src/go/types/universe.go}':

\begin{gocode}
const (
	// universe scope
	_Append builtinId = iota
	_Prepend
	_Cap
    // ...omitted
)

var predeclaredFuncs = [...]struct {
	name     string
	nargs    int
	variadic bool
	kind     exprKind
}{
	_Append:  {"append", 1, true, expression},
	_Prepend: {"prepend", 2, false, expression},
	_Cap:     {"cap", 1, false, expression},
    // ...omitted
}
\end{gocode}

Prepend takes two arguments and is not a variadic function. The expression kind can either be a `conversion',
`expression' or a `statement'. prepend is an expression, like append, as it returns a value.


The type-checking that is implemented here is not related, in any way, to compiling the source code to machine code, which will do its own type-checking. It is only used for external tools.
To add the type-checking to prepend, a test has to be added too (a separate
test ensures that all builtins are type-checked).

\begin{gocode}
var builtinCalls = []struct {
	name, src, sig string
}{
	// ...omitted
	{"prepend", `var s []int; _ = prepend(0, s)`, `func(int, []int) []int`},
	{"prepend", `type T int; var s []T; var n T; _ = prepend(n, s)`, `func(p.T, []p.T) []p.T`},
	{"prepend", `var s []int; _ = (prepend)(0, s)`, `func(int, []int) []int`},
    // ...omitted
}
\end{gocode}

The type-checking itself is straightforward. The implementation starts by checking that the second
argument is a slice, and then extracts the type and checks that the first argument is of the same type.

% TODO: line length?
\textit{src/go/types/builtins.go}
\inputminted[
    linenos,
    numberfirstline,
    stepnumber=5,
    firstline=135,
    lastline=172,
    breaklines,
    breakanywhere,
    tabsize=4,
    gobble=1,
    bgcolor=bg,
]{go}{../work/go/src/go/types/builtins.go}
\begin{gocode}
func (check *Checker) builtin(x *operand, call *ast.CallExpr, id builtinId) (_ bool) {
	// ...omitted
	switch id {
	case _Prepend:
		// prepend(x T, s S) S, where T is the element type
		// of S.
		// spec: prepend is like append, but adds to the
		// beginning instead of the end of the slice. The
		// values x are passed to a parameter of type T where
		// T is the element type of S and the respective
		// parameter passing rules apply."

		// the second argument is the slice, so we start by
		// getting that type.
		arg(x, 1)
		S := x.typ
		var T Type
		if s, _ := S.Underlying().(*Slice); s != nil {
			T = s.elem
		} else {
			check.invalidArg(x.pos(), "%s is not a slice", x)
			return
		}

		// save the already evaulated argument
		arg1 := *x
		// reset to the first argument
		arg(x, 0)

		// check general case by creating custom signature
		sig := makeSig(S, T, S)
		check.arguments(x, call, sig, func(x *operand, i int) {
			// only evaluate arguments that have not been
			// evaluated before
			if i == 1 {
				*x = arg1
				return
			}
			arg(x, i)
		}, nargs)

		x.mode = value
		x.typ = S
		if check.Types != nil {
			check.recordBuiltinType(call.Fun, sig)
		}
	}
	// ...omitted
}
\end{gocode}

This concludes the type-checking for external tools and makes `gopls' return errors
if the wrong types are used. For example, when trying to prepend an integer to
a string slice.

\begin{gocode}
package main

import "fmt"

func main() {
	fmt.Println(prepend(3, []string{"hello", "world"}))
}
\end{gocode}

Gopls then reports the type-checking error:
\begin{bashcode}
$ gopls check main.go
/tmp/playground/main.go:6:22-23: cannot convert 3 (untyped int constant) to string
\end{bashcode}

\subsubsection{Implementing the function in the compiler}

The compiler is implemented at \textit{src/cmd/compile/internal/gc},
so all mentioned files are located within this directory.

\newglossaryentry{ast}{name=AST, description={Abstract Syntax Tree}}
\newglossaryentry{dag}{name=DAG, description={Directed Acyclic Graph}}

First, prepend needs to be registered as a builtin to parse prepend into
the\gls{ast} (technically, the syntax tree is a syntax \gls{dag}\cite{ast-node-dag},
but this is an implementation detail):


\textit{universe.go}
\begin{gocode}
// ...omitted
var builtinFuncs = [...]struct {
	name string
	op   Op
}{
	{"append", OAPPEND},
	{"prepend", OPREPEND},
	// ...omitted
}
\end{gocode}

`OPREPEND' is an operation that has to be defined in \textit{syntax.go}:

\begin{gocode}
// Node ops.
const (
	OXXX Op = iota
	// ...omitted
	OAPPEND       // append(List); after walk, Left may contain elem type descriptor
	OPREPEND      // prepend(Left, Right)
	OBYTES2STR    // Type(Left) (Type is string, Left is a []byte)
	// ...omitted
)
\end{gocode}

This also shows that an `OPREPEND' node expects its arguments to be in `node.Left' and `node.Right', and not, like `OAPPEND', in `node.List'. This is achieved in
\textit{typecheck.go}. Before typechecking, a node's arguments are always
defined in `node.List`. In the implementation for `OPREPEND', the arguments are typechecked and parsed
into `node.Left' and `node.Right' by

\begin{gocode}
// typecheck the arguments, expand function arguments etc.
typecheckargs(n)
// twoarg ensures there's exactly two arguments and adds
// them to n.Left and n.Right
if !twoarg(n) ...
\end{gocode}

After `node.Left' and `node.Right' are populated, `node.Right'\'s type is checked to be
a slice, and `node.Left' is of the same element type as the slice.
The full typecheck is added as case within \textit{typecheck.go}:

\inputminted[
    linenos,
    numberfirstline,
    stepnumber=5,
    firstline=1579,
    lastline=1613,
    breaklines,
    breakanywhere,
    tabsize=2,
    gobble=1,
    bgcolor=bg,
]{go}{../work/go/src/cmd/compile/internal/gc/typecheck.go}

With this, type-checking prepend in the AST is implemented.
The next step in the Go compiler is escape analysis. This is similar
to the `OAPPEND' node, though `OPREPEND' always has only two arguments:

\textit{escape.go}
\inputminted[
    linenos,
    numberfirstline,
    stepnumber=5,
    firstline=814,
    lastline=820,
    breaklines,
    breakanywhere,
    tabsize=2,
    gobble=2,
    bgcolor=bg,
]{go}{../work/go/src/cmd/compile/internal/gc/escape.go}

The next phase within the compiler are AST transformations to
lower the AST to a more easily compilable form. The first call
is made to `order'. `order' reorders expressions and enforces the
evaluation order. For prepend, this is done by adding another `case'
expression within \mintinline{go}|func (o *Order) expr(n, lhs *Node) *Node|.
Again, this can be implemented similarly to append:

\textit{order.go}
\inputminted[
    linenos,
    numberfirstline,
    stepnumber=5,
    firstline=1182,
    lastline=1188,
    breaklines,
    breakanywhere,
    tabsize=2,
    gobble=1,
    bgcolor=bg,
]{go}{../work/go/src/cmd/compile/internal/gc/order.go}

The compiler now calls `walk' to do more AST transformations, for example replacing
nodes like `OAPPEND' with the actual implementation of the algorithm, in AST form.
This is where the logic for `OPREPEND' needs to be added too.
The general algorithm for `prepend' is:
\begin{gocode}
s := make([]<T>, 1, len(dst)+1)
s[0] = x
append(s, dst...)
\end{gocode}

By allocating the slice with the full length, another slice allocation within
the call to append is saved. The element to prepend is added as the first element
of the slice, and append will then copy the `dst' slice into `s'.
The implementation within `walkprepend' reflects these lines of Go code, but
as AST nodes:

\textit{walk.go}
\inputminted[
    linenos,
    numberfirstline,
    stepnumber=5,
    firstline=2987,
    lastline=3017,
    breaklines,
    breakanywhere,
    tabsize=2,
    bgcolor=bg,
]{go}{../work/go/src/cmd/compile/internal/gc/walk.go}

This `walkprepend' function is called from `walkexpr'. Within `walkepr', it is called
only if the parent node is of the operation `OAS' or `OASOP', as prepend is an expression that always
returns a value, meaning it must be assigned back.

This concludes the implementation of prepend, though all changes can be found by comparing the
changes between the git tag `go1.14'\cite{ba-go-1-14} and `ba-added-prepend'\cite{ba-added-prepend}.

\section{Functional Check}


\chapter{Application}
\label{ch:application} % chktex 24
% -*- mode: latex; coding: utf-8; TeX-master: ../thesis -*-
% !TEX TS-program = pdflatexmk
% !TEX encoding = UTF-8 Unicode
% !TEX root = ../thesis.tex

\section{Refactoring the Prettyprint Package}

\newglossaryentry{stdout}{name=stdout, description={Standard Output, the default
output stream for programs}}

The code blocks~\ref{code:assign-pos} and~\ref{code:func-reassign} have been
generated by a small package `prettyprint' contained in the funcheck repository.

To see how the newly built-in functions and funcheck can be used, this `prettyprint' package
can be refactored to a purely functional version.
The current version of the package is written in what could be considered idiomatic
Go\footnote{
	There is no exact definition of what idiomatic Go is, so this interpretation
	could be challenged. It is idiomatic Go code to the author of this thesis.
}. %TODO?


The prettyprinter is based on the same framework as assigncheck\footnote{Assigncheck
is the main package for funcheck and checks the reassignments}, but instead
of reporting anything, it prints AST information to \gls{stdout}.

Similarly to assigncheck, the main logic of the package is within a
function literal that is being passed to the \mintinline{go}|ast.Inspect|
function.

Prettyprint only checks two AST node types, \mintinline{go}|*ast.DeclStmt|
and \mintinline{go}|*ast.AssignStmt| (declarations and assignments).

For example, for the program
\begin{gocode}
package main

import "fmt"

func main() {
	x, y := 1, 2
	y = 3
	fmt.Println(x, y)
}
\end{gocode}
the following AST information is printed:

\begin{gocode}
Assignment "x, y := 1, 2": 2958101
		Ident "x": 2958101
				Decl "x, y := 1, 2": 2958101
		Ident "y": 2958104
				Decl "x, y := 1, 2": 2958101
Assignment "y = 3": 2958115
		Ident "y": 2958115
				Decl "x, y := 1, 2": 2958101
\end{gocode}

To refactor it to a purely functional version, funcheck can be used to
list reassignments:

\begin{bashcode}
$> funcheck .
prettyprint.go:20:2: internal reassignment (for loop) in "for _, file := range pass.Files { ... }"
prettyprint.go:42:2: internal reassignment (for loop) in "for i := range decl.Specs { ... }"
prettyprint.go:67:2: internal reassignment (for loop) in "for _, expr := range as.Lhs { ... }"
\end{bashcode}
As can be seen in the output, the package uses 3 \mintinline{go}|for| loops to range over
slices. However, there are no other reassignments of variables in the code.

The code to print declarations, which causes the second lint message, is as shown in~\ref{code:decl-printing}.

\begin{code}
\begin{gocode}
func checkDecl(as *ast.DeclStmt, fset *token.FileSet) {
	fmt.Printf("Declaration %q: %v\n", render(fset, as), as.Pos())
	decl, ok := as.Decl.(*ast.GenDecl)
	if !ok {
		return
	}

	for i := range decl.Specs {
		val, ok := decl.Specs[i].(*ast.ValueSpec)
		if !ok {
			continue
		}

		if val.Values != nil {
			continue
		}

		if _, ok := val.Type.(*ast.FuncType); !ok {
			continue
		}

		fmt.Printf("\tIdent %q: %v\n", render(fset, val), val.Names[0].Pos())
	}
}
\end{gocode}
	\caption{Pretty-printing declarations in idiomatic Go\label{code:decl-printing}}
\end{code}
To convert this for-loop appropriately, the new built-in `foldl' can be used.
To recapitulate, the `foldl' function is being defined as:
\begin{gocode}
func foldl(fn func(Type1, Type) Type1, acc Type1, slice []Type) Type1
\end{gocode}
As `foldl' requires a return type, a dummy type `null" can be introduced, which
is just an empty struct:
\begin{gocode}
type null struct{}
\end{gocode}
Now the code within the foor loop can be used to create a function literal:
\begin{gocode}
check := func(_ null, spec ast.Spec) (n null) {
	// implementation
}
\end{gocode}
There are two subtleties in regards to the introduced null type:
First, the null value that is being passed as an argument is being discarded
by the use of an empty identifier.
Secondly, the return value is `named', which means the variable `n' is
already declared in the function block. Because of this, `naked returns' can
be used, so there is no need to specify which variable is being returned.

The snippet~\ref{code:decl-printing} can be translated to

\begin{code}
	\begin{gocode}
func checkDecl(as *ast.DeclStmt, fset *token.FileSet) {
	fmt.Printf("Declaration %q: %v\n", render(fset, as), as.Pos())

	check := func(_ null, spec ast.Spec) (n null) {
		val, ok := spec.(*ast.ValueSpec)
		if !ok {
			return
		}

		if val.Values != nil {
			return
		}

		if _, ok := val.Type.(*ast.FuncType); !ok {
			return
		}

		fmt.Printf("\tIdent %q: %v\n", render(fset, val), val.Names[0].Pos())
		return
	}

	if decl, ok := as.Decl.(*ast.GenDecl); ok {
		_ = foldl(check, null{}, decl.Specs)
	}
}
\end{gocode}
	\caption{Pretty-printing declarations in functional Go}
\end{code}
The for-loop has been replaced by a `foldl', where a function closure
that contains the actual processing is passed.

While this still looks similar to the original example, this is mostly due to
the `if' statements. In Haskell, pattern matching would be used and nil checks
could be omitted entirely. Also, as Haskell's type system is more advanced, the
handling of those types would be different too.

However, the goal of this thesis is to make functional code look more familiar
to programmers that are used to imperative code.
And while it may not look like it, the code does not use any mutation of
variables\footnote{Libraries may do, but the scope is not to rewrite any existing
libraries.}, for loops or global state. Therefore, it can be concluded that this
snippet is purely functional as per the definition from Chapter~\ref{sec:func-purity}.

\section{Quicksort}

In Chapter~\ref{code:haskell-quicksort}, a naive implementation of the Quicksort sorting
algorithm has been introduced.
Implementing this algorithm in Go is now straightforward and the similarities between
the Haskell implementation and the functional Go implementation are striking:

\begin{listing}
	\begin{gocode}
func quicksort(p []int) []int {
	if len(p) == 0 {
		return []int{}
	}

	lesser := filter(func(x int) bool { return p[0] > x }, p[1:])
	greater := filter(func(x int) bool { return p[0] <= x }, p[1:])

	return append(quicksort(lesser), prepend(p[0], quicksort(greater))...)
}
\end{gocode}
\begin{haskellcode}
quicksort :: Ord a => [a] -> [a]
quicksort []     = []
quicksort (p:xs) = (quicksort lesser) ++ [p] ++ (quicksort greater)
    where
        lesser  = filter (< p) xs
        greater = filter (>= p) xs
\end{haskellcode}
	\caption{Quicksort implementations compared}
\end{listing}

Again, the Go implementation bridges the gap between being imperative and functional,
while still being obvious about the algorithm.
Furthermore, as expected, when inspecting the code with funcheck, no non-functional
constructs are reported.

\section{Comparison to Java Streams}

In Java 8, concepts from functional programming have been introduced to the language.
The major new feature was Lambda Expressions --- anonymous function literals --- and
streams. Streams are an abstract layer to process data in a functional way, with `map',
`filter', `reduce' and more.

Java Streams are similar to the new built-in functions in this thesis:

\begin{listing}
	\begin{javacode}
List<Integer> even = list.stream()
	.filter(x -> x % 2 == 0)
	.collect(Collectors.toList());
	\end{javacode}
	\begin{gocode}
even := filter(
	func(x int) bool { return x%2 == 0 },
	list)
	\end{gocode}
	\caption{Comparison Java Streams and functional Go}
\end{listing}

The lambda-syntax in Java is more concise than Go's function literals, where the
complete function header has to be provided\footnote{There is an open proposal
	to add a lightweight anonymous function syntax to Go 2, which, if implemented,
would resolve this verbosity\autocite{go-lambdas}}.
%Before lambdas have been introduced in Java 8, the way to achieve this was through
%anonymous inner classes. This means that due to the historic growth of Java, there
%are currently two methods to define an anonymous functions, which may cause inconsistencies
%in the code. However, anonymous inner classes are extremely verbose if the goal is
%to introduce only one function\autocite{java-lambda-expressions}:

%\begin{javacode}
%printPersons(
	%roster,
	%new CheckPerson() {
		%public boolean test(Person p) {
			%return p.getGender() == Person.Sex.MALE
				%&& p.getAge() >= 18
				%&& p.getAge() <= 25;
		%}
	%}
%);
%\end{javacode}

%Already because of the overhead in typing, most programmers will prefer to use
%lambdas.
%%If this is used, Go's function literal syntax has an edge over Java's when it comes to
%%readability and conciseness.

% TODO: Is this really about only one list type?
However, the conversion to a stream and back to a list (with \mintinline{java}|list.stream()| and
\mintinline{java}|.collect(Collectors.toList())|)
is not required in Go, as the operations all work on slices. Here, only having a single
list-like type built into the language is an advantage, as the (at least syntactical)
overhead to convert the list only to run a `filter' function can be avoided.

Apart from syntactical differences, Java Streams contain all the functions that
have been added as built-ins to Go too, and a lot more.

On the other hand, Java's Syntax is arguably more complex than Go. An indicator for this might be
the language specification; Go's Language Specification is roughly 110 pages, while
Java's specification spans more than 700 pages\footnote{
	The Java 8 Specification is 724\autocite{java-8-spec}, the Java 14
	Specification 774\autocite{java-14-spec} pages.}, more than 6 times the size.

The consideration of which language to choose comes down to the experience with either language.
An experienced Java programmer will find it easier to start with Java's toolset, while programmers
coming from a C background may choose Go over Java.


\IfLanguageName{nswissgerman}{\chapter{Resultate}}{\chapter{Experiments and Results}}
\label{ch:results} % chktex 24
% -*- mode: latex; coding: utf-8; TeX-master: ../thesis -*-
% !TEX TS-program = pdflatexmk
% !TEX encoding = UTF-8 Unicode
% !TEX root = ../thesis.tex

To learn functional programming without being introduced to a new syntax at the same time
ensures that programmers can fully concentrate on functional concepts. Although Go already
supported a functional programming style, the programmer may not have known if the code is
purely functional or if there are still imperative constructs embedded.

In the last chapters, functional purity has been defined as a law based on two rules; immutability
and function purity. Immutability means that once assigned, a variable's value never
changes. Function purity entails that functions do not have side effects and their return
value is solely influenced by the function's parameters.

It has been shown that although purely functional languages like Haskell aim to be completely
pure, this objective is difficult to accomplish. The reason for this are Input / Output actions; user
input, network connections, randomness and time are all impure. Haskell wraps
these impure functions in the IO monad, which is a way to work around the compiler's optimisations
based on functional purity. While the IO monad does not make impure functions pure, it
does serve as documentation to it's users (`if the function has IO, it is impure') and
guarantees a certain execution order.

Go on the other hand does not have this issue. The Go compiler does not optimise execution
based on purity guarantees. Having a similar construct like the IO monad in Go would as such
only serve documentation purposes. Because of this, the decision has been taken to ignore
the impurity that is implied with IO actions.

Apart from IO, to achieve functional purity, the global state of a program should not influence
the return values of specific function. This ties into immutablitiy; if global state can
not be mutated, it can also not influence or change the result value of a function.

Based on these observations, a static code analysis tool has been developed that reports
all reassignments of variables. In other words, it forbids the usage of the regular
assignment operator (\mintinline{go}|=|), only allowing the short variable declarations
(\mintinline{go}|:=|). However, the experienced Go developer may know that the \mintinline{go}|:=|
operator can also reassign previously declared variables, implying that the solution to the
problem is not as simple as forbidding the assignment operator.
Further, there are many more edge cases that have been detected with careful testing:
To recursively call function literals, they must be declared beforehand (before assigning
the actual function to it) because of Go's scoping rules. Additionally, exceptions
had to be made for the blank identifier (\mintinline{go}|_|) and variables that are declared
outside of the current file.

With all of this in place, an algorithm has been chosen that is based on the identifier's
declaration position. In the \gls{ast} that is being checked, every identifier node has a field
which contains the position of it's declaration. If this does not match the current identifier's
position, the operation must be a reassignment.
The resulting binary, called `funcheck', successfully reports such reassignments:

\begin{gocode}
s := "hello world"
fmt.Println(s)
s = "and goodbye"
fmt.Println(s)
\end{gocode}

\begin{bashcode}
\$> funcheck .
file.go:3:2: reassignment of s
\end{bashcode}

This linter can be used and ran against any Go package. To eliminate the reported errors,
code has to be rewritten in what ends up being purely functional code.

However, functional code often relies heavily on lists and list-processing functions.
Although Go does not have a built-in list datatype, Go's slices, an abstraction on arrays,
mitigate a lot of downsides when comparing regular arrays to lists\footnote{Arrays / Slices
and Lists have a completely different runtime behaviour (indexing, adding or removing elements).
However, the performance of the code was not considered to be relevant in this thesis.}.

What Go's slices lack on the other side are the typical higher-order functions like `map',
`filter' or `reduce'. These are commonly used in functional programming and most languages
contain implementations of these functions already --- Go does not.

Due to the lack of polymorphism, writing implementations for these functions
would result in a lot of duplicated code. To mitigate this issue, the most
common higher-order functions have been added to the list of Go's built-in functions,
which are registered, type-checked and implemented within the Go compiler.
As these are handled directly at compile time, built-in functions may be polymorphic, for
example allowing the programmer to use the same `filter' function for all list-types.

To determine which higher-order functions are most commonly used, we analysed the most
popular open-source Haskell projects (pandoc, shellcheck and purescript, to name a few).
As a result, `fmap', `fold', `filter' and `prepend'
(`cons') have been added as built-ins into the compiler.
These functions make it easier to write purely functional code in Go, in turn helping
the programmer to learn functional programming with a familiar language and syntax.

While implementing these functions in a regular Go program would be a matter of minutes,
adding them to the Go compiler is more effort. To illustrate, the functions
have been written out in regular Go in the chapters~\ref{ch:impl-fmap} to~\ref{ch:impl-filter}
and are 33 lines of code, all functions combined. In the Go compiler, it is necessary to
register the functions, type-check the calls and manipulate the \gls{ast} instead of writing
the algorithm in Go code directly. This took more than 800 lines of code to do so.

As a result, using these functions is equal to using any other built-in function: there
is documentation in Godoc, type-checking support in the language server\footnote{If the
language server (gopls) is compiled against the modified version of Go\ref{appendix:build-gopls}}
and in the compiling phase, as well as a polymorphic function header, allowing the
programmer to call the function with any allowed type.

A demonstration of these functions and how functional Go code looks like can be found at~\ref{code:funcexample}

\begin{listing}[ht]
\begin{code}
	\captionof{listing}{Demonstration of the new built-in functions}
	\label{code:funcexample}
\begin{gocode}
package main

import (
	"fmt"
	"strconv"
)

type parity bool

const (
	even parity = true
	odd  parity = false
)

// shouldBe returns a function that returns true if an int is of the
// given parity
func shouldBe(p parity) func(i int) bool {
	return func(i int) bool {
		return (i%2 == 0) == p
	}
}

func main() {
	l := []int{1, 2, 3, 4, 5}
	l5 := fmap(func(i int) int { return i * 5 }, prepend(0, l))

	// fold over even / odd numbers and add them to a string
	evens := foldl(
		func(s string, i int) string { return s + strconv.Itoa(i) + " " },
		"even: ",
		filter(shouldBe(even), l5),
	)
	odds := foldl(
		func(s string, i int) string { return s + strconv.Itoa(i) + " " },
		"odd: ",
		filter(shouldBe(odd), l5),
	)

	fmt.Println(evens, odds) // even: 0 10 20  odd: 5 15 25
}

\end{gocode}
\end{code}
\end{listing}

With these additions to Go and its ecosystem, aspiring functional programmers
can fully concentrate of the concepts of functional programming while keeping
a familiar syntax at hand. However, it should not be considered a fully featured
purely functional programming language. Rather, it should serve as a starting point and
make the transition to a language like Haskell easier.


\IfLanguageName{nswissgerman}{\chapter{Diskussion}}{\chapter{Discussion}}
\label{ch:discussion} % chktex 24
% -*- mode: latex; coding: utf-8; TeX-master: ../thesis -*-
% !TEX TS-program = pdflatexmk
% !TEX encoding = UTF-8 Unicode
% !TEX root = ../thesis.tex

The aforementioned extensions to the Go language and its tooling should be a help to learn
functional programming. I believe that through these extensions it is easier to write
purely functional code in Go, enabling a developer to learn functional programming with
a familiar syntax in an obvious way. Here, Go's simplicity and verbosity are a key differentiator
to other languages. Instead of having as many features as possible to support every usecase,
Go has been designed with simplicity in mind\footnote{For example, Go only has 25 keywords, compared
to 37 in C99 and 84 in C++11}.

In many cases, this leads to more `verbose' code --- more lines of code compared to a similar
implementation in other languages. However, I argue that, especially for the first steps
in functional programming,

\begin{quote}
Clear is better than clever\autocite{cheney-clear}
\end{quote}

Staying in touch with this core Go principle, this results in functional code that may be
verbose, but easy to read and understand.

It should be clear that the result is not a `production-ready' functional programming language.
It is a language to help getting started with functional programming; either by re-implemeting pieces
of code that have not been clear in how they work, or by taking an imperative block of code
and refactoring it to make it purely functional.

In many cases, the resulting code will still look familiar to the imperative counterpart,
even if `funcheck' assures that it is purely functional. This, I believe, bridges the gap
that developers usually have to overcome by themselves.

To be a purely functional programming language, Go is missing too many features that would be
required to write concise functional code. The very basic type system\footnote{Not only
	does Go not have polymorphism (yet), Go's type system is simple by design: there are no
	implicit type conversions, no sum types (tagged unions, variant) and almost no type inference.
}, no advanced pattern matching and only explicit currying are all examples why Go is not useful
in day-to-day functional programming.

At the same time, the obvious nature of Go is exactly because it is missing all
of these features. The Go team explicitly tries not to include too many features within
the language in order to keep the complexity of code to a minimum\autocite{go-feature}.
The simplicity of the language is a key feature of Go and an important reason why it was
chosen to implement the ideas in the first place.
Especially for learning new concepts, hiding implementations and ideas behind features
may not be what is desired and helpful.

On another note, what has not been an aspect in this thesis is
performance. Go by itself is relatively performant, however functional constructs, for example
recursive function calls, come with a performance cost. While in purely functional languages
this can be optimised, Go cannot or does not want to do these optimisations\footnote{For example,
with tail call optimisation, the Go team explicitly decided not to do it because the stack trace
would be lost}. Regardless, the performance of a language is not as important if it
is not used in a production environment, which is why it was never a criteria in the first place.

\newglossaryentry{sumtypes}{name=sum types,description={Sum types, often also called aggregated types,
variant or tagged union is a data structure that can hold one of several, predefined data types. For
example, Haskell's \mintinline{haskell}|Either| holds either a value of type A or type B. Similar to that,
\mintinline{haskell}|Maybe| can hold either a concrete value, or `Nothing'}}

The number one issue that still exists is the simple type system. Not only the lack of polymorphism, which
has been mitigated slightly by providing the most used higher-order functions as built-ins, but also the
lack of algebraic data types, especially \gls{sumtypes}.

Algebraic data types can be split up into two groups, product types and sum types.
Most product types can be built with Go too; records are basically
equal to structs, and tuples are not needed too often, as functions can just return multiple values.
Sum types however are not available in Go at all. It is possible to imitate sum types in Go
with interfaces (see the example code in~\ref{code:funcexample}), but the compiler does not ensure that
all cases are matched against\footnote{There is tooling available to check this\autocite{sushi-sumtypes},
	but it is third party and not built into the language}. This may be an interesting area for further
research and implementation possibilities.



%%%%%%%%%%%%%%%%%%%%%%%%%%%%%%%%%%%%%%%%
\backmatter % chktex 1

% \let\clearpage\relax
% \vspace{-4em}
\printbibliography
% \endgroup


% \begingroup
% \let\clearpage\relax
% \vspace{-4em}
\listoffigures
% \endgroup

% \begingroup
% \let\clearpage\relax
% \vspace{-4em}
\listoftables
% \endgroup
\cleardoublepage % chktex 1


%%%%%%%%%%%%%%%%%%%%%%%%%%%%%%%%%%%%%%%%
\appendix

% - Add your appendix here:

\todo[inline]{
  Anhang/Appendix:

  \quad -- Projektmanagement: \\ % chktex 8
  \qquad -- Offizielle Aufgabenstellung, Projektauftrag \\ % chktex 8
  \qquad -- (Zeitplan) \\ % chktex 8
  \qquad -- (Besprechungsprotokolle oder Journals) % chktex 8

  \quad -- Weiteres: \\ % chktex 8
  \qquad -- CD/USB-Stick mit dem vollständigen Bericht als PDF-File inklusive Film- und Fotomaterial \\ % chktex 8
  \qquad -- (Schaltpläne und Ablaufschemata) \\ % chktex 8
  \qquad -- (Spezifikation u. Datenblätter der verwendeten Messgeräte und/oder Komponenten) \\ % chktex 8
  \qquad -- (Berechnungen, Messwerte, Simulationsresultate) \\ % chktex 8
  \qquad -- (Stoffdaten) \\ % chktex 8
  \qquad -- (Fehlerrechnungen mit Messunsicherheiten) \\ % chktex 8
  \qquad -- (Grafische Darstellungen, Fotos) \\ % chktex 8
  \qquad -- (Datenträger mit weiteren Daten (z. B. Software-Komponenten) inkl. Verzeichnis der auf diesem Datenträger abgelegten Dateien) \\ % chktex 8
  \qquad -- (Softwarecode) % chktex 8
}

\end{document}
