% -*- mode: latex; coding: utf-8; TeX-master: ../thesis -*-
% !TEX TS-program = pdflatexmk
% !TEX encoding = UTF-8 Unicode
% !TEX root = ../thesis.tex

\IfLanguageName{nswissgerman}{\section{Ausgangslage}}{\section{Starting point}}
\label{sec:starting-point}

\todo[inline]{%
  \quad -- Nennt bestehende Arbeiten/Literatur zum Thema - Literaturrecherche \\
  \quad -- Stand der Technik: Bisherige Lösungen des Problems und deren Grenzen \\
  \quad -- (Nennt kurz den Industriepartner und/oder weitere Kooperationspartner und dessen/deren Interesse am Thema Fragestellung)
}


\IfLanguageName{nswissgerman}{\section{Zielsetzung}}{\section{Objective}}
\label{sec:objective}

\todo[inline]{%
  \quad -- Formuliert das Ziel der Arbeit \\
  \quad -- Verweist auf die offizielle Aufgabenstellung des/der Dozierenden im Anhang \\
  \quad -- (Pflichtenheft, Spezifikation) \\
  \quad -- (Spezifiziert die Anforderungen an das Resultat der Arbeit) \\
  \quad -- (Übersicht über die Arbeit: stellt die folgenden Teile der Arbeit kurz vor) \\
  \quad -- (Angaben zum Zielpublikum: nennt das für die Arbeit vorausgesetzte Wissen) \\
  \quad -- (Terminologie: Definiert die in der Arbeit verwendeten Begriffe)
}
